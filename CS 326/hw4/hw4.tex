\documentclass[ 12pt ]{article}

\usepackage{amsmath}
\usepackage{amssymb}
\usepackage{cancel}
\usepackage{tikz}
\usepackage{listings}
\usepackage{mathdots}
\usepackage[margin=.75in]{geometry}

\begin{document}

% title page
\title{%
	Homework 4 \\
	\large CS 326 \\
	Section 1001}
\author{Landon Fox}
\date{March 5, 2020}
\maketitle
\newpage

\lstset{morecomment=[l][keywordstyle]{;}, keywords={define, and, or, cons, car, cdr, if, else, cond, append, list}}

\begin{itemize}
	% problem 1
	\item[] {1) \large}
	\begin{itemize}
		% problem 1a
		\item[] a)
		\begin{flalign}
			b\, b\, b\, \times\, 4\, a\, \times\, c\, \times\, -\, sqrt\, +\, 2\, a\, \times\, / \nonumber
		\end{flalign}

		% problem 1b
		\item[] b)
		\begin{flalign}
			/\, +\, b\, sqrt\, -\, \times\, b\, b\, \times\, \times\, 4\, a\, c\, \times\, 2\, a \nonumber
		\end{flalign}
	\end{itemize}

	% problem 2
	\item[] {2) \large}
	\begin{itemize}
		% problem 2a
		\item[] a)
		\begin{flalign}
			&A\, and\, B \nonumber \\
			&if\, A\, then\, B\, else\, false \nonumber
		\end{flalign}

		% problem 2b
		\item[] a)
		\begin{flalign}
			&A\, or\, B \nonumber \\
			&if\, A\, then\, true\, else\, B \nonumber
		\end{flalign}
	\end{itemize}

	% problem 3
	\item[] {3) \large}
	\begin{itemize}
		% problem 3a
		\item[] a)
		\begin{lstlisting}[language=C]
    line = read_line();
    while ( !all_blanks( line ) )
    {
        process_line( line );
        line = read_line();
    }
		\end{lstlisting}

		% problem 3b
		\item[] b)
		\begin{lstlisting}[language=C]
    line = read_line();
    if ( !all_blanks( line ) )
    {
        do
        {
            process_line( line );
            line = read_line();
        } while ( !all_blanks( line ) );
    }
		\end{lstlisting}
	\end{itemize}
	\newpage

	\item[] {4) \large}
	\begin{lstlisting}
    ( define ( factorial n )
        ( fac_helper n 2 1 )
    )

    ( define ( fac_helper n i p )
        ( if ( < ( - n i ) 0 )
            p
            ( fac_helper n ( + i 1 ) ( * p i ) )
        )
    )
	\end{lstlisting}

	\item[] {5) \large}
	\begin{itemize}
		\item[] a)
		Inline performs faster than macro:
		\begin{lstlisting}[language=C]
    #define CUBE( x ) x*x*x
    inline int cube( x ) { return x*x*x; }
		\end{lstlisting}
		The inline performs faster because for each use of the parameter, the inline will use applicable analysis.

		\item[] b)
		Macro performs faster than inline:
		\begin{lstlisting}[language=C]
    #define PI 3.1415
    inline double pi() { return 3.1415; }
		\end{lstlisting}
		The macro performs faster because it uses the preprocessor to insert the value rather than a function call.
	\end{itemize}

\end{itemize}

\end{document}