\documentclass[ 12pt ]{article}

\usepackage{amsmath}
\usepackage{amssymb}
\usepackage{cancel}
\usepackage{tikz}
\usepackage{listings}
\usepackage{enumitem}
\usepackage[margin=.75in]{geometry}

\begin{document}

\title{Homework 2}
\author{Landon Fox}
\date{February 18, 2020}

\begin{flushleft}
Landon Fox \\
CS 326 \\
Section 1001 \\
February 18, 2020
\end{flushleft}
\begin{center}
{\Large Homework 2}
\end{center}

\lstset{morecomment=[l][keywordstyle]{;}, keywords={define, and, or, cons, car, cdr, if, else, cond, append, list}}

\begin{itemize}
  % problem 1
  \item[] {\large 1)}
  \newline
  \begin{lstlisting}
    ; problem 1: returns an identical list except for every occurrence
    ; of x has been replaced with y
    ( define ( subst x y L )
      ( if ( null? L )
        L
        ( cons
          ( if ( eqv? ( car L ) x )
            y
            ( car L )
          )
          ( subst x y ( cdr L ) )
        )
      )
    )
  \end{lstlisting}

  % problem 2
  \item[] {\large 2)}
  \newline
  \begin{lstlisting}
    ; problem 2: determines if a list has unique elements
    ( define ( all-different? L )
      ( cond
        [ ( null? ( cdr L ) ) #t ]
        [ ( eqv? ( car L ) ( car ( cdr L ) ) ) #f ]
        [ else
          ( and
            ( all-different? ( cons ( car L ) ( cdr ( cdr L ) ) ) )
            ( all-different? ( cdr L ) )
          )
        ]
      )
    )
  \end{lstlisting}
  \newpage

  % problem 3
  \item[] {\large 3)}
  \newline
  \begin{lstlisting}
    ( define ( val T )
      ( car T )
    )

    ( define ( left T )
      ( car ( cdr T ) )
    )

    ( define ( right T )
      ( car ( cdr ( cdr T ) ) )
    )
  \end{lstlisting}

  % problem 3a
  \item[] {\large 3a)}
  \newline
  \begin{lstlisting}
    ; problem 3a: returns number of nodes in tree
    ( define ( n-nodes T )
      ( cond
        [ ( null? T ) 0 ]
        [ ( and ( null? ( left T ) ) ( null? ( right T ) ) ) 1 ]
        [ ( null? ( left T ) ) ( + 1 ( n-nodes ( right T ) ) ) ]
        [ ( null? ( right T ) ) ( + 1 ( n-nodes ( left T ) ) ) ]
        [ else
          ( +
            1
            ( n-nodes ( left T ) )
            ( n-nodes ( right T ) )
          )
        ]
      )
    )
  \end{lstlisting}
  \newpage

  % problem 3b
  \item[] {\large 3b)}
  \newline
  \begin{lstlisting}
    ; problem 3b: returns the number of leaves in tree
    ( define ( n-leaves T )
      ( cond
        [ ( null? T )  0 ]
        [ ( and ( null? ( left T ) ) ( null? ( right T ) ) ) 1 ]
        [ ( null? ( left T ) ) ( n-leaves ( right T ) ) ]
        [ ( null? ( right T ) ) ( n-leaves ( left T ) ) ]
        [ else
          ( +
            ( n-leaves ( left T ) )
            ( n-leaves ( right T ) )
          )
        ]
      )
    )
  \end{lstlisting}

  % problem 3c
  \item[] {\large 3c)}
  \begin{lstlisting}
    ; problem 3c: returns the height of the tree
    ( define ( height T )
      ( cond
        [ ( null? T )  0 ]
        [ ( and ( null? ( left T ) ) ( null? ( right T ) ) ) 1 ]
        [ ( null? ( left T ) ) ( + 1 ( height ( right T ) ) ) ]
        [ ( null? ( right T ) ) ( + 1 ( height ( left T ) ) ) ]
        [
          ( >
            ( height ( left T ) )
            ( height ( right T ) )
          )
            ( + 1 ( height ( left T ) ) )
        ]
        [ else
          ( + 1 ( height ( right T ) ) )
        ]
      )
    )
  \end{lstlisting}
  \newpage

  % problem 3d
  \item[] {\large 3d)}
  \begin{lstlisting}
    ; problem 3d: return postorder traversal of tree
    ( define ( postorder T )
      ( append
        ( if ( null? ( left T ) )
          '()
          ( postorder ( left T ) )
        )
        ( if ( null? ( right T ) )
          '()
          ( postorder ( right T ) )
        )
        ( list ( val T ) )
      )
    )
  \end{lstlisting}

  % problem 4
  \item[] {\large 4)}
  \begin{lstlisting}
    ; problem 4: returns all elements in list including sublists
    ( define ( flatten L )
      ( if ( null? L )
        L
        ( append
          ( if ( list? ( car L ) )
            ( flatten ( car L ) )
            ( list ( car L ) )
          )
          ( flatten ( cdr L ) )
        )
      )
    )
  \end{lstlisting}

  % problem 5
  \item[] {\large 5)}
  \begin{lstlisting}
    ; problem 5: determines if element is a member of bst
    ( define ( member-bst? V T )
      ( cond
        [ ( null? T ) #f ]
        [ ( eqv? V ( val T ) ) #t ]
        [ ( < V ( val T ) ) ( member-bst? V ( left T ) ) ]
        [ ( > V ( val T ) ) ( member-bst? V ( right T ) ) ]
      )
    )
  \end{lstlisting}
\end{itemize}

\end{document}
