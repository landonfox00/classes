\documentclass[ 12pt ]{article}

\usepackage{amsmath}
\usepackage{amssymb}
\usepackage{cancel}
\usepackage{tikz}
\usepackage{listings}
\usepackage{enumerate}
\usepackage[margin=0.5in]{geometry}
\footskip = -0.75in

\begin{document}

% title page
\title{Homework 5}
\author{Landon Fox}
\date{April 9, 2020}

\begin{flushleft}
Landon Fox \\
CS 326 \\
Section 1001 \\
April 9, 2020
\end{flushleft}
\begin{center}
\Large Homework 5
\end{center}

\begin{itemize}
	% problem 1
	\item[] {\large 1)}
	Let us consider the size of a single record. The record has the following
	items: $short$, $char$, $short$, $char$, $real$, $integer$ in that respective
	order. Knowing that addresses have the following form $n \cdot sizeof( DATA\_TYPE )$;
	we will have $s$ followed by $c$, then free space until the forth byte where $t$ is
	located, followed by $d$. Now we must place our $real$ value $r$, the next multiple
	of the size of a $real$ is 8 where $r$ is located then finally $i$ proceeds it.
	Resulting in a total of 20 bytes for a single record. \\
	Therefore 200 bytes are required to allocate an array of 10 records. \\

	% problem 2
	\item[] {\large 2)}
	\begin{itemize}
		% problem 2a
		\item[] {\large 2a)}
		Structural equiviliance: $a,\, b,\, c,\, d$ \\

		% problem 2b
		\item[] {\large 2b)}
		Strict name equiviliance: $a,\, b$ \\

		% problem 2c
		\item[] {\large 2c)}
		Loose name equiviliance: $a,\, b,\, c$ \\
	\end{itemize}

	% problem 3
	\item[] {\large 3)}
	The program produces a run-time error because it attempts to use and free
	memory of a dangling pointer. The pointer $c$ never is assigned to the $Cell$
	memory. When $c$ is passed into $AllocateCell$, it is a pass by value. $q$ 
	is then assigned to the memory address, afterwards $q$ is immediately destroyed, 
	never interacting with $c$. To fix this, allow $c$ to be passed by reference.
	\begin{lstlisting}[language=C]
void AllocateCell( Cell** q )
{
	q* = ( Cell* ) malloc ( sizeof( Cell ) );
}

void main()
{
	Cell* c;
	AllocateCell( &c );
	c->a = 1;
	free( c );
}
	\end{lstlisting}

	% problem 4
	\item[] {\large 4)}
	Since this is a 32-bit system, we know that $char$ and $int$ have a size of
	one and four bytes respectively. Also addresses have the form $4n$, where ours
	starts specifically at 1000. Each struct in our array takes up 8 bytes due
	to the 5 from the $int$ and $char$, the rest is free space due to the next
	multiple of 4. The two dimensional array is one dimensional contiguous array
	therefore it can be treated as so. \\
	To find the address of $A[3][7]$ we have the following \\
	$( col \cdot i + j ) \cdot sizeof( A[0][0] ) + off = ( 10 \cdot 3 + 7 ) \cdot 8 + 1000 = 1296$ \\

	% problem 5
	\item[] {\large 5)}
	A union in C stores only one allocation of memory that can be viewed as many
	different types depending on the variable that is being accessed. This
	allows data of one type to be viewed as another without casting. The below
	code illustrates this with primitive data types. For instance a $char$ can be
	placed in a union to find the respective ascii value when viewed as an $int$.
	\begin{lstlisting}[language=C]
#include <stdio.h>

union Type_Conv
{
	char c;
	short s;
	int i;
	long l;
	float f;
} tc;

int main()
{
	tc.c = 21;
	printf( "%d", tc.i );
	return 0;
}
	\end{lstlisting}
\end{itemize}

\end{document}