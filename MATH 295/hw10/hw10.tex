\documentclass[ 12pt ]{article}

\usepackage{amsmath}
\usepackage{amssymb}
\usepackage{cancel}
\usepackage{tikz}
\usepackage{listings}
\usepackage{enumerate}
\usepackage[margin=0.5in]{geometry}
\usepackage{mathtools}

\newcommand\Myperm[2][^n]{\prescript{#1\mkern-2.5mu}{}P_{#2}}
\newcommand\Mycomb[2][^n]{\prescript{#1\mkern-0.5mu}{}C_{#2}}
\footskip = -0.75in

\begin{document}

% title page
\title{Homework 10}
\author{Landon Fox}
\date{May 4, 2020}

\begin{flushleft}
Landon Fox \\
Math 295 \\
Section 1001 \\
May 4, 2020
\end{flushleft}
\begin{center}
\Large Homework 10
\end{center}

\begin{itemize}
	% problem 1
	\item[] {\large 1)}
	Use the definition of the limit of a sequence to show that 
	$\left \{ \frac{n^2+n+1}{3n^2+1} \right \} \rightarrow \frac{1}{3}$. \\
	Limit of a sequence definition:
	$\lim_{n \rightarrow \infty} x_n = L \rightarrow \forall \epsilon > 0,\, \exists N \in \mathbb{N}\; s.t.\; \forall n>N,\, |x_n - L|<\epsilon$. \\
	\begin{flalign}
		\left | \frac{n^2+n+1}{3n^2+1} - \frac{1}{3} \right | &< \epsilon \nonumber \\
		\left | \frac{3n+2}{9n^2+3} \right | &< \epsilon \nonumber
	\end{flalign}
	Let us observe that $3n+2 \leq 3n+2n$, clearly this is true for all $n>0$, $N \in \mathbb{N}$
	therefore this is always true. \\
	Similiarly it's true that $9n^2+3 \geq 9n^2$ for all $n$. Thus
	\begin{flalign}
		\left | \frac{3n+2}{9n^2+3} \right | < \left | \frac{3n+2n}{9n^2} \right | \nonumber
	\end{flalign}
	Let us attempt to show that the right hand side of the inequality can be less than any
	threshold $\epsilon$.
	\begin{flalign}
		\left | \frac{5n}{9n^2} \right | &< \epsilon \nonumber \\
		\frac{5}{9n} &< \epsilon \nonumber \\
		n &> \frac{5}{9\epsilon} \nonumber \\
		N &= \left \lceil \frac{5}{9\epsilon} \right \rceil \nonumber
	\end{flalign}
	Since $N$ from the greater inequality can be constructed for any $\epsilon$, it implies that
	$N$ can also be constructed for the original inequality, thus the limit exists. $\blacksquare$

	% problem 2
	\item[] {\large 2)}
	\begin{itemize}
		% problem 2i
		\item[] {\large 2i)}
		Use Limit laws to find the limit of the sequence $x_n= \sqrt{n^2+n}-n$. \\
		First, algebraic manipulation
		\begin{flalign}
			x_n &= \sqrt{n^2+n}-n \nonumber \\
			&= \frac{\sqrt{n} (\sqrt{n+1}-\sqrt{n})(\sqrt{n+1}+\sqrt{n})}{\sqrt{n+1}+\sqrt{n}} \nonumber \\
			&= \frac{\sqrt{n}}{\sqrt{n+1}+\sqrt{n}} \nonumber \\
			&= \frac{1}{\sqrt{\frac{n+1}{n}}+\sqrt{\frac{n}{n}}} \nonumber \\
			x_n &= \frac{1}{\sqrt{1+\frac{1}{n}}+1} \nonumber
		\end{flalign}
		Applying Limit laws
		\begin{flalign}
			\lim_{n \rightarrow \infty} x_n &= \lim_{n \rightarrow \infty} \frac{1}{\sqrt{1+\frac{1}{n}}+1} \nonumber \\
			&= \frac{1}{\lim_{n \rightarrow \infty} \left (\sqrt{1+\frac{1}{n}}+1 \right )} \;\;\; Quotient\; Law \nonumber \\
			&= \frac{1}{\lim_{n \rightarrow \infty} \sqrt{1+\frac{1}{n}}+1} \;\;\; Sum\; Law \nonumber \\
			&= \frac{1}{\sqrt{\lim_{n \rightarrow \infty} \left ( 1+\frac{1}{n} \right )}+1}\;\;\; Root\; Law \nonumber \\
			&= \frac{1}{\sqrt{1+\lim_{n \rightarrow \infty} \frac{1}{n}}+1}\;\;\; Sum\; Law \nonumber \\
			&= \frac{1}{\sqrt{1+ 0}+1}\;\;\; Evaluate\; limit \nonumber \\
			\lim_{n \rightarrow \infty} x_n &= \frac{1}{2} \nonumber
		\end{flalign}

		% problem 2ii
		\item[] {\large 2ii)}
		Use Squeeze Law to find the limit of the sequence
		$x_n = \frac{1}{(n+1)^2} + \frac{1}{(n+2)^2} + \hdots +\frac{1}{(2n)^2}$. \\
		Let us attempt to find convergent sequences $a_n$ and $b_n$ such that
		$\forall n,\; a_n \leq x_n \leq b_n$ and
		$\lim_{n \rightarrow \infty}a_n=\lim_{n \rightarrow \infty}b_n$. \\
		Let $a_n=\sum_{k=1}^n \frac{1}{(2n)^2}$. For every term in $a_n$, clearly it is less
		than or equal to every term in $x_n$ for any $n$, therefore we can conclude that
		$\forall n>0,\; a_n \leq x_n$. Similarly let $b_n = \sum_{k=1}^n \frac{1}{n^2}$.
		By the same argument it can be seen that every term in $b_n$
		is greater than $x_n$; $\forall n>0,\, x_n \leq b_n$. Therefore
		$\forall n>0,\; a_n \leq x_n \leq b_n$. \\
		Let us also observe that $\lim_{n \rightarrow \infty} a_n = \lim_{n \rightarrow \infty} b_n = 0$.
		This is true because $a_n=\sum_{k=1}^n \frac{1}{(2n)^2} = \frac{n}{4n^2} = \frac{1}{4n}$ and
		$b_n = \sum_{k=1}^n \frac{1}{n^2}=\frac{n}{n^2}=\frac{1}{n}$; clearly the limit of both
		sequences evaluates to 0. Thus by the Squeeze Law, $\lim_{n \rightarrow \infty} x_n = 0$.
	\end{itemize}

	% problem 3
	\item[] {\large 3)}
	Discuss the convergence of the sequence $x_n = \frac{1}{n+1} + \frac{1}{n+2} + \hdots + \frac{1}{2n}$.
	\begin{itemize}
		% problem 3i
		\item[] {\large 3i)}
		Show that $\{x_n\}$ is bounded. \\
		Clearly we can see that
		\begin{flalign}
			0 \leq \sum_{k=1}^n &\frac{1}{n+k} \leq \sum_{k=1}^n \frac{1}{n} \nonumber \\
			0 \leq &x_n \leq \frac{n}{n} \nonumber \\
			0 \leq &x_n \leq 1 \nonumber
		\end{flalign}
		Thus $x_n$ has an upper and lower bound.
		\newpage

		% problem 3ii
		\item[] {\large 3ii)}
		Show that $\{x_n\}$ is monotonic increasing. \\
		Let us attempt to show that $x_{n+1} - x_n \geq 0$.
		\begin{flalign}
			\frac{1}{2(n+1)(2n+1)} &\geq 0\;\;\; clearly\; true\; \forall n > 0 \nonumber \\
			\frac{1}{2n+2} + \frac{1}{2n+1} - \frac{1}{n+1} &\geq 0 \nonumber \\
			\frac{1}{n+n+2} + \frac{1}{n+n+1} - \frac{1}{n+1} + \sum_{k=2}^{n} \frac{1}{n+k} - \sum_{k=2}^{n} \frac{1}{n+k} &\geq 0 \nonumber \\
			\sum_{k=2}^{n+2} \frac{1}{n+k} - \sum_{k=1}^{n} \frac{1}{n+k} &\geq 0 \nonumber \\
			\sum_{k=1}^{n+1} \frac{1}{n+1+k} - \sum_{k=1}^n \frac{1}{n+k} &\geq 0\;\;\; change\; index \nonumber \\
			x_{n+1} - x_n \geq 0 \nonumber
		\end{flalign}
		$\forall n>0,\; x_{n+1} - x_n \geq 0$ thus $x_n$ is monotonic increasing.

		% problem 3iii
		\item[] {\large 3iii)}
		Find the limit of $\{x_n\}$ by comparing it to an integral. \\
		Note: $\int_a^b f(x)\,dx = \lim_{n \rightarrow \infty} \sum_{k=1}^n f(x_k^*) \Delta x$. \\
		Let's attempt to find $\lim_{n \rightarrow \infty} x_n$.
		\begin{flalign}
			\lim_{n \rightarrow \infty} x_n &= \lim_{n \rightarrow \infty} \sum_{k=1}^n \frac{1}{n+k} \nonumber \\
			&= \lim_{n \rightarrow \infty} \sum_{k=1}^n \frac{1}{1+\frac{n}{k}} \cdot \frac{1}{n} \nonumber
		\end{flalign}
		The limit now is summing the function $\frac{1}{1+x}$, with input $\frac{k}{n}$ where $k$
		is incrementing, $n$ is the amount of partitions in the interval, and
		$\frac{k}{n}$ is $x_k^*$, the upper value of each partitioned interval. This implies that
		the interval goes from 0 to 1. Therefore
		\begin{flalign}
			\lim_{n \rightarrow \infty} \sum_{k=1}^n \frac{1}{1+\frac{n}{k}} \cdot \frac{1}{n} &= \int_0^1 \frac{1}{1+x} \,dx \nonumber \\
			&= \left [ ln(1+x) \right ]_0^1 \nonumber \\
			&= ln(2) - ln(1) \nonumber \\
			\lim_{n \rightarrow \infty} x_n = ln(2) \nonumber
		\end{flalign}
	\end{itemize}
	\newpage

	% problem 4
	\item[] {\large 4)}
	Show that $e \notin \mathbb{Q}$. \\
	Let's assume that $e \in \mathbb{Q}$. This implies that $e=\frac{p}{q};\; p,q \in \mathbb{Z}$.
	Let us consider $(q!)e=p(q-1)! \in \mathbb{Z}$. We know that
	$e=\lim_{n \rightarrow \infty} \sum_{k=0}^n \frac{1}{k!}$. Thus
	$(q!)e=(q!)\lim_{n \rightarrow \infty} \sum_{k=0}^n \frac{1}{k!}$. If $e$ is rational then it
	implies that the result of the following limit must belong to the integers.
	\begin{flalign}
		(q!)e&=(q!)\lim_{n \rightarrow \infty} \sum_{k=0}^n \frac{1}{k!} \in \mathbb{Z} \nonumber \\
		&= \lim_{n \rightarrow \infty} \sum_{k=0}^n \frac{q!}{k!} \nonumber \\
		&= \lim_{n \rightarrow \infty} \left ( \frac{q!}{0!} + \frac{q!}{1!} + \hdots + \frac{q!}{n!} \right ) \nonumber
	\end{flalign}
	For the following limit, if its result must belong to the integers, it implies that
	$\forall n\geq 0,\; n! | q!$. A contradiction. Let $n=q+1$, $(q+1)! \cancel{|} q!$.
	Therefore $e \notin \mathbb{Q}$. $\blacksquare$

\end{itemize}

\end{document}