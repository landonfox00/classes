\documentclass[ 12pt ]{article}

\usepackage{amsmath}
\usepackage{amssymb}
\usepackage{cancel}
\usepackage{tikz}
\usepackage{enumerate}
\usepackage[margin=0.5in]{geometry}
\footskip = -0.75in

\begin{document}

% title page
\title{Homework 4}
\author{Landon Fox}
\date{February 24, 2020}

\begin{flushleft}
Landon Fox \\
MATH 295 \\
Section 1001 \\
February 24, 2020
\end{flushleft}
\begin{center}
Homework 4 \large
\end{center}

\begin{itemize}
	% problem 1
	\item[] {1) \large}
	Prove that there is no positive integer solutions that satisfy
	\begin{flalign}
		x^2 - y^2 = 10 \nonumber
	\end{flalign}
	\begin{flalign}
		assume\;\;\; \exists(x_0, y_0) \epsilon\, \mathbb{Z}^+, (x_0^2 - y_0^2 &= 10) \nonumber \\
		x_0^2 - y_0^2 &= 10 \nonumber \\
		(x_0 - y_0)(x_0 + y_0) &= 10 \nonumber
	\end{flalign}
	The factored terms must both be integers based on the closure of addition and subtraction among integers. Also both must be either positive or negative do to their positive product.
	Since our assumed solution has both terms positive, their addition must also be positive, resulting in the left term in the product also being positive.
	\begin{flalign}
		(x_0 - y_0)(x_0 + y_0) = 10 &\rightarrow ((x_0 - y_0) > 0\, \wedge (x_0 + y_0) > 0)\, \vee ((x_0 - y_0) < 0\, \wedge (x_0 + y_0) < 0) \nonumber \\
		x_0, y_0 \epsilon\, \mathbb{Z}^+ &\rightarrow (x_0 + y_0) \epsilon\, \mathbb{Z}^+ \nonumber \\
		(x_0 + y_0) \epsilon\, \mathbb{Z}^+ &\rightarrow (x_0 - y_0) \epsilon\, \mathbb{Z}^+ \nonumber
	\end{flalign}
	\begin{flalign}
		let\;\;\; x_0 + y_0 &= c_1\, \epsilon\, \mathbb{Z}^+ \nonumber \\
		x_0 - y_0 &= c_2\, \epsilon\, \mathbb{Z}^+ \nonumber \\
		such\; that\;\;\; c_1 c_2 &= 10 \nonumber \\
		(c_1, c_2)\, \epsilon\,& \{ (1, 10), (10, 1), (2, 5), (5, 2) \} \nonumber
	\end{flalign}
	\begin{flalign}
		&x_0 + y_0 = c_1 \nonumber \\
		&x_0 - y_0 = c_2 \nonumber \\
		&\noindent\rule{2cm}{0.4pt} \nonumber \\
		&2x_0 = c_1 + c_2 \nonumber \\
		&x_0 = \frac{1}{2}(c_1 + c_2)\, \epsilon\, \mathbb{Z}^+ \nonumber
	\end{flalign}
	No values of $c_1$ and $c_2$ satisfy the given expression of $x_0$ such that it is a positive integer. Therefore contradicting the initial assumption.
	\begin{flalign}
		\therefore \lnot \exists(x_0, y_0) \epsilon\, \mathbb{Z}^+, (x_0^2 - y_0^2 &= 10)\; \square \nonumber
	\end{flalign}
	\newpage

	% problem 2
	\item[] {2) \large}
	Prove that there is no rational number that satisfies
	\begin{flalign}
		x^3 + x + 1 = 0 \nonumber
	\end{flalign}
	\begin{flalign}
		assume\;\;\; \exists (p, q)\, \epsilon\, \mathbb{Z}, (q \neq 0, gcd(p, q) = 1, \frac{p}{q}^3 + \frac{p}{q} + 1 &= 0) \nonumber
	\end{flalign}
	\begin{flalign}
		\frac{p}{q}^3 + \frac{p}{q} + 1 &= 0 \nonumber \\
		\frac{p}{q}(\frac{p^2}{q^2} + 1) &= -1 \nonumber \\
		\frac{p}{q}(\frac{p^2}{q^2} + 1)\, &\epsilon\, \mathbb{Z} \nonumber \\
		\frac{p}{q} \frac{p^2+q^2}{q^2}\, &\epsilon\, \mathbb{Z} \nonumber
	\end{flalign}
	\begin{flalign}
		gcd(p, q) = 1 &\rightarrow q \cancel{|} p \nonumber \\
		q \cancel{|} p \wedge p \cdot \frac{p^2+q^2}{q^3}\, \epsilon\, \mathbb{Z} &\rightarrow q^3 | (p^2 + q^2) \nonumber
	\end{flalign}
	\begin{flalign}
		let\;\;\; p^2 + q^2 &= cq^3\, \epsilon\, \mathbb{Z} \nonumber \\
		p^2 &= q^2(cq - 1)\, \epsilon\, \mathbb{Z} \nonumber \\
		\frac{p}{q}^2 &= (cq - 1)\, \epsilon\, \mathbb{Z} \nonumber
	\end{flalign}
	\begin{flalign}
		\frac{p}{q}\, \epsilon\, \mathbb{Z} &\rightarrow q | p \nonumber \\
		q | p &\rightarrow gcd(p, q) \neq 1 \nonumber
	\end{flalign}
	\begin{flalign}
		\therefore \lnot \exists (p, q)\, \epsilon\, \mathbb{Z}, (q \neq 0, gcd(p, q) = 1, \frac{p}{q}^3 + \frac{p}{q} + 1 &= 0)\; \square \nonumber
	\end{flalign}
	\newpage

	% problem 3
	\item[] {3) \large}
	Show that if $n$ is odd, then
	\begin{flalign}
		n^2 \equiv 1 (mod\, 8) \nonumber
	\end{flalign}
	\begin{flalign}
		let\;\;\; n = 2k+1,\; k\, \epsilon\, \mathbb{Z} \nonumber \\
		(2k+1)^2 &\equiv 1 (mod\, 8) \nonumber \\
		4k(k + 1) + 1 &\equiv 1 (mod\, 8) \nonumber
	\end{flalign}
	Case 1: $k$ is even
	\begin{flalign}
		let\;\;\; k = 2m,\; m\, \epsilon\, \mathbb{Z} \nonumber \\
		4(2m)(2m + 1) + 1 &\equiv 1 (mod\, 8) \nonumber \\
		8(2m^2 + 1) + 1 &\equiv 1 (mod\, 8) \nonumber
	\end{flalign}
	The above statement is a factor of $8$ with a remainder of $1$, therefore holds true. \\
	Case 2: $k$ is odd
	\begin{flalign}
		let\;\;\; k = 2m + 1,\; m\, \epsilon\, \mathbb{Z} \nonumber \\
		4(2m + 1)(2m + 2) + 1 &\equiv 1 (mod\, 8) \nonumber \\
		8(2m + 1)(m + 1) + 1 &\equiv 1 (mod\, 8) \nonumber
	\end{flalign}
	Again, the above statement also holds true since it is a factor of $8$ with a remainder $1$.
	\begin{flalign}
		\therefore n^2 \equiv 1 (mod\, 8)\; \square \nonumber \\
		\nonumber \\
		\nonumber
	\end{flalign}

	% problem 4
	\item[] {4) \large}
	Show that both $a$ and $b$ cannot both be odd for the following expression
	\begin{flalign}
		\forall a, b\, \epsilon\, \mathbb{Z}, ( 4 | (a^2 + b^2) &\rightarrow even(a) \vee even(b) ) \nonumber \\
		4 | (a^2 + b^2) &\rightarrow even(a) \vee even(b) \nonumber \\
		\lnot (even(a) \vee even(b)) &\rightarrow \lnot (4 | (a^2 + b^2)) \nonumber \\
		odd(a) \wedge odd(b) &\rightarrow 4 \cancel{|} (a^2 + b^2) \nonumber
	\end{flalign}
	\begin{flalign}
		let\;\;\; a &= 2m + 1,\; m\, \epsilon\, \mathbb{Z} \nonumber \\
		b &= 2n + 1,\; n\, \epsilon\, \mathbb{Z} \nonumber \\
		4 &\cancel{|} (2m + 1)^2 + (2n + 1)^2 \nonumber \\
		4 &\cancel{|} 4m^2 + 4m + 1 + 4n^2 + 4n + 1 \nonumber \\
		4 &\cancel{|} 4(m^2 + m + n^2 + n) + 2 \nonumber \\
		4 &\cancel{|} 4(m^2 + m + n^2 + n) \vee 4 \cancel{|} 2 \nonumber \\
		4 &\cancel{|} 2 \nonumber
	\end{flalign}
	\begin{flalign}
		\therefore \forall a, b\, \epsilon\, \mathbb{Z}, ( 4 | (a^2 + b^2) &\rightarrow even(a) \vee even(b) )\; \square \nonumber
	\end{flalign}
	\newpage

	% problem 5
	\item[] {5) \large}
	Show that
	\begin{flalign}
		\forall n\, \epsilon\, \mathbb{N}, (n \geq 2,\, composite(n!+2) \wedge composite(n!+3) \wedge \hdots \wedge composite(n! + n)) \nonumber
	\end{flalign}
	\begin{flalign}
		let\;\;\; 2 &\leq k \leq n,\; k\, \epsilon\, \mathbb{N} \nonumber \\
		n! &= n(n-1)(n-2) \hdots (2)(1) \nonumber \\
		&= n(n-1) \hdots (n-(n-k-1))(n-(n-k)) \hdots (2)(1) \nonumber \\
		&= n(n-1) \hdots (k-1)(k) \hdots (2)(1) \nonumber \\
		k &| n! \nonumber \\
		k &| n! \wedge k | k \rightarrow k | (n! - k) \nonumber
	\end{flalign}
	\begin{flalign}
		\therefore \forall n\, \epsilon\, \mathbb{N}, \forall 2 \leq k \leq n, (n \geq 2,\, composite(n!+k)) \nonumber
	\end{flalign}

	% problem 6
	\item[] {6) \large}
	Prove that
	\begin{flalign}
		\forall a, b\, \epsilon\, \mathbb{N}, (a \geq b,\, gcd(a, b) = gcd(a-b, b)) \nonumber
	\end{flalign}
	\begin{flalign}
		assume\;\;\; gcd(a, b) &\neq gcd(a-b, b) \nonumber \\
		let\;\;\; gcd(a, b) = c,&\; gcd(a-b, b) \neq c \nonumber \\
		gcd(a, b) = c &\rightarrow c | a \wedge c | b \nonumber \\
		gcd(a-b, b) \neq c &\rightarrow c \cancel{|} (a-b) \vee c \cancel{|} b \nonumber \\
		(c | a \wedge c | b) \wedge (c \cancel{|} (a-b) \vee c \cancel{|} b) &\rightarrow c \cancel{|} (a-b) \nonumber \\ 
		c | a \wedge c | b &\rightarrow a = cx \wedge b = cy \nonumber \\
		(a = cx \wedge b = cy) \wedge c \cancel{|} (a-b) &\rightarrow c \cancel{|} (cx-cy) \nonumber \\
		c \cancel{|}& c(x-y) \nonumber
	\end{flalign}
	The above statement is a falacy because any term can divide another if it has the term as a factor.
	\begin{flalign}
		\therefore \forall a, b\, \epsilon\, \mathbb{N}, (a \geq b,\, gcd(a, b) = gcd(a-b, b))\; \square \nonumber
	\end{flalign}
	\newpage

	% problem 7
	\item[] {7) \large}
	\begin{itemize}
		% problem 7i
		\item[] 7i)
		Lemma 1:
		\begin{flalign}
			\forall a\, \epsilon\, \mathbb{Z}, (even(a^2) \rightarrow even(a)) \nonumber \\
			let\;\;\; a^2 = 2k,\; k\, \epsilon\, \mathbb{Z} \nonumber
		\end{flalign}
		Case 1: $a$ is even
		\begin{flalign}
			let\;\;\; a = 2m,\; m\, \epsilon\, \mathbb{Z} \nonumber \\
			(2m)(2m) = 2k \nonumber \\
			2(2m^2) = 2k \nonumber \\
			k = 2m^2\, \epsilon\, \mathbb{Z} \nonumber
		\end{flalign}
		The above statement is a tautology if $a$ is even. \\
		Case 2: $a$ is odd
		\begin{flalign}
			let\;\;\; a = 2m + 1,\; m\, \epsilon\, \mathbb{Z} \nonumber \\
			(2m+1)(2m+1) = 2k \nonumber \\
			4m^2 + 4m + 1 = 2k \nonumber \\
			2(2m^2 + 2m) + 1 = 2k \nonumber
		\end{flalign}
		The above statement is a falacy, no integer $k$ will statisfy the equation.
		\begin{flalign}
			\therefore \forall a\, \epsilon\, \mathbb{Z}, (even(a^2) \rightarrow even(a)) \nonumber
		\end{flalign}
		Lemma 2:
		\begin{flalign}
			\forall a\, \epsilon\, \mathbb{Z}, (even(3a) \rightarrow even(a)) \nonumber \\
			let\;\;\; 3a = 2k,\; k\, \epsilon\, \mathbb{Z} \nonumber
		\end{flalign}
		Case 1: $a$ is even
		\begin{flalign}
			let\;\;\; a = 2m,\; m\, \epsilon\, \mathbb{Z} \nonumber \\
			3(2m) = 2k \nonumber \\
			2(3m) = 2k \nonumber \\
			k = 3m\, \epsilon\, \mathbb{Z} \nonumber
		\end{flalign}
		The above statement is a tautology if $a$ is even. \\
		Case 2: $a$ is odd
		\begin{flalign}
			let\;\;\; a = 2m + 1,\; m\, \epsilon\, \mathbb{Z} \nonumber \\
			3(2m+1) = 2k \nonumber \\
			2(3m + 1) + 1 = 2k \nonumber
		\end{flalign}
		The above statement is a falacy, no intger $k$ will statisfy the equation.
		\begin{flalign}
			\therefore \forall a\, \epsilon\, \mathbb{Z}, (even(3a) \rightarrow even(a)) \nonumber
		\end{flalign}
		\newpage
		Show that
		\begin{flalign}
			\sqrt6\, \cancel{\epsilon}\, \mathbb{Q} \nonumber
		\end{flalign}
		\begin{flalign}
			assume\;\;\; \sqrt6\, &\epsilon\, \mathbb{Q} \nonumber \\
			let\;\;\; \sqrt6 = \frac{p}{q}\; p,q\, &\epsilon\, \mathbb{Z}\; q \neq 0\; gcd(p, q) = 1 \nonumber
		\end{flalign}
		\begin{flalign}
			6 = \frac{p^2}{q^2}& \nonumber \\
			p^2 = 6q^2& \nonumber \\
			p^2 = 2(3q^2) &\rightarrow even(p^2) \nonumber \\
			even(p^2) &\rightarrow even(p)\; via\, lemma\, 1 \nonumber \\
			even(p) &\rightarrow p=2k,\; k\, \epsilon\, \mathbb{Z} \nonumber \\
			(2k)^2 = 6q^2& \nonumber \\
			2k^2 = 3q^2 &\rightarrow even(3q^2) \nonumber \\
			even(3q^2) &\rightarrow even(q^2)\; via\, lemma\, 2 \nonumber \\
			even(q^2) &\rightarrow even(q)\; via\, lemma\, 1 \nonumber \\
			even(p) \wedge even(q) &\rightarrow gcd(p, q) \neq 1 \nonumber
		\end{flalign}
		\begin{flalign}
			\therefore \sqrt6\, \cancel{\epsilon}\, \mathbb{Q}\; \square \nonumber
		\end{flalign}

		% problem 7ii
		\item[] 7ii)
		Show that
		\begin{flalign}
			\sqrt3 + \sqrt2\, \cancel{\epsilon}\, \mathbb{Q} \nonumber
		\end{flalign}
		\begin{flalign}
			assume\;\;\; \sqrt3 + \sqrt2\, &\epsilon\, \mathbb{Q}, \nonumber \\
			let\;\;\; \sqrt3 + \sqrt2 = \frac{p}{q}\; p,q\, &\epsilon\, \mathbb{Z}\; q \neq 0\; gcd(p, q) = 1 \nonumber \\
			(\sqrt3 + \sqrt2)^2 &= \frac{p^2}{q^2} \nonumber \\
			3 + 2\sqrt3\sqrt2 + 2 &= \frac{p^2}{q^2} \nonumber \\
			5 + 2\sqrt6 &= \frac{p^2}{q^2} \nonumber \\
			2\sqrt6 &= \frac{p^2-5q^2}{q^2} \nonumber \\
			\sqrt6 &= \frac{p^2-5q^2}{2q^2} \nonumber
		\end{flalign}
		\newpage
		\begin{flalign}
			let\;\;\; m &= p^2-5q^2\, \epsilon\, \mathbb{Z} \nonumber \\
			n &= 2q^2\, \epsilon\, \mathbb{Z} \nonumber \\
			\sqrt6 &= \frac{m}{n}\, \epsilon\, \mathbb{Q} \nonumber
		\end{flalign}
		The above statement is a falacy as shown by (7i).
		\begin{flalign}
			\therefore \sqrt3 + \sqrt2\, \cancel{\epsilon}\, \mathbb{Q}\; \square \nonumber
		\end{flalign}

		% problem 7iii
		\item[] 7iii)
		Show that
		\begin{flalign}
			log_23\, \cancel{\epsilon}\, \mathbb{Q} \nonumber
		\end{flalign}
		\begin{flalign}
			assume\;\;\; log_23\, &\epsilon\, \mathbb{Q}, \nonumber \\
			let\;\;\; log_23 = \frac{p}{q}\; p,q\, &\epsilon\, \mathbb{Z}\; q \neq 0\; gcd(p, q) = 1 \nonumber \\
			2^{log_23} &= 2^{\frac{p}{q}} \nonumber \\
			3 &= 2^{\frac{p}{q}} \nonumber \\
			\frac{3^q}{2^p} &= 1 \nonumber
		\end{flalign}
		Case 1: $p=q=0$, 
			Satisfies the equation. \\
		Case 2: $p>0,\, q>0$, 
			An arbitrary amount of $2$'s will never divide an arbitrary amount of $3$'s. \\
		Case 3: $p>0,\, q<0$, 
			An arbitrary amount of $2$'s and $3$'s will never divide $1$. \\
		Case 4: $p<0,\, q>0$, 
			Similiar to case 3, an arbitrary amount of $2$'s and $3$'s will never equal $1$. \\
		Case 5: $p<0,\, q<0$, 
			Similiar to case 2, an arbitrary amount of $3$'s will never divide an arbitrary amount of $2$'s. \\
		$p=q=0$ is the only solution, violating $q \neq 0$.
		\begin{flalign}
			\therefore log_23\, \cancel{\epsilon}\, \mathbb{Q} \nonumber
		\end{flalign}


	\end{itemize}
\end{itemize}

\end{document}