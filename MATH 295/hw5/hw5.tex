\documentclass[ 12pt ]{article}

\usepackage{amsmath}
\usepackage{amssymb}
\usepackage{cancel}
\usepackage{tikz}
\usepackage{enumerate}
\usepackage[margin=0.5in]{geometry}
\footskip = -0.75in

\begin{document}

% title page
\title{Homework 5}
\author{Landon Fox}
\date{March 11, 2020}

\begin{flushleft}
Landon Fox \\
MATH 295 \\
Section 1001 \\
March 11, 2020
\end{flushleft}
\begin{center}
Homework 5 \large
\end{center}

\begin{itemize}
	% problem 1
	\item[] {1) \large}
	Case 1: All given values are non-zero
	\begin{center}
		\begin{tabular}{ |c|c|c| }
		 \hline
		 x & y & z \\
		 + & + & + \\
		 + & + & - \\
		 + & - & + \\
		 + & - & - \\
		 - & + & + \\
		 - & + & - \\
		 - & - & + \\
		 - & - & - \\
		 \hline
		\end{tabular}
	\end{center}
	No matter the result of their product, based off the table there always results in two positive or two negative numbers whose product will always be non-negative. \\ \\
	Case 2: There exists one value being zero. The product of any value and zero will always result in a non-negative value therefore the zero value can be multiplied by any other value. \\
	This is a non-constructive proof.

	% problem 2
	\item[] {2) \large}
	Consider the difference between any two consecutive squares
	\begin{flalign}
		&(n+1)^2 - n^2 \nonumber \\
		&n^2 + 2n + 1 + n^2 \nonumber \\
		&2n + 1 \nonumber
	\end{flalign}
	Let this difference be greater than $100$
	\begin{flalign}
		2n + 1 &\geq 100 \nonumber \\
		n &\geq \frac{99}{2} \nonumber
	\end{flalign}
	This implies that the distance between the squares of $50$ and $51$ is greater than $100$ implying that there exists $100$ consecutive positive integers that 
	are not perfect squares. If this were not true, that would imply there is an integer between $50$ and $51$ which is a falacy. \\
	This is a constructive proof.
	\newpage

	% problem 3
	\item[] {3) \large}
	\begin{flalign}
		\exists\, x\, \epsilon\, \mathbb{Q},\, y\, \cancel{\epsilon}\, \mathbb{Q}\; s.t.\; x^y\, \cancel{\epsilon}\, \mathbb{Q} \nonumber \\
		let\;\;\; x = 3,\; y = \frac{1}{2}log_32 \nonumber \\
		x^y = 2^{\frac{1}{2}log_32} \nonumber \\
		= \sqrt{2}\, \cancel{\epsilon}\, \mathbb{Q} \nonumber
	\end{flalign}
	As we have found from lecture and previous homeworks, the following belong holds. \\
	This is a constructive proof.

	% problem 4
	\item[] {4) \large}
	\begin{flalign}
		let\;\;\; S = \mathbb{N} \cup \{0\}& \nonumber \\
		n\, \epsilon\, S \rightarrow \exists\, m\, &\epsilon\, S\; s.t.\; m^2 \leq n < (m+1)^2 \nonumber \\
		m^2 &\leq n < (m+1)^2 \nonumber \\
		m &\leq \sqrt{n} < m + 1 \nonumber
	\end{flalign}
	Existence: \\
	Case 1: $n$ is a square
	\begin{flalign}
		let\;\;\; n=k^2\;\; k\, \epsilon\, S \nonumber \\
		m \leq k < m+1 \wedge k, m\, \epsilon\, S &\rightarrow k = m \nonumber \\
		k = m &\rightarrow m = \sqrt{n} \nonumber
	\end{flalign}
	Case 2: $n$ is not a square
	\begin{flalign}
		let\;\;\; n \neq k^2\;\; k\, \epsilon\, S \nonumber
	\end{flalign}
	Based off the definition of the floor function:
	\begin{flalign}
		\sqrt{n}\, \cancel{\epsilon}\, &S \rightarrow \left \lfloor \sqrt{n} \right \rfloor < \sqrt{n} < \left \lfloor \sqrt{n} \right \rfloor+1 \nonumber \\
		\left \lfloor \sqrt{n} \right \rfloor\, \epsilon\, &S \wedge \left \lfloor \sqrt{n} \right \rfloor < \sqrt{n} < \left \lfloor \sqrt{n} \right \rfloor+1 \rightarrow m = \left \lfloor \sqrt{n} \right \rfloor \nonumber
	\end{flalign}
	Uniqueness: \\
	Assume there is another value $m_2 \neq m_1$
	\begin{flalign}
		let\;\;\; \sqrt{n} = m_1 + k_1 = m_2 + k_2&\;\;\; m_1, m_2\, \epsilon\, S,\; k_1, k_2\, \epsilon\, \mathbb{R},\, 0 \leq k < 1 \nonumber \\
		m_1 + k_1 &= m_2 + k_2 \nonumber \\
		m_1 - m_2 &= k_2 - k_1 \nonumber \\
		m_ 1 - m_2\, \epsilon\, \mathbb{Z} \wedge m_1 - m_2 = k_2 - k_1 &\rightarrow k_2 - k_1\, \epsilon\, \mathbb{Z} \nonumber \\
		0 \leq k_1 < 1 \wedge 0 \leq k_2 < 1 &\rightarrow -1 < k_2 - k_1 < 1 \nonumber \\
		-1 < k_2 - k_1 < 1 \wedge k_2 - k_1\, \epsilon\, \mathbb{Z} &\rightarrow k_2 - k_1 = 0 \nonumber \\
		k_2 - k_1 = 0 &\rightarrow m_1 - m_2 = 0 \nonumber \\
		m_1 - m_2 = 0 &\rightarrow m_1 = m_2 \nonumber
	\end{flalign}
	Condradiction $m_1 = m_2$ \\
	This is a non-constructive proof.

	% problem 5
	\item[] {5) \large}
	Contradiction: lets assume the sum of every $3$ consecutive values are less than $17$
	\begin{flalign}
		a_1 + a_2 + a_3 &\leq 16 \nonumber \\
		a_2 + a_3 + a_4 &\leq 16 \nonumber \\
		a_3 + a_4 + a_5 &\leq 16 \nonumber \\
		&\ddots \nonumber \\
		a_9 + a_{10} + a_1 &\leq 16 \nonumber \\
		a_{10} + a_1 + a_2 &\leq 16 \nonumber
	\end{flalign}
	\begin{flalign}
		\noindent\rule{4cm}{0.4pt} \nonumber \\
	\end{flalign}
	\begin{flalign}
		3(a_1 + a_2 + a_3 + \hdots + a_{10}) &\leq 160 \nonumber \\
		3 \cdot \frac{10(10+1)}{2} &\leq 160 \nonumber \\
		165 &\leq 160 \nonumber
	\end{flalign}
	A contradiction, therefore there must always exist a consecutive $3$ values whose sum is greater than or equal to $17$. \\
	This is a non-constructive proof.

	% problem 6
	\item[] {6 \large}
	
\end{itemize}

\end{document}
