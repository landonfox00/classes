\documentclass[ 12pt ]{article}

\usepackage{amsmath}
\usepackage{amssymb}
\usepackage{cancel}
\usepackage{tikz}
\usepackage{listings}
\usepackage{enumerate}
\usepackage[margin=0.5in]{geometry}
\usepackage{mathtools}

\newcommand\Myperm[2][^n]{\prescript{#1\mkern-2.5mu}{}P_{#2}}
\newcommand\Mycomb[2][^n]{\prescript{#1\mkern-0.5mu}{}C_{#2}}
\footskip = -0.75in

\begin{document}

% title page
\title{Homework 9}
\author{Landon Fox}
\date{April 29, 2020}

\begin{flushleft}
Landon Fox \\
Math 295 \\
Section 1001 \\
April 29, 2020
\end{flushleft}
\begin{center}
\Large Homework 9
\end{center}

\begin{itemize}
	% problem 1
	\item[] {\large 1)}
	Use the Pigeonhole principle to prove that in any group of $n$ people there are
	at least two people who have the same number of friends in the group. \\
	Let us attempt to partition the group into classes based on the number
	of friends they have in the group. Clearly we can say that any individual
	can have anywhere from 0 to $n-1$ friends, providing $n$ classes. \\
	If there is an individual with exactly $n-1$ friends, this would
	imply that there does not exist an individual
	with 0 friends in the group. Furthermore if we have only $n-1$ classes
	and $n$ individuals, via the Pigeonhole principle there would exist at least
	two individuals with the same number of friends in the group. \\
	Otherwise if there does not exist an individual with $n-1$ friends, there would
	not exist anyone in that class. By the same argument, there exists
	$n-1$ classes and $n$ individuals, there must be at least two individuals with
	the same number of friends. $\blacksquare$

	% problem 2
	\item[] {\large 2)}
	Use the Pigeonhole principle to prove that any set of 5 integer lattice points
	in $\mathbb{R}^2$ contains a pair of points whose centroid is
	also an integer lattice point. \\
	Let us attempt to partition the group into classes based on the parity
	of the coordinates. For the centroid of any two points to also be an
	integer lattice point, the $x$ value in the first coordinate can share the
	same parity as the second along the $y$ values as well ($x_1$, $x_2$ have the
	same parity and so does $y_1$ and $y_2$). This provides 4 classes, therefore
	if we are given 5 points via the Pigeonhole principle there must exist two within
	the same class thus they will result in a centroid that is also an interger lattice
	point. $\blacksquare$

	% problem 3
	\item[] {\large 3)}
	\begin{itemize}
		% problem 3i
		\item[] {\large 3i)}
		Show by combinatorial counting that
		$k^2=2\binom{k}{2}+\binom{k}{1}$ for any positive integer $k$. \\
		Let's assume we have a string of two characters with an alphabet of $k$
		characters, how many strings can we create? \\
		By permutations with repetition there is $k^2$ ways. A different
		approach, for both characters we have $\Myperm[k]{2}=2\binom{k}{2}$
		for unique characters and $\Mycomb[k]{1}=\binom{k}{1}$ if they share
		the same character. Thus by the principle of counting in two ways
		$k^2=2\binom{k}{2}+\binom{k}{1}$. $\blacksquare$

		% problem 3ii
		\item[] {\large 3ii)}
		Show that $\sum_{k=1}^n k^2 = \frac{1}{6}n(n+1)(2n+1)$.
		\begin{flalign}
			\sum_{k=1}^n k^2 &= \frac{1}{6}n(n+1)(2n+1) \nonumber \\
			\sum_{k=1}^n \left ( 2\binom{k}{2}+\binom{k}{1} \right ) &=\;\;\; via\; 3i \nonumber \\
			2\sum_{k=2}^n \binom{k}{2} + \sum_{k=1}^n \binom{k}{1} &= \nonumber \\
			2\binom{n+1}{3} + \binom{n+1}{2} &=\;\;\; via\; "Hockey-Stick"\; Identity \nonumber \\
			\frac{2}{3}(n+1)\binom{n}{2} + \frac{1}{2}(n+1)\binom{n}{1} &=\;\;\; via\; "Committee-Chair"\; Identity \nonumber \\
			\frac{1}{6}(n+1)\left ( 4\binom{n}{2} + 3\binom{n}{1} \right ) &= \nonumber \\
			\frac{1}{6}(n+1)( 2n^2 + n ) &=\;\;\; via\; 3i \nonumber \\
			\frac{1}{6}n(n+1)(2n+1) &= \frac{1}{6}n(n+1)(2n+1)\;\; \blacksquare \nonumber
		\end{flalign}
	\end{itemize}

	% problem 4
	\item[] {\large 4)}
	\begin{itemize}
		% problem 4i
		\item[] {\large 4i)}
		Let $p$ be a prime number and $0< k < p$. Show that $p|\binom{p}{k}$. \\
		Observe that $\binom{p}{k}=\frac{p}{p-k}\binom{p-1}{k}$. Also we know that
		$\binom{p}{k} \in \mathbb{Z}$ and that $\binom{p-1}{k} \in \mathbb{Z}$. \\
		Let us consider the case where $p-k=1$, this provides $p|p\binom{p-1}{k}$,
		which is clearly true. \\
		If $p-k > 1$, then $p-k \cancel{|} p$ since it is prime. That would imply that
		$p-k | \binom{p-1}{k}$ since $\binom{p}{k} \in \mathbb{Z}$. Futhermore
		$\frac{1}{p-k}\binom{p-1}{k} \in \mathbb{Z}$ thus $p|\frac{p}{p-k}\binom{p-1}{k}$.
		$\blacksquare$

		% problem 4ii
		\item[] {\large 4ii)}
		Show that $a^p \equiv a\; (mod\; p)$ given $a \in \mathbb{N}$ and $p$ is prime. \\
		Base case $a=1$, $1^p \equiv 1\; (mod\; p)$ clearly holds. \\
		Inductive step, assume $a^p \equiv a\; (mod\; p)$, show that $(a+1)^p \equiv a+1\; (mod\; p)$.
		\begin{flalign}
			(a+1)^p &\equiv a+1\; (mod\; p) \nonumber \\
			\sum_{k=0}^p \binom{p}{k} a^k 1^{p-k} &\equiv\;\;\; via\; Binomial\; Thm \nonumber \\
			\binom{p}{p}a^p + \binom{p}{0}a^0 + \sum_{k=1}^{p-1} \binom{p}{k} a^k &\equiv \nonumber \\
			a^p + 1 + \sum_{k=1}^{p-1} \binom{p}{k} a^k &\equiv \nonumber
		\end{flalign}
		From $4i$ we know that $p|\binom{p}{k}$ if $0< k < p$ and $p$ is prime. Thus every term
		in $\sum_{k=1}^{p-1} \binom{p}{k} a^k$ is divisible by $p$. Therefore
		\begin{flalign}
			a^p + 1 &\equiv a + 1\; (mod\; p) \nonumber \\
			a^p &\equiv a\; (mod\; p) \nonumber
		\end{flalign}
		Based on the assumption this holds, therefore
		$\forall a \in \mathbb{N},\; prime\; p,\;\; a^p \equiv a\; (mod\; p).\;\;\; \blacksquare$
	\end{itemize}

	% problem 5
	\item[] {\large 5)}
	A six-sided die is rolled repeatedly until all the numbers one through six have appeared
	at least once each. What is the probability that this happens in the first $n$ rolls?
	Use a calculator to find the first $n$ where the probability exceeds 0.5. \\
	By using the princible of inclusion-exclusion, the number of cases where all faces
	occur in the first $n$ rolls should be the cardinality of the entire universe, then
	subtracting all cases where one character does not appear, adding all cases where
	two characters do not appear, and so on until we reach the case where zero characters
	appear. This provides
	\begin{flalign}
		\binom{6}{0}6^n - \binom{6}{1}5^n + \binom{6}{2}4^n &- \binom{6}{3}3^n + \binom{6}{4}2^n - \binom{6}{5}1^n + \binom{6}{0}0^n \nonumber \\
		&\sum_{k=0}^{6} (-1)^n\binom{6}{k}(6-k)^n \nonumber
	\end{flalign}
	The total cardinality of the universe is $6^n$ thus the probability is
	\begin{flalign}
		\frac{1}{6^n} \sum_{k=0}^{6} (-1)^n\binom{6}{k}(6-k)^n \nonumber
	\end{flalign}
	The first value of $n$ that results in a probability greater than 0.5 is 13,
	evaluating to roughly 51.4\%.


\end{itemize}

\end{document}