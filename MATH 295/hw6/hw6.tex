\documentclass[ 12pt ]{article}

\usepackage{amsmath}
\usepackage{amssymb}
\usepackage{cancel}
\usepackage{tikz}
\usepackage{enumerate}
\usepackage[margin=0.5in]{geometry}
\footskip = -0.75in

\begin{document}

% title page
\title{Homework 6}
\author{Landon Fox}
\date{March 25, 2020}

\begin{flushleft}
Landon Fox \\
MATH 295 \\
Section 1001 \\
March 25, 2020
\end{flushleft}
\begin{center}
\large Homework 6
\end{center}

\begin{itemize}
	% problem 1
	\item[] {\large 1)}
	\begin{itemize}
		% problem 1i
		\item[] 1i)
		\begin{flalign}
			A &= \{ x\, \epsilon\, \mathbb{R} : x^2 < x \} \nonumber \\
			B &= (0,1) \nonumber
		\end{flalign}
		Show that $A=B$ \\
		$A \subseteq B$
		\begin{flalign}
			let\;\;\; &x\, \epsilon\, A \nonumber \\
			\rightarrow\; &x^2 < x \wedge x\, \epsilon\, \mathbb{R} \nonumber \\
			\rightarrow\; &x \neq 0 \wedge x^2 \geq 0\; iff\; x\, \epsilon\, \mathbb{R} \nonumber \\
			\rightarrow\; &0 < x^2 < x \nonumber \\
			\rightarrow\; &0 < x < 1 \nonumber \\
			\rightarrow\; &x\, \epsilon\, B \nonumber
		\end{flalign}
		$A \supseteq B$
		\begin{flalign}
			let\;\;\; &x\, \epsilon\, B \nonumber \\
			\rightarrow\; &0 < x < 1 \nonumber \\
			\rightarrow\; &0 < x^2 < x \wedge x^2 \geq 0\, iff\, x\, \epsilon\, \mathbb{R} \nonumber \\
			\rightarrow\; &x^2 < x \nonumber \\
			\rightarrow\; &x\, \epsilon\, A \nonumber
		\end{flalign}
		\begin{flalign}
			\therefore A \subseteq B \wedge A \supseteq B \rightarrow A=B\; \blacksquare \nonumber
		\end{flalign}

		% problem 1ii
		\item[] 1ii)
		\begin{flalign}
			A &= \{ x\, \epsilon\, \mathbb{R} : x(x-1)(x-2)(x-3) < 0 \} \nonumber \\
			B &= (0,1) \nonumber \\
			C &= (2,3) \nonumber
		\end{flalign}
		Show that $A=B \cup C$ \\
		$A \subseteq B \cup C$
		\begin{flalign}
			let\;\;\; &x\, \epsilon\, A \nonumber \\
			\rightarrow\; &x(x-1)(x-2)(x-3) < 0 \nonumber \\
			\rightarrow\; &x < 0,\, x < 1,\, x < 2,\, x < 3 \nonumber
		\end{flalign}
		This implies that the interval with an odd amount of terms must be negative. The intervals defined my the terms are
		$(-\infty,-3), (-3,-2e), (-2, -1), (-1, 0), (0, \infty)$. The intervals have the following amount of overlaps.
		\begin{flalign}
			(-\infty,-3) &: 4 \nonumber \\
			(-3,-2) &: 3 \nonumber \\
			(-2, -1) &: 2 \nonumber \\
			(-1, 0) &: 1 \nonumber \\
			(0, \infty) &: 0 \nonumber
		\end{flalign}
		The only intervals with an odd amount of overlaps are $(-3,-2), (-1, 0)$
		\begin{flalign}
			\therefore x\, \epsilon\, B \cup C \nonumber
		\end{flalign}

		$A \supseteq B \cup C$
		\begin{flalign}
			let\;\;\; &x\, \epsilon\, (m, n);\; m < n;\; m,n\, \epsilon\, \mathbb{Z} \nonumber \\
			\rightarrow\; &m < x \wedge x < n \nonumber \\
			\rightarrow\; &0 < x - m \wedge x - n < 0 \nonumber \\
			\rightarrow\; & (x - m)(x - n) < 0 \nonumber \\
			let\;\;\; &p, q\, \epsilon\, (-\infty, m) \vee p, q\, \epsilon\, (n, \infty) \nonumber \\
			\rightarrow\; &(p < x \wedge q < x) \vee (p > x \wedge q > x) \nonumber \\
			\rightarrow\; &(0 < x - p \wedge 0 < x - q) \vee (0 > x - p \wedge 0 > x - q) \nonumber \\
			\rightarrow\; &0 < (x - p)(x - q) \nonumber \\
			\rightarrow\; &(x - m)(x - n)(x - p)(x - q) < 0 \nonumber
		\end{flalign}
		\begin{flalign}
			let\;\;\; &x\, \epsilon\, B \cup C \nonumber \\
			\rightarrow\; &x\, \epsilon\, (0,1) \cup (2,3) \nonumber \\
			WLOG\; let\;\;\; &m = 0,\, n = 1,\, p = 2,\, q = 3 \nonumber \\
			\rightarrow\; &x(x-1)(x-2)(x-3) < 0 \nonumber \\
			\rightarrow\; &x\, \epsilon\, A \nonumber
		\end{flalign}
		\begin{flalign}
			\therefore A \subseteq B \cup C \wedge A \supseteq B \cup C \rightarrow A = B \cup C \; \blacksquare \nonumber
		\end{flalign}
	\end{itemize}

	% problem 2
	\item[] {\large 2)}
	\begin{itemize}
		% problem 2i
		\item[] 2i)
		\begin{flalign}
			x^2 < y^2 \rightarrow x < y;\; x, y\, \epsilon\, \mathbb{R} \nonumber
		\end{flalign}
		Disproof: counter-example
		\begin{flalign}
			let\;\;\; &x = 0,\, y = -1 \nonumber \\
			\rightarrow\; &0 < 1 \rightarrow 0 < -1\; \blacksquare \nonumber
		\end{flalign}
		A fallacy.

		% problem 2ii
		\item[] 2ii)
		\begin{flalign}
			x^3 < y^3 \rightarrow x < y;\; x, y\, \epsilon\, \mathbb{R} \nonumber
		\end{flalign}
		Proof: contraposition
		\begin{flalign}
			x \geq y \rightarrow x^3 \geq y^3 \nonumber
		\end{flalign}
		\begin{flalign}
			&x \geq y \nonumber \\
			\rightarrow\; &x^3 \geq y^3\; \blacksquare \nonumber
		\end{flalign}
	\end{itemize}

	% problem 3
	\item[] {\large 3)}
	\begin{itemize}
		% problem 3i
		\item[] 3i)
		\begin{flalign}
			\exists\, p, q\;\; s.t.\;\; p - q = 2020;\; p, q\; prime \nonumber
		\end{flalign}
		Proof: constructive
		\begin{flalign}
			let\;\;\; &p = 2039,\, q = 19 \nonumber \\
			\rightarrow\; &2039 - 19 = 2020\; \blacksquare \nonumber
		\end{flalign}

		% problem 3ii
		\item[] 3ii)
		\begin{flalign}
			\exists\, p, q\;\; s.t.\;\; p - q = 295;\; p, q\; prime \nonumber
		\end{flalign}
		Disproof: contradiction
		\begin{flalign}
			&p - q = 295 \nonumber \\
			\rightarrow\; &odd(p - q) \wedge p > q \nonumber \\
			\rightarrow\; &p, q\; different\; parity \wedge p > q \nonumber \\
			\rightarrow\; &q = 2 \nonumber \\
			\rightarrow\; &p - 2 = 295 \nonumber \\
			\rightarrow\; &p = 297 \nonumber \\
			\rightarrow\; &p\; not\; prime\; \blacksquare \nonumber
		\end{flalign}
		A contradiction, p was defined to be prime.
	\end{itemize}
	\newpage

	% problem 4
	\item[] {\large 4)}
	\begin{itemize}
		% problem 4i
		\item[] 4i)
		Proof: constructive \\
		Tile all dominos horizontally such that the top row (row with missing corners) has three dominos and all others have four.

		% problem 4ii
		\item[] 4ii)
		Disproof: contradiction \\
		Let $h$ and $v$ be the total amount of horizontal and vertical dominos.
		\begin{flalign}
			2h + 2v &= 62 \nonumber \\
			h + v &= 31 \nonumber \\
			\rightarrow\; h, v\; &different\; parity \nonumber
		\end{flalign}
		Let $h_i$ be the amount of horizontal dominos in the $i$th row. \\
		and $v_i$ be the amount of vertical dominos in the $i$th row, accounting only when the lowest tile is came upon. \\
		Then we should have the following equations
		\begin{flalign}
			&2h_1 + v_1 = 7 \nonumber \\
			&2h_2 + v_2 + v_1 = 8 \nonumber \\
			&2h_3 + v_3 + v_2 = 8 \nonumber \\
			&\;\;\;\;\;\;\vdots \nonumber \\
			&2h_7 + v_7 + v_6 = 8 \nonumber \\
			&2b_8 + v_7 = 7 \nonumber
		\end{flalign}
		Mod both sides of all the equations by 2.
		\begin{flalign}
			&v_1 \equiv 1 (mod\, 2) \nonumber \\
			&v_2 + v_1 \equiv 0 (mod\, 2) \nonumber \\
			&v_3 + v_2 \equiv 0 (mod\, 2) \nonumber \\
			&\;\;\;\;\;\;\vdots \nonumber \\
			&v_7 + v_6 \equiv 0 (mod\, 2) \nonumber \\
			&v_7 \equiv 1 (mod\, 2) \nonumber \\
			\rightarrow\; &\forall 1\leq i\leq7, odd(v_i) \nonumber \\
			\rightarrow\; &odd(v) \nonumber \\
			\rightarrow\; &even(h) \nonumber
		\end{flalign}
		Let us now repeat the process for columns. Now the role for horizontal and vertical dominos swap. \\
		WLOG, just as we found that the amount of horizontal dominos is even, we will find that it is also odd via the same logic of the column equations. \\
		A fallacy, the amount of horizontal dominos cannot be both even and odd. $\blacksquare$
		\newpage

		% problem 4iii
		\item[] 4iii)
		Disproof: contradiction \\
		Let us create similiar equations as we did in $ii$ for triominos.
		\begin{flalign}
			&3h_1 + v_1 = 7 \nonumber \\
			&3h_2 + v_2 + v_1 = 8 \nonumber \\
			&3h_3 + v_3 + v_2 + v_1 = 8 \nonumber \\
			&\;\;\;\;\;\;\vdots \nonumber \\
			&3h_6 + v_6 + v_5 + v_4 = 8 \nonumber \\
			&3h_7 + v_6 + v_5 = 8 \nonumber \\
			&3b_8 + v_6 = 8 \nonumber
		\end{flalign}
		Mod both sides of all the equations by 3.
		\begin{flalign}
			&v_1 \equiv 1 (mod\, 3) \nonumber \\
			&v_2 + v_1 \equiv 2 (mod\, 3) \nonumber \\
			&v_3 + v_2 + v_1 \equiv 2 (mod\, 3) \nonumber \\
			&\;\;\;\;\;\;\vdots \nonumber \\
			&v_6 + v_5 + v_4 \equiv 2 (mod\, 3) \nonumber \\
			&v_6 + v_5 \equiv 2 (mod\, 3) \nonumber \\
			&v_6 \equiv 2 (mod\, 3) \nonumber \\
			\rightarrow\; &v_2 \equiv 1 (mod\, 3) \nonumber \\
			\rightarrow\; &v_3 \equiv 0 (mod\, 3) \nonumber \\
			\rightarrow\; &v_4 \equiv 1 (mod\, 3) \nonumber \\
			\rightarrow\; &v_5 \equiv 1 (mod\, 3) \wedge v_5 \equiv 2 (mod\, 3) \nonumber \\
			\rightarrow\; &v_6 \equiv 0 (mod\, 3) \wedge v_6 \equiv 2 (mod\, 3) \nonumber
		\end{flalign}
		Both $v_5$ and $v_6$ themselves are of different congruences, a fallacy. $\blacksquare$
	\end{itemize}

	% problem 5
	\item[] {\large 5)}
	\begin{itemize}
		% problem 5i
		\item[] 5i)
		\begin{flalign}
			A_1 &= \{ (x,y)\, \epsilon\, \mathbb{R}^2 : x^2 + y^2 = 1, x \neq 1 \} \nonumber \\
			B_1 &= \{ ( \frac{1-t^2}{1+t^2}, \frac{2t}{1+t^2} ) : t\, \epsilon\, \mathbb{R} \} \nonumber
		\end{flalign}
		\newpage
		Show that $A_1=B_1$ \\
		$A_1 \subseteq B_1$ \\
		Let us consider the line $y = t(x+1)$
		\begin{flalign}
			let\;\;\; S &= \{ (x,y)\, \epsilon\, \mathbb{R}^2 : y = t(x + 1), t\, \epsilon\, \mathbb{R} \} \nonumber \\
			clearly\;\;\; A_1 &\subseteq S \nonumber
		\end{flalign}
		\begin{flalign}
			let\;\;\; &(x,y)\, \epsilon\, A_1 \nonumber \\
			\rightarrow\; &(x,y)\, \epsilon\, S \nonumber \\
			\rightarrow\; &(x,y) = (x, t(x+1));\; x, t\, \epsilon\, \mathbb{R} \nonumber \\
			\rightarrow\; &x^2 + t^2(x+1)^2 = 1 \nonumber \\
			\rightarrow\; &(1+t^2)x^2 + 2t^2x + t^2 - 1 = 0 \nonumber \\
			\rightarrow\; &x = \frac{-2t \pm \sqrt{4t^4 - 4(t^2 + 1)(t^2 - 1)}}{2(t^2+1)} \nonumber \\
			\rightarrow\; &x = \cancel{-1}, \frac{1 - t^2}{1 + t^2} \nonumber \\
			\rightarrow\; &y = \frac{2t}{1 + t^2} \nonumber \\
			\rightarrow\; &(x,y) = (\frac{1 - t^2}{1 + t^2}, \frac{2t}{1 + t^2}) \nonumber \\
			\rightarrow\; &(x,y)\, \epsilon\, B_1 \nonumber
		\end{flalign}

		$A_1 \supseteq B_1$
		\begin{flalign}
			let\;\;\; &(x,y)\, \epsilon\, B_1 \nonumber \\
			\rightarrow\; &(x,y) = (\frac{1 - t^2}{1 + t^2}, \frac{2t}{1 + t^2});\; t\, \epsilon\, \mathbb{R} \nonumber \\
			\rightarrow\; &x = \frac{1 - t^2}{1 + t^2},\, y = \frac{2t}{1 + t^2} \nonumber \\
			\rightarrow\; &t^2x + t^2 = 1 - x \nonumber \\
			\rightarrow\; &t = \pm \sqrt{\frac{1 - x}{1 + x}} \nonumber \\
			\rightarrow\; &y = \pm \frac{2\sqrt{ \frac{1-x}{1+x} }}{1 + \frac{1-x}{1+x}} \nonumber \\
			\rightarrow\; &y = \pm \sqrt{1 - x^2} \nonumber \\
			\rightarrow\; &x^2 + y^2 = 1 \wedge \neg \exists t\, \epsilon\, \mathbb{R}\; s.t.\; \frac{1-t^2}{1+t^2} = -1 \nonumber \\
			\rightarrow\; &(x,y)\, \epsilon\, A_1 \nonumber
		\end{flalign}
		\begin{flalign}
				\therefore A_1 \subseteq B_1 \wedge A_1 \supseteq B_1 \rightarrow A_1=B_1\; \blacksquare \nonumber
		\end{flalign}
		\newpage

		% problem 5ii
		\item[] 5ii)
		\begin{flalign}
			A_2 &= \{ (x,y)\, \epsilon\, \mathbb{Q}^2 : x^2 + y^2 = 1, x \neq 1 \} \nonumber \\
			B_2 &= \{ ( \frac{1-t^2}{1+t^2}, \frac{2t}{1+t^2} ) : t\, \epsilon\, \mathbb{Q} \} \nonumber
		\end{flalign}
		Show that $A_2=B_2$
		\begin{flalign}
			assuming\;\;\; (x, y)\, \epsilon\, A_1 \wedge x, y\, \epsilon\, \mathbb{Q} \rightarrow (x, y)\, \epsilon\, A_2 \nonumber
		\end{flalign}
		$A_2 \subseteq B_2$
		\begin{flalign}
			let\;\;\; &(x, y)\, \epsilon\, A_2 \nonumber \\
			\rightarrow\; &(x, y)\, \epsilon\, A_1 \wedge x, y\, \epsilon\, \mathbb{Q} \nonumber \\
			\rightarrow\; &(x, y)\, \epsilon\, B_1 \wedge x, y\, \epsilon\, \mathbb{Q} \nonumber \\
			\rightarrow\; &(x, y)\, \epsilon\, B_2 \nonumber
		\end{flalign}
		WLOG we can presume via the same logic that $A_2 \supseteq B_2$.
		\begin{flalign}
				\therefore A_2 \subseteq B_2 \wedge A_2 \supseteq B_2 \rightarrow A_2=B_2\; \blacksquare \nonumber
		\end{flalign}

		% problem 5iii
		\item[] 5iii)
		\begin{flalign}
			A_3 &= \{ (x,y,z)\, \epsilon\, \mathbb{Z}^3 : x^2 + y^2 = z^2 \} \nonumber \\
			B_3 &= \{ ( m^2 - n^2, 2mn, m^2 + n^2 ) : m,n\, \epsilon\, \mathbb{Z} \} \nonumber
		\end{flalign}
		Show that $A_3=B_3$ \\
		$A_3 \subseteq B_3$
		\begin{flalign}
			let\;\;\; &(x, y, z)\, \epsilon\, A_3 \nonumber \\
			\rightarrow\; &x^2 + y^2 = z^2 \wedge x, y, z\, \epsilon\, \mathbb{Z} \nonumber \\
			\rightarrow\; &(\frac{x}{z})^2 + (\frac{y}{z})^2 = 1 \nonumber \\
			let\;\;\; &u = \frac{x}{z},\, v = \frac{y}{z};\; u, v\, \epsilon\, \mathbb{Q} \nonumber \\
			\rightarrow\; &u^2 + v^2 = 1 \nonumber \\
			\rightarrow\; &(u, v)\, \epsilon\, A_2 \nonumber \\
			\rightarrow\; &(u, v)\, \epsilon\, B_2\; via\; ii \nonumber \\
			\rightarrow\; &u = \frac{1-\frac{n}{m}^2}{1+\frac{n}{m}^2}, v = \frac{2\frac{n}{m}}{1+\frac{n}{m}^2};\; m, n\, \epsilon\, \mathbb{Z} \nonumber \\
			\rightarrow\; &\frac{x}{z} = \frac{m^2 - n^2}{m^2 + n^2},\, \frac{y}{z} = \frac{2mn}{m^2 + n^2} \nonumber \\
			\rightarrow\; &x = m^2 - n^2,\, y = 2mn,\, z = m^2 + n^2;\; m, n\, \epsilon\, \mathbb{Z} \nonumber \\
			\rightarrow\; &(x, y, z)\, \epsilon\, B_3 \nonumber
		\end{flalign}

		$A_3 \supseteq B_3$
		\begin{flalign}
			let\;\;\; &(x, y, z)\, \epsilon\, B_3 \nonumber \\
			\rightarrow\; &x = m^2 - n^2,\, y = 2mn,\, z = m^2 + n^2;\; m, n\, \epsilon\, \mathbb{Z} \nonumber \\
			\rightarrow\; &x + z = 2m^2,\, m^2 = \frac{y^2}{4n^2} \nonumber \\
			\rightarrow\; &x + z = \frac{y^2}{2n^2} \nonumber \\
			\rightarrow\; &n^2 = \frac{y^2}{2(x+z)} \nonumber \\
			\rightarrow\; &m^2 = \frac{y^2}{2\frac{y^2}{(x+z)}} \nonumber \\
			\rightarrow\; &m^2 = \frac{1}{2}(x+z) \nonumber \\
			\rightarrow\; &x = \frac{1}{2}(x+z) - \frac{y^2}{2(x+z)} \nonumber \\
			\rightarrow\; &2x(x + z) = (x + z)^2 - y^2 \nonumber \\
			\rightarrow\; &x^2 + y^2 = z^2 \nonumber \\
			\rightarrow\; &(x, y, z)\, \epsilon\, A_3 \nonumber
		\end{flalign}
		\begin{flalign}
				\therefore A_3 \subseteq B_3 \wedge A_3 \supseteq B_3 \rightarrow A_3=B_3\; \blacksquare \nonumber
		\end{flalign}
	\end{itemize}

\end{itemize}

\end{document}
