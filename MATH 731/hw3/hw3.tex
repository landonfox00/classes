\documentclass[ 12pt ]{article}
\usepackage{amsmath, amsthm, amssymb, csquotes, enumitem, extpfeil, graphicx, listings, mathrsfs, tikz-cd}
\usepackage[margin=0.5in]{geometry}
\graphicspath{ ./ }

\begin{document}

\noindent Landon Fox \\
\noindent Math 731, Modern Algebra \\
\noindent September 22, 2021

\begin{center}
\Large Homework 3
\end{center}

\begin{enumerate}
	% problem 1
	\item[\textbf{1.}] Let $R$ be a commutative ring and $N \subseteq R$, the collection of all nilpotent elements of $R$.
		\begin{enumerate}
			\item[\textbf{a.}] Show that $N$ is an ideal of $R$.
			\item[\textbf{b.}] Show that the only nilpotent element of $R/N$ is zero.
		\end{enumerate}

		\textbf{Lemma.} Let $a \in R$ be a nilpotent element of a commutative ring $R$. Then its additive inverse $-a \in R$ is also nilpotent.

		\begin{proof}[Lemma Proof.]
			Suppose $R$ is a commutative ring. Let $a, b \in R$ be arbitrary ring elements. Observe that $$0 = a \cdot 0 = a(b + (-b)) = ab + a(-b)$$ and so $a(-b) = -ab$. Furthermore by commutativity, it also holds that $(-b)a = -ba$. Now, let $a \in R$ be nilpotent. By definition, there exists an $n \in \mathbb{N}$ such that $a^n = 0$. We show that for all $m \in \mathbb{N}$, $$(-a)^m = \begin{cases} a^m; & m\; \mathrm{even}, \\ -a^m; & m\; \mathrm{odd}. \end{cases}$$ It is trivial that $m = 1$ satisfies the requirement. Assume $(-a)^m$ satisfies the above criteria for a particular $m \in \mathbb{N}$. If $m$ is even, then $$(-a)^{m+1} = (-a)^m (-a) = a^m (-a) = -a^{m+1}$$ since we know that $a(-b) = -ab$ for all $a, b \in R$. Similarly, if $m$ is odd, then $$(-a)^{m+1} = (-a)^m (-a) = (-a^m) (-a) = -((-a^m)a) = -(-a^{m + 1}) = a^{m+1}$$ as desired. Therefore, if $a^n = 0$, it must follow that $(-a)^n = \pm a^n = 0$ (abusing notation), proving the claim.
		\end{proof}

		\begin{proof}
			Let $R$ denote a commutative ring and $N \subseteq R$, the set of all nilpotent elements in $R$.
			\begin{enumerate}
				\item[\textbf{a.}] Further, let $r \in R$ and $x, y \in N$ be arbitrary ring elements were $x$ and $y$ are nilpotent. By our lemma, we also know that $-y$ is nilpotent. Hence, $x - y \in N$ by our last problem set and so $N$ is an abelian subgroup of $R$. As for multiplicative closure, by definition there exists a natural $n \in \mathbb{N}$ such that $x^n = 0$. Furthermore, $$(rx)^n = r^n x^n = r^n \cdot 0 = 0$$ and $$(xr)^n = x^n r^n = 0 \cdot r^n = 0.$$ Hence, $rx, xr \in N$ are nilpotent and $rN = N = Nr$ for all $r \in R$.

				\item[\textbf{b.}] Suppose $a + N \in R/N$ is nilpotent. Let $n \in \mathbb{N}$ be a natural such that $(a + N)^n = 0 + N = N$, the zero of $R/N$. By definition of the quotient ring, it follows that $$a^n + N = (a + N)^n = 0 + N$$ and so $a^n = a^n - 0 \in N$. Furthermore, $a^n$ is nilpotent in $R$. Then there exists a $m \in \mathbb{N}$ such that $$a^{mn} = (a^n)^m = 0 \in R;$$ consequently, $a$ is nilpotent by definition. Hence, $a - 0 = a \in N$ allowing us to conclude that $$a + N = 0 + N = N.$$
			\end{enumerate}
		\end{proof}


	% problem 2
	\item[\textbf{2.}] Let $f : R \to S$ be a ring homomorphism and let $I \subseteq R$ and $J \subseteq S$ be ideals in $R$ and $S$ respectively.
	\begin{enumerate}
		\item[\textbf{a.}] Show that $f^{-1}(J)$ is an ideal in $R$.
		\item[\textbf{b.}] Show that $f(I)$ may fail to be an ideal in $S$. Show that $f(I)$ is a subring of $S$.
		\item[\textbf{c.}] Show that if $f$ is surjective, then $f(I)$ is an ideal in $S$.
	\end{enumerate}

		\begin{proof}
			Suppose $f : R \to S$ is a ring homomorphism and $I \subseteq R$ and $J \subseteq S$ are ideals in their respective rings.
			\begin{enumerate}
				\item[\textbf{a.}] Provided that $J$ is an ideal, there exists a ring homomorphism $g : S \to T$ for some ring $T$ such that $\mathrm{ker}\, g = J$. Then the image of $f^{-1}(J)$ under the composition $gf$ provides $$gf( f^{-1}(J) ) = g(J) = \{ 0 \in T \}$$ and so $f^{-1}(J) \subseteq \mathrm{ker}\, gf$. Then for any $x \in R$ such that $gf(x) = 0$, it must hold that $f(x) \in \mathrm{ker}\, g = J$. Therefore, $x \in f^{-1}(J)$. Thus, $f^{-1}(J) = \mathrm{ker}\, gf$, demonstrating that $f^{-1}(J)$ is an ideal of $R$.

				\item[\textbf{b.}] Consider the inclusion $\iota : \mathbb{Z} \hookrightarrow \mathbb{Z}[t]$. It is easy to see that $\iota$ is a ring homomorphism; indeed, $$\iota(a + b) = a + b = \iota(a) + \iota(b)\;\;\; \mathrm{and}\;\;\; \iota(ab) = ab = \iota(a) \iota(b)$$ hold for all $a, b \in \mathbb{Z}$. Consider an ideal $(n) \subseteq \mathbb{Z}$ constructed from a natural $n \in \mathbb{N}$. Additionally, take any polynomial $\sum_{i = 0}^k m_i t^i \in \mathbb{Z}[t]$ with at least one $0 < j \leq k \in \mathbb{N}$ such that $m_j \neq 0$. Then it follows that $$\iota(n) \sum_{i = 0}^k m_i t^i = n \sum_{i = 0}^k m_i t^i = \sum_{i = 0}^k n m_i t^i \notin (n)$$ since $nm_j \neq 0$ because $\mathbb{Z}$ has no non-zero zero divisors and so the polynomial above is non-constant (one may also choose a simpler example with $(2) \subseteq \mathbb{Z}$ and $x \in \mathbb{Z}[t]$ and apply the same argument). Hence, $\iota((n))$ is not an ideal. \\

				For the example provided above, $\iota((n))$ is in fact a subring of $\mathbb{Z}[t]$. We will prove this more generally. Suppose that $x, y \in I$. Since $I$ is a subgroup of $R$, we know that $x - y \in I$. Similarly, it holds that $f(I)$ is an abelian group of $S$ as a result of $$f(x) - f(y) = f(x) + f(-y) = f(x - y) \in f(I)$$ where additive inverses are preserved since $f$ is also a group homomorphism. As for multiplication, we can see that $xy \in I$ and so $$f(x) f(y) = f(xy) \in f(I).$$ Therefore, $f(I)$ is a subring of $S$.

				\item[\textbf{c.}] Let $f$ be a surjection. As shown above, it is clear that $f(I) \subseteq S$ is a subring of $S$; all we must demonstrate is multiplicative closure. Let $x \in I$ and $s \in S$. Due to surjectivity, there exists an $r \in R$ such that $f(r) = s$. Additionally, by assumption, $rx, xr \in I$. Furthermore, it follows that $$s \cdot f(x) = f(r) f(x) = f(rx) \in f(I)$$ and $$f(x) \cdot s = f(x) f(r) = f(xr) \in f(I).$$ Thus, $f(I)$ is an ideal of $S$.
			\end{enumerate}
		\end{proof}


\end{enumerate}

\end{document}
