\documentclass[ 12pt ]{article}
\usepackage{amsmath, amsthm, amssymb, csquotes, enumitem, extpfeil, graphicx, listings, mathrsfs, tikz-cd}
\usepackage[margin=0.5in]{geometry}
\graphicspath{ ./ }

\begin{document}

\noindent Landon Fox \\
\noindent Math 731, Modern Algebra \\
\noindent October 18, 2021

\begin{center}
\Large Homework 6
\end{center}

\begin{enumerate}
	% problem 1
	\item[\textbf{1.}] Let $R$ be a subset of $\mathbb{C}$ given by $R = \left\{ a + b \frac{1 + i\sqrt{19}}{2} : a, b \in \mathbb{Z} \right\}$.
	\begin{enumerate}
		\item[\textbf{a.}] Show that $R$ is a subring of $\mathbb{C}$.
		\item[\textbf{b.}] Show that $R$ is not a Euclidean domain.
	\end{enumerate}

		\begin{proof}
			Define a subset $$R = \left \{ a + b \frac{1 + i\sqrt{19}}{2} : a, b \in \mathbb{Z} \right \} \subseteq \mathbb{C}.$$
			\begin{enumerate}
				\item[\textbf{a.}] Let $a, b, c, d \in \mathbb{Z}$ be arbitrary integers. We can see that $$\left(a + b \frac{1 + i\sqrt{19}}{2}\right) - \left(c + d \frac{1 + i\sqrt{19}}{2}\right) = (a - c) + (b - d) \frac{1 + i\sqrt{19}}{2} \in R,$$ illustrating that $R$ is a subgroup of $\mathbb{C}$. As for multiplication, observe that
				\begin{align*}
					\left (a + b \frac{1 + i\sqrt{19}}{2} \right ) \left (c + d \frac{1 + i\sqrt{19}}{2} \right ) &= ac + (ad + bc) \frac{1 + i\sqrt{19}}{2} + bd \left ( \frac{1 + i\sqrt{19}}{2} \right )^2 \\
					&= (ac - 5bd) + (ad + bc + bd) \frac{1 + i\sqrt{19}}{2} \in R
				\end{align*}
				and so $R$ is a subring of $\mathbb{C}$.

				\item[\textbf{b.}] We will begin by showing that $2$ and $3$ are uninvertible and irreducible in $R$. First, notice that the elements with the smallest magnitude, $1$, in $R$ are $-1$ and $1$. Provided our knowledge of the complex numbers, the multiplicative inverses of $2$ and $3$ must have magnitudes $\frac{1}{2}$ and $\frac{1}{3}$, respectively. Therefore, $2$ and $3$ are nonunits in $R$. Further, the only units of $R$ are limited to $-1$ and $1$ via the same argument. Suppose that $2 = rs$ for elements $r, s \in R$. Since $$\left | a + b \frac{1 + i\sqrt{19}}{2} \right | = \left | \left(a + \frac{1}{2} b\right) + \frac{\sqrt{19}}{2} ib \right |^2 = \left(a + \frac{1}{2} b\right)^2 + \frac{19}{4} b^2 = a^2 + ab + 5b^2 \in \mathbb{N},$$ for all $a, b \in \mathbb{Z}$, there are only a few cases where $4 = |2|^2 = |r|^2 |s|^2$: $$|r|^2 = 1,\; |s|^2 = 4,\;\;\; |r|^2 = |s|^2 = 2,\;\;\; \mathrm{or}\;\;\; |r|^2 = 4,\; |s|^2 = 1.$$ We may neglect the cases where $|r|^2 = 1$ and $|s|^2 = 1$ as they are units. Assume $|r|^2 = |s|^2 = 2$. If $r = a + b \frac{1 + i\sqrt{19}}{2}$, then $$a^2 + ab + 5b^2 = |r|^2 = 2,$$ illustrating that $b = 0$; however, $a \notin \mathbb{Z}$ provided that $a^2 = 2$. We may use the same argument to illustrate that $3$ is also irreducible. \\

				Now we show that there exists no $\varphi : R \setminus \{ 0 \} \to \mathbb{N}$ providing a Euclidean ring structure on $R$. Suppose that $R$ is a Euclidean ring with function $\varphi$. Let $m \in R$ be a minimal nonzero nonunit under $\varphi$. Next, we apply the division algorithm on $2$ and $\frac{1 + i\sqrt{19}}{2}$ with $m$. We obtain elements $q_1, q_2, r_1, r_2 \in R$ such that $$2 = mq_1 + r_1\;\;\; \mathrm{and}\;\;\; \frac{1 + i\sqrt{19}}{2} = mq_2 + r_2$$ as well as $\varphi(r_1), \varphi \left ( \frac{1 + i\sqrt{19}}{2} \right ) < \varphi(m)$. Furthermore, due to the minimality of $\varphi(m)$, it holds that $r_1, r_2$ are both either zero or units; that is, $r_1, r_2 \in \{ -1, 0, 1 \}$. We show that all cases lead to a contradiction. Suppose $r_1 = -1$. It follows that $mq_1 = 3$, illustrating that $q_1 = -1$ or $q_1 = 1$ by the irreducibility of $3$. Therefore, $m = -3$ or $m = 3$. In the case that $r_1 = 0$, we have $q_1 = -1$ or $q_1 = 1$ and so $m = -2$ or $m = 2$. Lastly, if $r_1 = 1$, then $mq_1 = 1$, contradicting the nonunitality of $m$ since $R$ is commutative. We have now shown that $m \in \{ -3, -2, 2, 3 \}$. Next, we utilize $r_2$. For all values of $m$ and $r_2$, it is easy to see that $$q_2 = \frac{1}{m} \left ( \frac{1 + i\sqrt{19}}{2} - r_2 \right ) \notin R,$$ a contradiction.
			\end{enumerate}
		\end{proof}


	% problem 2
	\item[\textbf{2.}] Let $R$ be a subring of $\mathbb{C}$ given by $R = \{ a + bi \sqrt{5} : a, b \in \mathbb{Z} \}$. Let $I \subseteq R$ be the ideal generated by the set $\{ 2, 1 + i\sqrt{5} \}$.
	\begin{enumerate}
		\item[\textbf{a.}] Show that $I$ is not a principal ideal.
		\item[\textbf{b.}] Show that the product of non-principal ideals can be principal.
	\end{enumerate}

		\begin{proof}
			Let $R \subseteq \mathbb{C}$ be the subring defined above and define $I = (2, 1 + i\sqrt{5}) \subseteq R$.
			\begin{enumerate}
				\item[\textbf{a.}] By the same argument of \textbf{1b}, we can see that $-1$ and $1$ are the only units of $R$; more specifically, $2$ and $1 + i\sqrt{5}$ are noninvertible. If there exists integers $a, b, c, d \in \mathbb{Z}$ such that $$2 = (a + bi\sqrt{5})(c + di\sqrt{5}) = (ac - 5bd) + (ad + bc)i\sqrt{5},$$ it must hold that $$a^2 + 5b^2 \neq 0,\;\;\; c = \frac{2a}{a^2 + 5b^2},\;\;\; \mathrm{and}\;\;\; d = -\frac{2b}{a^2 + 5b^2}$$ when solving the system. As $a$ and $b$ cannot both be zero, assume $b = 0$. Then $c \in \mathbb{Z}$ precisely when $a \in \{ -2, -1, 1, 2 \}$, all illustrating that a unit is included in our original factorization of $2$. Next, we consider the case $b \neq 0$. Without loss of generality, we may assume $a \geq 0$ and $b > 0$. Furthermore, $$0 \leq (a - b)^2 = a^2 - 2ab + b^2$$ provides $$2a \leq 2ab \leq a^2 + b^2 < a^2 + 5b^2$$ and so $a^2 + 5b^2 \nmid 2a$ provided our knowledge of integer division. Hence, $2$ is irreducible. As before, consider a product in $R$ such that $$1 + i\sqrt{5} = (a + bi\sqrt{5})(c + di\sqrt{5}).$$ For equality to hold, it must be that $$a^2 + 5b^2 \neq 0,\;\;\; c = \frac{a + 5b}{a^2 + 5b^2},\;\;\; \mathrm{and}\;\;\; d = -\frac{a - b}{a^2 + 5b^2}.$$ If $a = 0$, then $d = -\frac{1}{5b} \notin \mathbb{Z}$ for all $b \in \mathbb{Z}$. Likewise, if $b = 0$, then $a = -1$ or $a = 1$; moreover, $a + bi\sqrt{5} = -1$ or $a + bi\sqrt{5} = 1$, both units in the factorization above. Next, in the case that $|a| = |b| = 1$, we can see that $d \in \mathbb{Z}$ implies $a = b$ and $d = 0$, yet this provides $c = -1$ or $c = 1$, again a unit belongs to the factorization. Finally, if $|a|, |b| > 1$, we have that $c \notin \mathbb{Z}$ as $$|a + 5b| < |a^2 + 5b^2| = a^2 + 5b^2$$ and so $a^2 + 5b^2 \nmid a + 5b$. Thus, $1 + i\sqrt{5}$ is also irreducible. \\

				Suppose by contradiction that $I = (x)$ where $x \in R$. It must hold that there exist $y, z \in R$ such that $$xy = 2\;\;\; \mathrm{and}\;\;\; xz = 1 + i\sqrt{5}.$$ Recall that the only units of $R$ are $-1$ and $1$. If $x$ is not unital, then $y$ and $z$ must be units; however, this implies that $$|x| = 2\;\;\; \mathrm{and}\;\;\; |x| = \sqrt{6}.$$ Hence, $x$ is a unit, yet $I \neq R$; indeed, using elementary arguments as utilized above would illustrate that $3 \notin I$, contradicting the fact that $I = (x) = R$.

				\item[\textbf{b.}] Consider an arbitrary element in $I \cdot I$: $$(2(a + bi\sqrt{5}) + (1 + i\sqrt{5})(c + di\sqrt{5}))(2(a' + b'i\sqrt{5}) + (1 + i\sqrt{5})(c' + d'i\sqrt{5})).$$ In the expanded binomial product, we can see that all are multiplies of $2$ and elements of $R$, except possibly $$(1 + i\sqrt{5})^2(c + di\sqrt{5})(c' + d'i\sqrt{5}).$$ After further expanding, $(1 + i\sqrt{5})^2 = 2(-2 + i\sqrt{5})$ and so the latter term is also a product of $2$ and the elements of $R$. Hence, $I \cdot I \subseteq (2)$. \\

				To show that $(2) \subseteq I \cdot I$, I was having difficulty continuing. It would appear that the minimal element of $I$ in regard to magnitude is $2$, then the least element from $I \cdot I$ would be $4$, and so $2 \notin I \cdot I$ which would contradict the fact that $I \cdot I = (2)$. There must be something wrong with my reasoning.
			\end{enumerate}
		\end{proof}



	% problem 3
	\item[\textbf{3.}] Let $R$ be an integral domain and $S$ a multiplicative subset of $R$ such that $0 \notin S$. Show that if $R$ is a principal ideal domain, then so is $S^{-1}R$.

		\begin{proof}
			Let $R$ and $S \subseteq R$ be defined as stated above. Since $R$ is an integral domain and $0 \notin S$, we know that $S^{-1}R$ is also an integral domain. Let $\varphi : R \to S^{-1}R$ be the canonical homomorphism to the ring of fractions. Provided an ideal $J \subseteq S^{-1}R$, there exists a principal ideal $(r) \subseteq R$ such that $$J = \varphi((r)) = \left \{ \frac{rst}{s} : t \in R \right \}.$$ Furthermore, $J = (\varphi(r))$ is principal; indeed, for any element $\frac{rst}{s} \in J$, $$\frac{rst}{s} = \frac{st}{s} \frac{rs}{s} = \frac{ts}{s} \varphi(r)$$ since $rs^3t = rs^3t$. Thus, $S^{-1}R$ is a principal ideal domain.
		\end{proof}


\end{enumerate}

\end{document}
