\documentclass[ 12pt ]{article}
\usepackage{amsmath, amsthm, amssymb, enumitem, graphicx, listings, mathrsfs}
\usepackage[margin=0.5in]{geometry}
\graphicspath{ ./ }

\begin{document}

\noindent Landon Fox \\
\noindent Math 311 \\
\noindent October 22, 2020

\begin{center}
	\Large Homework 5
\end{center}

\begin{enumerate}
	% problem 1
	\item[\textbf{1.}] Let $A, B$ be connected subsets of $\mathbb{R}^n$ with $A \cap B \neq \varnothing$. Prove that $A \cup B$ is connected.

		\begin{proof}
			Suppose $A, B$ are connected subsets of $\mathbb{R}^n$ where $A \cap B \neq \varnothing$. Further, suppose by contradiction that $A \cup B$ is separable by relatively
			open sets $U, V$. Provided that there is a nonempty intersection between $A$ and $B$, we can see that both $U$ and $V$ must have a nonempty intersection with either
			$A$ or $B$; moreover, this is because particular elements in $A \cap B$ must belong to one of the seperating sets which illustrates that $U$ or $V$ must have an intersection
			with both sets while the other is nonempty implying that it must contain elements of at least one set. Without loss of generality, suppose $C = A \cap U \neq \varnothing$ and
			$D = A \cap V \neq \varnothing$. Observe that both $C \subseteq U$ and $D \subseteq V$ are relatively open in $A$ because the underlying open sets that create $U$ and $V$ can
			be used to intersect with $A \subseteq A \cup B$ rather than $A \cup B$. Additionally, notice that $A = C \cup D$ and $C \cap D = \varnothing$ which immediately follow from
			the fact that $U$ and $V$ are seperating sets of $A \cup B$. Thus, $A$ is separable by sets $C, D$ which is a contradiction.
		\end{proof}


	% problem 2
	\item[\textbf{2.}] Let $A \subseteq B \subseteq \mathbb{R}^n$. Prove that
		\begin{enumerate}
			\item[\textbf{a.}] $A^\circ \subseteq B^\circ$.
			\item[\textbf{b.}] $\overline{A} \subseteq \overline{B}$.
		\end{enumerate}

		\begin{proof}
			Suppose $A \subseteq B \subseteq \mathbb{R}^n$. Then it follows by Theorem 8.32i that $A^\circ \subseteq A \subseteq B \subseteq \overline{B}$.
			\begin{enumerate}
				\item[\textbf{a.}] Observe that $A^\circ$ is open by definition and Theorem 8.24. Therefore, Theorem 8.32ii states that $A^\circ \subseteq B^\circ$.

				\item[\textbf{b.}] Similarly, notice that $\overline{B}$ is closed by definition and Theorem 8.24. Hence, Theorem 8.32iii states that $\overline{A} \subseteq \overline{B}$.
			\end{enumerate}
		\end{proof}


	% problem 3
	\item[\textbf{3.}] Suppose $E \subseteq \mathbb{R}^n$ is connected and $E \subseteq A \subseteq \overline{E}$. Prove that $A$ is connected.

		\begin{proof}
			Suppose $E \subseteq \mathbb{R}^n$ is connected and $E \subseteq A \subseteq \overline{E}$. Further, suppose by contradiction that $A$ is separable by sets $U, V$. \\ \\
			Let us consider the intersection between our seperating sets and $E$. I claim that $E \cap U \neq \varnothing$. Suppose by contradiction that $E \cap U = \varnothing$.
			Since $U \subseteq U \cup V = A \subseteq \overline{E}$ it must hold that $U \subseteq \partial E$. Now consider a particular $\textbf{x} \in U$, we can see that for all
			$r > 0$, $B_r(x) \cap A \nsubseteq U$ because $\textbf{x}$ lies on the boundary of $E$ and $B_r(\textbf{x})$ will always contain elements outside $\overline{E}$. Therefore,
			$U$ cannot be relatively open in $A$ by Remark 8.27i which is a contradiction and illustrates that $E \cap U \neq \varnothing$ as claimed. By similar argument, $E \cap V \neq
			\varnothing$ must hold since $U$ was arbitrarily choosen. \\ \\
			Let $B = E \cap U$ and $C = E \cap V$. Observe that both $B \subseteq U$ and $C \subseteq V$ are relatively open
			in $E$ because the underlying open sets that create $U$ and $V$ can be used to intersect with $E \subseteq A$ rather than $A$. Additionally, notice that $A = B \cup C$ and
			$B \cap C = \varnothing$ which immediately follow from the fact that $U$ and $V$ are seperating sets of $A$. Thus, $E$ is separable by sets $B, C$ which is a
			contradiction.
		\end{proof}


	% problem 4
	\item[\textbf{4.}] Prove that $x_k = \left( \frac{k}{k+1}, \frac{1}{k} \right)$ converges.

		\begin{proof}
			Suppose $x_k = \left( \frac{k}{k+1}, \frac{1}{k} \right) \in \mathbb{R}^2$ for all $k \in \mathbb{N}$. Let $\epsilon > 0$ and $N \in \mathbb{N}$ where $N >
			\frac{\sqrt{2}}{\epsilon}$. Then it follows that $k \geq N$ implies $$\left | \left | \left( \frac{k}{k+1}, \frac{1}{k} \right) - (1, 0) \right | \right | =
			\left | \left | \left( -\frac{1}{k+1}, \frac{1}{k} \right) \right | \right | = \sqrt{ \frac{1}{(k+1)^2} + \frac{1}{k^2} } \leq \sqrt{ \frac{1}{k^2} + \frac{1}{k^2} } =
			\frac{\sqrt{2}}{k} \leq \frac{\sqrt{2}}{N} < \epsilon.$$ Hence, $x_k \to (1, 0)$ as $k \to \infty$ by definition.
		\end{proof}

\end{enumerate}

\end{document}
