\documentclass[ 12pt ]{article}
\usepackage{amsmath, amsthm, amssymb, enumitem, graphicx, mathrsfs}
\usepackage[margin=0.5in]{geometry}
\graphicspath{ ./ }

\begin{document}

\noindent Landon Fox \\
\noindent Math 311 \\
\noindent September 15, 2020

\begin{center}
\Large Homework 1
\end{center}

\begin{enumerate}
	% problem 1
	\item[\textbf{1.}] Consider the partition $P = \left \{ 0, \frac{1}{3}, 1, 2 \right \}$ of the interval $[0, 2]$ and let $f(x) = x^2 + 1$. Find the upper and lower sums of $U(f, P)$
		and $L(f, P)$. Which sum is a better approximation of the integral $\int_0^2 f(x)\; \mathrm{d}x$?

		\begin{proof}[Solution]
			Suppose $P = \left \{ 0, \frac{1}{3}, 1, 2 \right \}$ is a partition of the interval $[0, 2]$ and let $f(x) = x^2 + 1$. Observe that $f'(x) = 2x \geq 0$ for all $x \geq 0$.
			Then it follows that $f$ is increasing on $[0, 2]$ by Theorem 4.17; moreover, the supremum and infimum of $f$ on every interval within $[0, 2]$ is the rightmost and leftmost
			endpoint, respectively. Hence,
			\begin{align*}
				U(f, P) &= \sum_{j \in [3]} M_j(f) \Delta x_j \\
				&= \sum_{j \in [3]} f(x_j) \Delta x_j \\
				&= f \left ( \frac{1}{3} \right ) \cdot \frac{1}{3} + f(1) \cdot \frac{2}{3} + f(2) \cdot 1 \\
				U(f, P) &= \frac{181}{27}
			\end{align*}
			and also,
			\begin{align*}
				L(f, P) &= \sum_{j \in [3]} m_j(f) \Delta x_j \\
				&= \sum_{j \in [3]} f(x_{j-1}) \Delta x_j \\
				&= f(0) \cdot \frac{1}{3} + f \left ( \frac{1}{3} \right ) \cdot \frac{1}{3} + f(1) \cdot \frac{2}{3} \\
				L(f, P) &= \frac{55}{27}.
			\end{align*}
			Now applying the Fundamental Theorem of Calculus, we can see that $$\int_0^2 f(x)\; \mathrm{d}x = \int_0^2 (x^2 + 1)\; \mathrm{d}x = \frac{14}{3},$$ illustrating that
			the the upper sum is a better approximation.
		\end{proof}


	% problem 2
	\item[\textbf{2.}] Prove that the series $\sum_{k \geq 1} \frac{(-1)^{k+2}}{e^{k-1}}$ converges and find its sum.

		\begin{proof}
			First observe that $$\sum_{k \geq 1} \frac{(-1)^{k+2}}{e^{k-1}} = -\sum_{k \geq 1} \frac{(-1)^{k-1}}{e^{k-1}} = -\sum_{k \geq 0} \left ( \frac{-1}{e} \right )^k$$
			Let $b_k = -\left ( \frac{-1}{e} \right )^k$ and $a_k = -b_k$. Observe that $a_k$ is a geometric sequence where $|x| = \left |
			\frac{-1}{e} \right | < 1$. Hence, by the Geometric Series, $\sum_{k \geq 1} a_n$ converges to $\frac{1}{1-\left (\frac{-1}{e} \right )} = \frac{e}{1 + e}$. Thus by
			Theorem 6.10, $\sum_{k \geq 1} -a_k = \sum_{k \geq 1} b_k$ converges to the value $-\frac{e}{1 + e}$.
		\end{proof}
		\newpage


	% problem 3
	\item[\textbf{3.}] Prove that if $\sum_{k \geq 1} a_k$ converges, then its sequence of partial sums $s_n$ are bounded.

		\begin{proof}
			Suppose that $\sum_{k \geq 1} a_k$ converges and its sequence of partial sums is denoted by $s_n$. Then it follows that $\lim_{n \to \infty} s_n$ exists by the definition
			of convergence of an infinite series. Thus by Theorem 2.8, $s_n$ is a bounded sequence.
		\end{proof}


	% problem 4
	\item[\textbf{4.}] Determine whether the following series converge or diverge, prove your answer for each.
		\begin{enumerate}
			\item[\textbf{a.}] $\sum_{k \geq 1} \frac{4k^2 + 1}{3k^5 - 2k^2 + 1}$
			\item[\textbf{b.}] $\sum_{k \geq 1} \frac{k^3}{e^k}$
		\end{enumerate}

		\begin{proof}
			\begin{enumerate}
				\item[\textbf{a.}] Let $a_k = \frac{4k^2 + 1}{3k^5 - 2k^2 + 1}$ and $b_k = \frac{5}{k^3}$. Clearly, we can see that $a_k \geq 0$ for all $k \in \mathbb{N}$. Observe that
					$$a_k = \frac{4k^2 + 1}{3k^5 - 2k^2 + 1} \leq \frac{4k^2 + 1}{3k^5 - 2k^2} \leq \frac{4k^2 + 1}{3k^5 - 2k^5} \leq \frac{4k^2 + k^2}{k^5} = \frac{5}{k^3} = b_k.$$
					Further, notice that $\sum_{k \geq 1} b_k$ converges by the p-Series Test. Thus by the Comparison Test part one, $\sum_{k \geq 1} a_k$ converges.

				\item[\textbf{b.}] Let $a_k = \frac{k^3}{e^k}$ and $b_k = \frac{1}{k^2}$. Clearly, we can see that both $a_k > 0$ and $b_k > 0$ for all $k \in \mathbb{N}$. Then we can
					see that $$L = \lim_{k \to \infty} \frac{a_k}{b_k} = \lim_{k \to \infty} \frac{k^3 / e^k}{1 / k^2} = \lim_{k \to \infty} \frac{k^5}{e^k} = 0.$$ Additionally, by the
					p-Series Test the series $\sum_{k \geq 1} b_k$ must converge. Thus by the Limit Comparison Test part two, $\sum_{k \geq 1} a_k$ must also converge.
			\end{enumerate}
		\end{proof}
\end{enumerate}

\end{document}