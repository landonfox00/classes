\documentclass[ 12pt ]{article}

\usepackage{amsmath}
\usepackage{amssymb}
\usepackage{cancel}
\usepackage{tikz}
\usepackage{mathdots}
\usepackage[margin=.75in]{geometry}

\begin{document}

% title page
\title{Homework 3}
\author{Landon Fox}
\date{February 19, 2020}

\begin{flushleft}
	Landon Fox \\
	STAT 461 \\
	Section 1001 \\
	February 19, 2020
\end{flushleft}
\begin{center}
	{Homework 3 \Large}
\end{center}

\begin{itemize}

	% problem 1
	\item[] {2.6.1) \large}
	Using the multiplication rule:
	\begin{flalign}
		x &= 2 \cdot 3 \cdot 2 \nonumber \\
		&= 12 \nonumber
	\end{flalign}
	To find every combination twice:
	\begin{flalign}
		2x &= 24 \nonumber
	\end{flalign}

	% problem 2
	\item[] {2.6.3) \large}
	Using the multiplication rule, the amount of terms are:
	\begin{flalign}
		x &= 3 \cdot 3 \cdot 5 \nonumber \\
		&= 45 \nonumber
	\end{flalign}
	Only $aeu$ and $cdx$ can be found within the product. $bef$ and $xvw$ will not be found because they are not combinations
	of all three polynomials. $aeu$ and $cdx$ both have single terms from all three of the product.

	% problem 3
	\item[] {2.6.4) \large}
	\begin{itemize}

		% problem 3a
		\item[] a)
		\begin{flalign}
			x &= 26^2 \cdot 10^4 \nonumber \\
			&= 6760000 \nonumber
		\end{flalign}

		% problem 3b
		\item[] b)
		\begin{flalign}
			x &= 26^2 \cdot 10 \cdot 9 \cdot 8 \cdot 7 \nonumber \\
			&= 3407040 \nonumber
		\end{flalign}

		% problem 3c
		\item[] c)
		\begin{flalign}
			x &= 26^2 \cdot 10^4 - 26^2 \nonumber \\
			&= 6759324 \nonumber
		\end{flalign}

	\end{itemize}

	% problem 4
	\item[] {2.6.10) \large}
	Two cases: starting with a black note, or starting with a white note.
	\begin{flalign}
		x &= 5 \cdot 7 \cdot 4 \cdot 6 \cdot 3 \cdot 5 \cdot 2 \cdot 4 + 7 \cdot 5 \cdot 6 \cdot 4 \cdot 5 \cdot 3 \cdot 4 \cdot 2 \nonumber \\
		&= 201600 \nonumber
	\end{flalign}

	% problem 5
	\item[] {2.6.15) \large}
	Thirteen cards to a suit, four to a card type. If the first card is an ace of clubs, it will alter the second draw of an ace.
	\begin{flalign}
		x &= 13 \cdot 4 - 4 + 3 \nonumber \\
		&= 51 \nonumber
	\end{flalign}

	% problem 6
	\item[] {2.6.18) \large}
	\begin{itemize}
		\item[] a)
		If all tires are distinct
		\begin{flalign}
			x &= 4 \cdot 3 \cdot 2 \cdot 1 \nonumber \\
			&= 24 \nonumber
		\end{flalign}

		\item[] b)
		If two are snow tires
		\begin{flalign}
			x &= \frac{4!}{2!} \nonumber \\
			&= 12 \nonumber
		\end{flalign}
	\end{itemize}

	% problem 7
	\item[] {2.6.26) \large}
	For each placement of a family, there are many ways the family members can be arranged.
	\begin{flalign}
		x = n&(m(m-1)(m-2) \cdots (2)(1)) \nonumber \\
		\cdot &(n-1)(m(m-1)(m-2) \cdots (2)(1)) \nonumber \\
		&\ddots \nonumber \\
		\cdot 2&(m(m-1)(m-2) \cdots (2)(1)) \nonumber \\
		\cdot &(m(m-1)(m-2) \cdots (2)(1)) \nonumber
	\end{flalign}
	\begin{flalign}
		x &= n!(m!)^n \nonumber
	\end{flalign}

	% problem 8
	\item[] {2.6.28) \large}
	\begin{flalign}
		let\;\;\; P(n):\; n(n-1)(n-2) \cdots (n-k+1) = \frac{n!}{(n-k)!} \nonumber
	\end{flalign}
	Base Case:
	\begin{flalign}
		P(k):\; k(k-1)(k-2) \cdots (k-k+1) &= \frac{k!}{(k-k)!} \nonumber \\
		k(k-1)(k-2) \cdots (2)(1) &= \frac{k!}{0!} \nonumber \\
		k(k-1)(k-2) \cdots (2)(1) &= k! \nonumber
	\end{flalign}
	Based off the definition of the factorial function, the above statement holds. \\
	\newpage
	Inductive step:
	\begin{flalign}
		assume\;\;\; n(n-1)(n-2) \cdots (n-k+1) &= \frac{n!}{(n-k)!} \nonumber \\
		show\;\;\; (n+1)(n)(n-1) \cdots (n+1-k+1) &= \frac{(n+1)!}{(n+1-k)!} \nonumber \\
		(n+1)(n)(n-1)(n-2) \cdots (n-k+1)(n+1-k+1) &= \frac{(n+1)n!}{(n+1-k)(n-k)!} \nonumber \\
		\cancel{(n+1)}(n)(n-1)(n-2) \cdots (n-k+1)(n+1-k+1)(n-k+1) &= \frac{\cancel{(n+1)}n!}{(n+1-k)(n-k)!}(n-k+1) \nonumber \\
		\frac{n!}{(n-k)!} &= \frac{n!}{\cancel{(n+1-k)}(n-k)!}\cancel{(n-k+1)} \nonumber \\
		\frac{n!}{(n-k)!} &= \frac{n!}{(n-k)!} \nonumber
	\end{flalign}
	\begin{flalign}
		&P(k) \nonumber \\
		&P(n) \rightarrow P(n+1) \nonumber \\
		&\noindent\rule{2cm}{0.4pt} \nonumber \\
		&\therefore \forall n \geq k, P(n)\; \square \nonumber
	\end{flalign}

	% problem 9
	\item[] {2.6.38) \large}
	Three L's, four vowels, four consonants other than L's. \\
	What must be accounted for: four consonants, also all the different placements of the vowels and orders (including the repeating U's) \\
	Note: Stars and Bars method
	\begin{flalign}
		\binom{n+k-1}{n} \nonumber
	\end{flalign}
	Let L's and consonants be bars and all vowels be stars
	\begin{flalign}
		\cdots L \cdots L \cdots L \cdots C \cdots C \cdots C \cdots C \cdots \nonumber
	\end{flalign}
	\begin{flalign}
		x &= 4! \binom{4+8-1}{4} \frac{4!}{2!1!1!} \nonumber \\
		&= \frac{4!4!11!}{2!4!7!} \nonumber \\
		&= 95040 \nonumber
	\end{flalign}
	\newpage

	% problem 10
	\item[] {2.6.46) \large}
	Prove:
	\begin{flalign}
		\sum_{k=0}^{n} \binom{n}{k} = 2^n \nonumber
	\end{flalign}
	Note: Binomial Theorem
	\begin{flalign}
		(x+y)^n = \sum_{k=0}^{n} \binom{n}{k} x^k y^{n-k} \nonumber
	\end{flalign}
	\begin{flalign}
		let\;\;\; x&=y=1 \nonumber \\
		(1+1)^n &= \sum_{k=0}^{n} \binom{n}{k} 1^k 1^{n-k} \nonumber \\
		2^n &= \sum_{k=0}^{n} \binom{n}{k} \nonumber
	\end{flalign}

\end{itemize}

\end{document}