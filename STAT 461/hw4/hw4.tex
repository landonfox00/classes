\documentclass[ 12pt ]{article}

\usepackage{amsmath}
\usepackage{amssymb}
\usepackage{cancel}
\usepackage{tikz}
\usepackage{enumerate}
\usepackage[margin=0.5in]{geometry}
\footskip = -0.75in

\begin{document}

% title page
\title{Homework 4}
\author{Landon Fox}
\date{February 26, 2020}

\begin{flushleft}
Landon Fox \\
STAT 461 \\
Section 1001 \\
February 26, 2020
\end{flushleft}
\begin{center}
{Homework 4 \large}
\end{center}

\begin{itemize}
	% problem 2.7.6
	\item[] {2.7.6) \large}
	\begin{flalign}
		P(A) &= \frac{ \binom{2}{1}^{50} }{ \binom{100}{50} } \nonumber
	\end{flalign}

	% problem 2.7.7
	\item[] {2.7.7) \large}
	\begin{flalign}
		P(A) &= \frac{6}{6^n} \nonumber
	\end{flalign}

	% problem 2.7.11
	\item[] {2.7.11) \large}
	\begin{itemize}
		% problem 2.7.11 a
		\item[] a)
		\begin{flalign}
			P(A) &= \frac{7!}{7^n} \nonumber
		\end{flalign}

		% problem 2.7.11 b
		\item[] b)
		\begin{flalign}
			P(B) &= \frac{7}{7^n} \nonumber
		\end{flalign}

		% problem 2.7.11 c
		\item[] c)
		We are assuming each outcome is equally likely. This assumption may not be reasonable because some floors may be more visited than others.
	\end{itemize}

	% problem 2.7.15
	\item[] {2.7.15) \large}
	\begin{flalign}
		P(A) &= \frac{ \binom{k}{2} \binom{365}{k-2} }{ \binom{365}{1}^k } \nonumber
	\end{flalign}

	% problem 2.7.22
	\item[] {2.7.22) \large}
	There are $4$ aces, 10s, jacks, queens, and kings. Therefore those $20$ cards are unwanted in our probability leaving us only $32$ of a $52$ card deck.
	\begin{flalign}
		P(A) &= \frac{ \binom{32}{13} }{ \binom{52}{13} } \nonumber
	\end{flalign}

	% problem 3.2.4
	\item[] {3.2.4) \large}
	\begin{flalign}
		P(A) &= 1 - ( \binom{6}{0}(0.153)^0(1-0.153)^{6-0} + \binom{6}{1}(0.153)^1(1-0.153)^{6-1}) \nonumber \\
		&= 1 - 0.769 \nonumber \\
		&= 0.231 \nonumber
	\end{flalign}

	% problem 3.2.11
	\item[] {3.2.11) \large}
	\begin{flalign}
		P(2B) &= \binom{4}{2}(0.5)^2(0.5)^{4-2} \nonumber \\
		&= \frac{6}{16} \nonumber \\
		P(1B \cup 3B) &= \binom{4}{1}(0.5)^1(0.5)^{4-1} + \binom{4}{3}(0.5)^3(0.5)^{4-3} \nonumber \\
		&= \frac{1}{2} \nonumber
	\end{flalign}
	You are more likely to get $3$ children of one gender.

	% problem 3.2.22
	\item[] {3.2.22) \large}
	\begin{flalign}
		P(A) &= \frac{\binom{514}{k} \binom{4050-514}{65-k}}{\binom{4050}{65}} \nonumber \\
		&= \frac{\binom{514}{k} \binom{3536}{65-k}}{\binom{4050}{65}} \nonumber
	\end{flalign}

	% problem 3.2.34
	\item[] {3.2.34) \large}
	\begin{flalign}
		P(A) &= 1 - ( \frac{\binom{7}{2} \binom{7}{1} \binom{7}{1} }{ \binom{21}{4} } + \frac{\binom{7}{1} \binom{7}{2} \binom{7}{1} }{ \binom{21}{4} } + \frac{\binom{7}{1} \binom{7}{1} \binom{7}{2} }{ \binom{21}{4} } )\nonumber \\
		&= 1 - 3 \cdot \frac{\binom{7}{2} \binom{7}{1} \binom{7}{1} }{ \binom{21}{4} } \nonumber \\
		&= 1 - \frac{49}{95} \nonumber \\
		&= \frac{46}{95} \nonumber
	\end{flalign}

	% problem 10
	\item[] {10) \large}
	\begin{itemize}
		% problem 10 a
		\item[] a)
		\begin{flalign}
			P(A) &= \frac{6}{6^6} \nonumber \\
			&= \frac{1}{7776} \nonumber
		\end{flalign}

		% problem 10 b
		\item[] b)
		\begin{flalign}
			P(B) &= \frac{6!}{6^6} \nonumber \\
			&= \frac{5}{324} \nonumber
		\end{flalign}
	\end{itemize}
	\newpage

	% problem 11
	\item[] {11) \large}
	All $S_i$ have the same sample space being
	\begin{flalign}
		S_i = \{ 0, 1, 2, 3, 4, 5, 6 \} \nonumber
	\end{flalign}
	\begin{itemize}
		% problem 11 a
		\item[] a)
		\begin{flalign}
			P(Y_1=k) &= \frac{3^k \cdot 3^{6-k}}{6^6} \nonumber \\
			&= \frac{3^6}{6^6} \nonumber \\
			&= \frac{1}{64} \nonumber
		\end{flalign}

		% problem 11 b
		\item[] b)
		\begin{flalign}
			P(Y_2=k) &= \frac{1^k \cdot 5^{6-k}}{6^6} \nonumber \\
			&= \frac{5^6}{6^{6+k}} \nonumber
		\end{flalign}

		% problem 11 c
		\item[] c)
		\begin{flalign}
			P(Y_3=k) &= \frac{3^k \cdot 3^{6-k}}{6^6} \nonumber \\
			&= \frac{3^6}{6^6} \nonumber \\
			&= \frac{1}{64} \nonumber
		\end{flalign}
	\end{itemize}	

\end{itemize}

\end{document}