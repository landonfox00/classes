\documentclass[ 12pt ]{article}
\usepackage{amsmath, amsthm, amssymb, enumitem, mathtools}
\usepackage[margin=0.5in]{geometry}

\begin{document}

\noindent Landon Fox \\
\noindent Math 330 \\
\noindent July 21, 2020

\begin{center}
\Large Homework 1
\end{center}

\begin{enumerate}
	\item[\textbf{1.}] Find the equation of the parabola that passes through the points $(1,1)$, $(2,2)$, and $(3,0)$.

	\begin{proof}[Solution]\renewcommand{\qedsymbol}{}
		Suppose we have a parabola in the form $y = ax^2 + bx + c$ that passes through the points $(1,1)$, $(2,2)$, and $(3,0)$. Then we can see that the following equations
		hold
		\begin{align*}
			(1)^2a + (1)b + c &= a + b + c = 1 \\
			(2)^2a + (2)b + c &= 4a + 2b + c = 2 \\
			(3)^2a + (3)b + c &= 9a + 3b + c = 0.
		\end{align*}
		Let us now convert the system into matrix form to solve for $a$, $b$, $c$.
		\begin{align*}
			\begin{bmatrix} 1 & 1 & 1 & 1 \\ 4 & 2 & 1 & 2 \\ 9 & 3 & 1 & 0 \end{bmatrix}
			&\sim \begin{bmatrix} 1 & 1 & 1 & 1 \\ 0 & -2 & -3 & -2 \\ 0 & -6 & -8 & -9 \end{bmatrix}
			\sim \begin{bmatrix} 1 & 1 & 1 & 1 \\ 0 & 2 & 3 & 2 \\ 0 & 0 & 1 & -3 \end{bmatrix}
			\sim \begin{bmatrix} 1 & 1 & 0 & 4 \\ 0 & 2 & 0 & 11 \\ 0 & 0 & 1 & -3 \end{bmatrix}
			\sim \begin{bmatrix} 1 & 1 & 0 & 4 \\ 0 & 1 & 0 & \frac{11}{2} \\ 0 & 0 & 1 & -3 \end{bmatrix} \\
			&\sim \begin{bmatrix} 1 & 0 & 0 & -\frac{3}{2} \\ 0 & 1 & 0 & \frac{11}{2} \\ 0 & 0 & 1 & -3 \end{bmatrix}
		\end{align*}
		The reduced row echelon form of the matrix illustrates that $a=-\frac{3}{2}$, $b=\frac{11}{2}$, $c=-3$. Thus we can see that the parabola that passes through 
		$(1,1)$, $(2,2)$, and $(3,0)$ is $y = -\frac{3}{2}x^2 + \frac{11}{2}x - 3$.
	\end{proof}


	\item[\textbf{2.}] Determine the conditions for which the following system:
		\begin{align*}
			x + y + z &= 1 \\
			x + 2y - z &= b \\
			5x + 7y + az &= b^2
		\end{align*}
		admits \textbf{(i)} no solution, \textbf{(ii)} unique solution, \textbf{(iii)} infinitely many solutions.

	\begin{proof}[Solution]\renewcommand{\qedsymbol}{}
		Let us place the given system into matrix form and compute the row echelon form.
		\begin{align*}
			\begin{bmatrix} 1 & 1 & 1 & 1 \\ 1 & 2 & -1 & b \\ 5 & 7 & a & b^2 \end{bmatrix}
			\sim \begin{bmatrix} 1 & 1 & 1 & 1 \\ 0 & 1 & -2 & b-1 \\ 0 & 2 & a-5 & b^2-5 \end{bmatrix}
			\sim \begin{bmatrix} 1 & 1 & 1 & 1 \\ 0 & 1 & -2 & b-1 \\ 0 & 0 & a-1 & (b-3)(b+1) \end{bmatrix}
		\end{align*}
		\begin{enumerate}
			\item[\textbf{(i).}] If $a=1$ and $b \notin \{-1, 3\}$ we can see that the 3rd row creates an inconsistent system, thus no solution exists.
			\item[\textbf{(ii).}] Observe that if $a \neq 1$ a pivot can be found in each column (disregarding the last), illustrating a unique solution.
			\item[\textbf{(iii).}] Suppose $a=1$ and $b \in \{-1, 3\}$ we can see that the 3rd row is a tautology and the 3rd column has no pivot, thus infinitely many
				solutions exists.
		\end{enumerate}
	\end{proof}


	\item[\textbf{3.}] In order to grow a certain crop, it is recommended that each square foot of the ground be treated with 10 units of phosphorus, 9 units of potassium
		and 19 units of nitrogen. Suppose that there are three brands of fertilizer on the market - $X$; $Y$ and $Z$. One pound of $X$ contains 2; 3; 5 units of phosphorus,
		potassium and nitrogen respectively. One pound of $Y$ contains 1; 3; 4 units of phosphorus, potassium and nitrogen respectively. One pound of $Z$ contains 1; 0; 1
		units of phosphorus, potassium and nitrogen respectively.
		\begin{enumerate}
			\item[\textbf{(i).}] Determine whether or not it is possible to meet exactly the recommendations by applying some combination of the three brands of fertilizer?
			\item[\textbf{(ii).}] Take into account that a negative number of pounds of any brand cannot be applied, and suppose that each brand is sold only in integral
				amounts. Under these conditions, determine all possible combinations of the three brands that can be applied to satisfy the recommendations exactly.
			\item[\textbf{(iii).}] Suppose $X$ costs \$1 per pound, $Y$ costs \$6 per pound and $Z$ costs \$3 per pound. Determine the least expensive solution that will
				satisfy the recommendations exactly as well as the constraints in \textbf{(ii)}.
		\end{enumerate}

	\begin{proof}[Solution]\renewcommand{\qedsymbol}{}
		\begin{enumerate}
			\item[\textbf{(i).}] Let $x$, $y$, $z$ represent the number of pounds purchased from the products $X$, $Y$, $Z$ respectively. For each square foot we require
				10, 9, and 19 units of phosphorus, potassium, and nitrogen repectively. To obtain an appropiate amount, the following equations must hold
				\begin{align*}
					2x + y + z &= 10 \\
					3x + 3y &= 9 \\
					5x + 4y + z &= 19.
				\end{align*}
				Let us place the given system into matrix form and compute the reduced row echelon form.
				\begin{align*}
					\begin{bmatrix} 2 & 1 & 1 & 10 \\ 3 & 3 & 0 & 9 \\ 5 & 4 & 1 & 19 \end{bmatrix}
					\sim \begin{bmatrix} 1 & 1 & 0 & 3 \\ 2 & 1 & 1 & 10 \\ 5 & 4 & 1 & 19 \end{bmatrix}
					\sim \begin{bmatrix} 1 & 1 & 0 & 3 \\ 0 & -1 & 1 & 4 \\ 0 & -1 & 1 & 4 \end{bmatrix}
					\sim \begin{bmatrix} 1 & 1 & 0 & 3 \\ 0 & 1 & -1 & -4 \\ 0 & 0 & 0 & 0 \end{bmatrix}
					\sim \begin{bmatrix} 1 & 0 & 1 & 7 \\ 0 & 1 & -1 & -4 \\ 0 & 0 & 0 & 0 \end{bmatrix}
				\end{align*}
				Now we can see the system is consistent and the solution is as follows
				$$\begin{pmatrix} x \\ y \\ z \end{pmatrix} = \begin{pmatrix} 7 \\ -4 \\ 0 \end{pmatrix} + z\begin{pmatrix} -1 \\ 1 \\ 1 \end{pmatrix}$$
				for all $z \in \mathbb{R}$.
			\item[\textbf{(ii).}] Suppose negative number of pounds of any brand cannot be applied and that each brand is sold only in integral amounts. Moreover,
				$x, y, z \in \mathbb{N} \cup \{ 0 \}$. So then,
				\begin{align*}
					7-z=x &\geq 0 \\
					-4+z=y &\geq 0 \\
					z &\geq 0
				\end{align*}
				illustrating that $z \in \{ 4, 5, 6, 7 \}$; thus $$\begin{pmatrix} x \\ y \\ z \end{pmatrix}  = \begin{pmatrix} 7 \\ -4 \\ 0 \end{pmatrix} +
				z\begin{pmatrix} -1 \\ 1 \\ 1 \end{pmatrix} \in \left \{ \begin{pmatrix} 3 \\ 0 \\ 4 \end{pmatrix}, \begin{pmatrix} 2 \\ 1 \\ 5 \end{pmatrix},
				\begin{pmatrix} 1 \\ 2 \\ 6 \end{pmatrix}, \begin{pmatrix} 0 \\ 3 \\ 7 \end{pmatrix} \right \}.$$
			\item[\textbf{(iii).}] Suppose that the products $X$, $Y$, $Z$ have costs \$1, \$6, \$3 respectively. Due to the conditions of \textbf{(ii)}, we have a finite
				amount of solutions so we can find the cost of each combination of products. $\left(\begin{smallmatrix} 3 \\ 0 \\ 4 \end{smallmatrix}\right)$ costs \$15;
				$\left(\begin{smallmatrix} 2 \\ 1 \\ 5 \end{smallmatrix}\right)$ costs \$23; $\left(\begin{smallmatrix} 1 \\ 2 \\ 6 \end{smallmatrix}\right)$ costs \$31;
				$\left(\begin{smallmatrix} 0 \\ 3 \\ 7 \end{smallmatrix}\right)$ costs \$39. Clearly we can see that $\left(\begin{smallmatrix} 3 \\ 0 \\ 4 \end{smallmatrix}
				\right)$ is the least expensive option.
		\end{enumerate}
	\end{proof}


	\item[\textbf{4.}] Suppose that 100 insects are distributed in an enclosure consisting of 4 chambers with passageways between them. At the end of one minute, the
		insects have redistributed themselves. Assume that a minute is not enough time for an insect to visit more than one chamber and that at the end of a minute 40\%
		of the insects in each chamber have not left the chamber they occupied at the beginning of the minute. The insects that leave a chamber disperse uniformly among
		the chambers that are directly accessible from the one they initially occupied.
		\begin{enumerate}
			\item[\textbf{(i).}] If at the end of one minute, there are 12, 25, 26, 37 insects in chambers \#1, \#2, \#3, \#4 respectively. Determine the initial distribution.
			\item[\textbf{(ii).}] If the initial distribution is 20, 20, 20, 40, what is the distribution at the end of one minute.
		\end{enumerate}

	\begin{proof}[Solution]\renewcommand{\qedsymbol}{}
		\begin{enumerate}
			\item[\textbf{(i).}] Suppose at the end of one minute there are 12, 25, 26, 37 insects in chambers \#1, \#2, \#3, \#4 respectively. Let $x_i$ be the amount of
				insects in the $i$th chamber at the begining of the minute. Since we are aware of the general distribution of the insects, the following equations must hold
				\begin{align*}
					\frac{2}{5}x_1+\frac{1}{3}\cdot\frac{3}{5}x_4 &= 12 \\
					\frac{2}{5}x_2+\frac{1}{2}\cdot\frac{3}{5}x_3+\frac{1}{3}\cdot\frac{3}{5}x_4 &= 25 \\
					\frac{1}{2}\cdot\frac{3}{5}x_2+\frac{2}{5}x_3+\frac{1}{3}\cdot\frac{3}{5}x_4 &= 26 \\
					1\cdot\frac{3}{5}x_1+\frac{1}{2}\cdot\frac{3}{5}x_2+\frac{1}{2}\cdot\frac{3}{5}x_3+\frac{2}{5}x_4 &= 37. \\
				\end{align*}
				Let us place the given system into matrix form and compute the reduced row echelon form.
				\begin{align*}
					\begin{bmatrix} \frac{2}{5} & 0 & 0 & \frac{1}{5} & 12 \\ 0 & \frac{2}{5} & \frac{3}{10} & \frac{1}{5} & 25 \\ 0 & \frac{3}{10} & \frac{2}{5} & \frac{1}{5} & 26 \\ \frac{3}{5} & \frac{3}{10} & \frac{3}{10} & \frac{2}{5} & 37 \end{bmatrix}
					&\sim \begin{bmatrix} 2 & 0 & 0 & 1 & 60 \\ 0 & 4 & 3 & 2 & 250 \\ 0 & 3 & 4 & 2 & 260 \\ 6 & 3 & 3 & 4 & 370 \end{bmatrix}
					\sim \begin{bmatrix} 2 & 0 & 0 & 1 & 60 \\ 0 & 4 & 3 & 2 & 250 \\ 0 & 3 & 4 & 2 & 260 \\ 0 & 3 & 3 & 1 & 190 \end{bmatrix}
					\sim \begin{bmatrix} 2 & 0 & 0 & 1 & 60 \\ 0 & 1 & -1 & 0 & -10 \\ 0 & 3 & 4 & 2 & 260 \\ 0 & 0 & -1 & -1 & -70 \end{bmatrix} \\
					&\sim \begin{bmatrix} 2 & 0 & 0 & 1 & 60 \\ 0 & 1 & -1 & 0 & -10 \\ 0 & 0 & 7 & 2 & 290 \\ 0 & 0 & 1 & 1 & 70 \end{bmatrix}
					\sim \begin{bmatrix} 2 & 0 & 0 & 1 & 60 \\ 0 & 1 & 0 & 1 & 60 \\ 0 & 0 & 0 & -5 & -200 \\ 0 & 0 & 1 & 1 & 70 \end{bmatrix}
					\sim \begin{bmatrix} 1 & 0 & 0 & 0 & 10 \\ 0 & 1 & 0 & 0 & 20 \\ 0 & 0 & 1 & 0 & 30 \\ 0 & 0 & 0 & 1 & 40 \end{bmatrix}
				\end{align*}
				The reduced row echelon form of the matrix illustrates that $\left (\begin{smallmatrix} x_1 \\ x_2 \\ x_3 \\ x_4 \end{smallmatrix}\right) = 
				\left (\begin{smallmatrix} 10 \\ 20 \\ 30 \\ 40 \end{smallmatrix}\right)$.
				\newpage
			\item[\textbf{(ii).}] Suppose we have an initial distribution of 20, 20, 20, 40. By using the same (unaugmented) matrix constructed in \textbf{(i)} we can see
				that the product $$\begin{bmatrix} \frac{2}{5} & 0 & 0 & \frac{1}{5} \\ 0 & \frac{2}{5} & \frac{3}{10} & \frac{1}{5} \\ 0 & \frac{3}{10} & \frac{2}{5} & \frac{1}{5} \\ \frac{3}{5} & \frac{3}{10} & \frac{3}{10} & \frac{2}{5} \end{bmatrix}
				\begin{pmatrix} 20 \\ 20 \\ 20 \\ 40 \end{pmatrix} = \begin{pmatrix} 16 \\ 22 \\ 22 \\ 40 \end{pmatrix}$$ depicts the final distribution.
		\end{enumerate}
	\end{proof}


	\item[\textbf{5.}] Suppose that $\left \{\vec{v_1}, \vec{v_2}, \vec{v_3} \right \}$ is linear independent in $\mathbb{R}^n$, then show that $\left \{\vec{v_1}, \vec{v_1}+\vec{v_2},
		\vec{v_1}+\vec{v_2}+\vec{v_3} \right \}$ is also linearly independent in $\mathbb{R}^n$.

	\begin{proof}
		Suppose that $\left \{\vec{v_1}, \vec{v_2}, \vec{v_3}\right \}$ is linear independent in $\mathbb{R}^n$. Further, suppose by contradiction that $\left \{\vec{v_1}, \vec{v_1}+\vec{v_2},
		\vec{v_1}+\vec{v_2}+\vec{v_3} \right \}$ are linearly dependent in $\mathbb{R}^n$. Observe that the linear combination $c_1\vec{v_1} + c_2\vec{v_2} + c_3\vec{v_3} =
		\vec{0}$ has only the trivial solution for the constants $c_1$, $c_2$, $c_3$. Let $c_1=k_1+k_2+k_3$, $c_2=k_2+k_3$, and $c_3=k_3$ where all $k_i \in \mathbb{R}$.
		We can see that $$k_1\vec{v_1} + k_2(\vec{v_1}+\vec{v_2}) + k_3(\vec{v_1}+\vec{v_2}+\vec{v_3}) = (k_1+k_2+k_3)\vec{v_1} + (k_2+k_3)\vec{v_2} + k_3\vec{v_3} =
		c_1\vec{v_1} + c_2\vec{v_2} + c_3\vec{v_3} = \vec{0}$$ has only the trivial solution which is a contradiction. The vectors $\left \{\vec{v_1}, \vec{v_1}+\vec{v_2},
		\vec{v_1}+\vec{v_2}+\vec{v_3} \right \}$ were said to have a nontrivial solution, thus illustrating that $\left \{\vec{v_1}, \vec{v_1}+\vec{v_2}, \vec{v_1}+\vec{v_2}+\vec{v_3} \right \}$
		are linearly independent in $\mathbb{R}^n$.
	\end{proof}


	\item[\textbf{6.}] Let $T : \mathbb{R}^2 \to \mathbb{R}^2$ be a linear transformation such that each point in $\mathbb{R}^2$ is rotated counterclockwise about the origin
		by an angle $\alpha$.
		\begin{enumerate}
			\item[\textbf{(i).}] Find the standard matrix of $T$, and thus find $T(\vec{x})$ for any $\vec{x} \in \mathbb{R}^2$.
			\item[\textbf{(ii).}] Find $\alpha$ such that $T\left (\begin{smallmatrix}4 \\ 3\end{smallmatrix} \right ) = \left (\begin{smallmatrix}5 \\ 0\end{smallmatrix} \right )$.
		\end{enumerate}

	\begin{proof}[Solution]\renewcommand{\qedsymbol}{}
		\begin{enumerate}
			\item[\textbf{(i).}] Suppose a counterclockwise rotation of angle $\alpha$ occured on $\mathbb{R}^2$. By assuming it's a linear transformation we can see that
				$T_\alpha(\vec{e_1}) = \begin{pmatrix} \cos \alpha \\ \sin \alpha \end{pmatrix}$ and $T_\alpha(\vec{e_1}) = \begin{pmatrix} \cos \left (\alpha +
				\frac{\pi}{2}\right ) \\ \sin \left (\alpha + \frac{\pi}{2}\right ) \end{pmatrix} = \begin{pmatrix} -\sin \alpha \\ \cos \alpha \end{pmatrix}$. Thus
				$$T_\alpha(\vec{x}) = \begin{bmatrix} \cos \alpha & -\sin \alpha \\ \sin \alpha & \cos \alpha \end{bmatrix}.$$
			\item[\textbf{(ii).}] Suppose $T_\alpha \left (\begin{smallmatrix}4 \\ 3\end{smallmatrix} \right ) = \left (\begin{smallmatrix}5 \\ 0\end{smallmatrix} \right )$.
				It follows that $$\begin{bmatrix} \cos \alpha & -\sin \alpha \\ \sin \alpha & \cos \alpha \end{bmatrix} \begin{pmatrix}4 \\ 3\end{pmatrix} =
				\begin{pmatrix} 4\cos \alpha -3\sin \alpha \\ 3\cos \alpha + 4\sin \alpha \end{pmatrix} = \begin{pmatrix}5 \\ 0\end{pmatrix}.$$
				Now we have a system of equations where the unknows are $\cos \alpha$ and $\sin \alpha$. Creating an augmented matrix, we have
				\begin{align*}
					\begin{bmatrix} 4 & -3 & 5 \\ 3 & 4 & 0 \end{bmatrix}
					\sim \begin{bmatrix} 1 & -7 & 5 \\ 3 & 4 & 0 \end{bmatrix}
					\sim \begin{bmatrix} 1 & -7 & 5 \\ 0 & 25 & -15 \end{bmatrix}
					\sim \begin{bmatrix} 1 & -7 & 5 \\ 0 & 1 & -\frac{3}{5} \end{bmatrix}
					\sim \begin{bmatrix} 1 & 0 & \frac{4}{5} \\ 0 & 1 & -\frac{3}{5} \end{bmatrix}
				\end{align*}
				In reduced row echelon form we can see that $\cos \alpha = \frac{4}{5}$ and $\sin \alpha = -\frac{3}{5}$ illustrating that $\alpha = -\sin^{-1}\frac{3}{5}$.
		\end{enumerate}
	\end{proof}
\end{enumerate}

\end{document}