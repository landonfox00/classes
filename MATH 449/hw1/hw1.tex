\documentclass[ 12pt ]{article}
\usepackage{amsmath, amsthm, amssymb, csquotes, enumitem, graphicx, listings, mathrsfs, tikz-cd}
\usepackage[margin=0.5in]{geometry}
\graphicspath{ ./ }

\begin{document}

\noindent Landon Fox \\
\noindent Math 449, Category Theory and TQFTs \\
\noindent August 31, 2021

\begin{center}
\Large Homework 1
\end{center}

\begin{enumerate}
	% problem 1
	\item[\textbf{1.}] Let $P$ be the set of all people, and say that there is a morphism from person $A$ to person $B$ if and only if $A$ is friends with $B$. Does this describe a category? Why or why not?

		\begin{proof}
			The construction depicted above may not necessarily form a category; the composition axiom is likely violated since it may be that person $A$ is friends with $B$ and $B$ is friends with $C$, yet $A$ and $C$ are not friends.
		\end{proof}


	% problem 2
	\item[\textbf{2.}] Prove that if $\mathscr{C}$ is a category, then so is $\mathscr{C}^\mathrm{op}$.

		\begin{proof}
			Recall that the opposite category $\mathscr{C}^\mathrm{op}$ of a category $\mathscr{C}$ simply reverses all the morphisms in $\mathscr{C}$. Therefore by definition, $\mathrm{ob}(\mathscr{C}^\mathrm{op}) = \mathrm{ob}(\mathscr{C})$ and $$\mathrm{hom}(\mathscr{C}^\mathrm{op}) = \{ f^\mathrm{op} : Y \to X \mid f : X \to Y \in \mathscr{C} \}.$$ Similarly, by definition we know that for any two morphisms $f^\mathrm{op} : Z \to Y,\, g^\mathrm{op} : Y \to X \in \mathscr{C}^\mathrm{op}$, there exists their composition $g^\mathrm{op} f^\mathrm{op} = (fg)^\mathrm{op} : Z \to X$. In regard to associativity, observe that $$(h^\mathrm{op} g^\mathrm{op}) f^\mathrm{op} = (gh)^\mathrm{op} f^\mathrm{op} = (f(gh))^\mathrm{op} = ((fg)h)^\mathrm{op} = h^\mathrm{op} (fg)^\mathrm{op} = h^\mathrm{op} (g^\mathrm{op} f^\mathrm{op}).$$ Lastly, we can see that $\mathrm{id}_X^\mathrm{op} = \mathrm{id}_X : X \to X$ acts as the identity on $X \in \mathscr{C}^\mathrm{op}$; indeed, for any two morphisms $f^\mathrm{op} : X \to Y,\, g^\mathrm{op} : Y \to X \in \mathscr{C}^\mathrm{op}$, it holds that $$f^\mathrm{op} = (f \mathrm{id}_X)^\mathrm{op} = \mathrm{id}_X^\mathrm{op}f^\mathrm{op}\;\; \mathrm{and}\;\; g^\mathrm{op} = (\mathrm{id}_X g)^\mathrm{op} = g^\mathrm{op} \mathrm{id}_X^\mathrm{op}.$$
		\end{proof}


	% problem 3
	\item[\textbf{3.}] Let $G$ be a group. Prove that the category, denoted $\mathsf{B}G$, defined by $\mathrm{ob}(\mathsf{B}G) = \{ \ast \}$ and $\mathsf{B}G(\ast, \ast) = G$, $\mathrm{id}_\ast = 0_G$ the identity of $G$, whose composition is given by group multiplication is truly a category. 

		\begin{proof}
			Provided $\mathsf{B}G$ as defined above, it suffices to show associativity and identity of morphisms. Assuming $G$ is a group, it holds that $(ab)c = a(bc)$ and $a 0_G = 0_G a = a$ for all morphisms (group elements) $a, b, c \in G$.
		\end{proof}


	% problem 4
	\item[\textbf{4.}] Prove that if $G$ is a group, then every morphism in $\mathsf{B}G$ is an isomorphism.

		\begin{proof}
			Let $G$ be a group and consider a morphism $a \in G$. Consequently, $a^{-1} \in G$ and so $$aa^{-1} = a^{-1}a = 0_G = \mathrm{id}_\ast.$$ Thus, every morphism in $\mathsf{B}G$ is an isomorphism by definition.
		\end{proof}


	% problem 5
	\item[\textbf{5.}] Describe a category $\mathsf{Sq}$ such that for any other category $\mathscr{C}$, the image of a functor $F : \mathsf{Sq} \to \mathscr{C}$ is a commutative square in $\mathscr{C}$. 

		\begin{proof}
			Define the category $\mathsf{Sq}$ as the following commutative diagram.
			\begin{center}
			\begin{tikzcd}
				A \arrow[dd, "f'"'] \arrow[rr, "f"] \arrow[rr] &  & B \arrow[dd, "g"] \\
				                                               &  &                   \\
				C \arrow[rr, "g'"]                             &  & D
			\end{tikzcd}
			\end{center}
			Then for any functor $F : \mathsf{Sq} \to \mathscr{C}$ mapping to an arbitrary category $\mathscr{C}$, the image $\mathrm{im}\, F$ also commutes;
			\begin{center}
			\begin{tikzcd}
				FA \arrow[dd, "Ff'"'] \arrow[rr, "Ff"] \arrow[rr] &  & FB \arrow[dd, "Fg"] \\
				                                                  &  &                   \\
				FC \arrow[rr, "Fg'"]                              &  & FD
			\end{tikzcd}
			\end{center}
			indeed, observe that
			\begin{align*}
				gf &= g'f' \\
				F gf &= F g'f' \\
				Fg\, Ff &= Fg'\, Ff'.
			\end{align*}
		\end{proof}


	% problem 6
	\item[\textbf{6.}] Prove that a set $X$ has a left $G$ action if and only if there is a functor $F : \mathsf{B}G \to \mathsf{Set}$ such that $F \ast = X$.

		\begin{proof}
			Suppose there is a functor $F : \mathsf{B}G \to \mathsf{Set}$ such that $F \ast = X \in \mathsf{Set}$. Let $\alpha : \mathrm{hom}(\mathsf{B}G) \to \mathrm{hom}(\mathsf{Set})$, defined as $\alpha g = Fg$, denote $F$'s mapping of morphisms. Then by the functoriality of $F$, it holds that $$\alpha(\mathrm{id}_\ast, x) = F \mathrm{id}_\ast x = \mathrm{id}_X x = x$$ and $$\alpha(g, \alpha(h, x)) = \alpha(g, Fhx) = Fg(Fhx) = (Fg\, Fh)x = F(gh)x = \alpha(gh, x)$$ for all $g, h \in G$ and $x \in X$. \\

			Conversely, suppose we have a left $G$ action $\alpha : G \times X \to X$ on a set $X$. Define a functor $F : \mathsf{B}G \to \mathsf{Set}$ such that $F \ast = X$ and $Fg = \alpha g : X \to X$ for all $g \in G$. To verify the functoriality of $F$, observe that $$F \mathrm{id}_\ast = \alpha \mathrm{id}_\ast = \mathrm{id}_X$$ and $$Fg\, Fh = \alpha g\, \alpha h = \alpha(gh) = F gh,$$ as desired.
		\end{proof}

\end{enumerate}

\end{document}
