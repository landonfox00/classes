\documentclass[ 12pt ]{article}
\usepackage{amsmath, amsthm, amssymb, enumitem, graphicx, listings, mathrsfs, tabularx}
\usepackage[margin=0.5in]{geometry}
\graphicspath{ ./ }

\begin{document}

\noindent Landon Fox \\
\noindent CS 456 \\
\noindent October 4, 2020

\begin{center}
	\Large Quiz 3
\end{center}

\begin{enumerate}
	% problem 1
	\item[\textbf{1.}] Is $L(r_1) = L(r_2)$ where $r_1 = (ab^* + c)(\lambda + \emptyset)$ and $r_2 = (a + c)(b^* + \emptyset)$? What are the languages of $L(r_1)$ and $L(r_2)$?

		\begin{proof}
			Suppose we have an alphabet $\Sigma = \{ a, b, c \}$. Further, suppose we have regular expressions $r_1 = (ab^* + c)(\lambda + \emptyset)$ and $r_2 = (a + c)(b^* +
			\emptyset)$. It is obvious that $r_1$ and $r_2$ are in fact regular expressions by Definition 3.1; moreover, observe that they stem from primitive regular expressions $a$,
			$b$, $c$, $\lambda$, and $\emptyset$, applying well defined operations upon them that guarantee closure. \\

			I claim that $L(r_1) \neq L(r_2)$. To show that two languages are not equal, it suffices to provide an element that belongs to one set but not the other. Furthermore,
			it is easy to see that the string $cb$ belongs to $L(r_2)$ but does not belong to $L(r_1)$ (this will be further illustrated below). \\

			Let us now compute $L(r_1)$ and $L(r_2)$. Since $r_1$ and $r_2$ are regular expressions, we can apply Definition 3.2. Then it follows that
			\begin{align*}
				L(r_1) &= L( (ab^* + c)(\lambda + \emptyset) ) \\
				&= L(ab^* + c) L(\lambda + \emptyset) \\
				&= ( L(ab^*) \cup L(c) ) ( L(\lambda) \cup L(\emptyset) ) \\
				&= ( L(a)L(b^*) \cup \{ c \} ) ( \{ \lambda \} \cup \emptyset ) \\
				&= ( \{ a \} L(b)^* \cup \{ c \} ) \{ \lambda \} \\
				&= ( \{ a \} \{ b \}^* \cup \{ c \} ) \{ \lambda \} \\
				&= ( \{ a, ab, abb, \hdots \} \cup \{ c \} ) \{ \lambda \} \\
				&= \{ c, a, ab, abb, \hdots \} \{ \lambda \} \\
				L(r_1) &= \{ c, a, ab, abb, \hdots \}
			\end{align*}
			and also,
			\begin{align*}
				L(r_2) &= L( (a + c)(b^* + \emptyset) ) \\
				&= L(a + c) L(b^* + \emptyset) \\
				&= ( L(a) \cup L(c) ) ( L(b^*) \cup L(\emptyset) ) \\
				&= ( \{ a \} \cup \{ c \} ) ( L(b)^* \cup \emptyset ) \\
				&= \{ a, c \} ( \{ b \}^* \cup \emptyset ) \\
				&= \{ a, c \} \{ b \}^* \\
				L(r_2) &= \{ a, ab, abb, \hdots, c, cb, cbb, \hdots \}.
			\end{align*}
			Now we can clearly see that $cb$, as well as $cbb$, $cbbb$, $\hdots$ all belong to $L(r_2)$ but not $L(r_1)$.
		\end{proof}
		\newpage


	% problem 2
	\item[\textbf{2.}]
		\begin{enumerate}
			\item[\textbf{a.}] Explain the four formal steps used in \textit{nfa} $\to$ \textit{dfa} conversions.
			\item[\textbf{b.}] Convert the given \textit{nfa} $$M_n = (\{q_0, q_1, q_2, q_3\}, \{a, b\}, \delta_n, q_0, \{q_1\})$$
				to a \textit{dfa} using the shortcut conversion technique.
		\end{enumerate}

		\begin{proof}[Solution]
			\begin{enumerate}
				\item[\textbf{a.}] To convert a nfa to a dfa, perform the following algorithm:
					\begin{enumerate}
						\item[\textbf{i.}] Create a vertex $\{ q_0 \}$ which will act as the initial vertex of our dfa.
						\item[\textbf{ii.}] For a missing edge of a vertex $\{ q_i, q_j, \hdots, q_k \}$ (i.e if $\Sigma = \{ a, b\}$ then vertex $\{ q_i, q_j, \hdots, q_k \}$ needs a
							transition for both $a$ and $b$),
							\begin{itemize}
								\item Compute $\delta^*\left( \{ q_i, q_j, \hdots, q_k \}, a \right)$ = $\{ q_\ell, q_m, \hdots, q_n \}$, where $a$ is the missing transition.
								\item Create the vertex $\{ q_\ell, q_m, \hdots, q_n \}$ if it doesn't already exist.
								\item Draw the edge $a$ from $\{ q_i, q_j, \hdots, q_k \}$ to $\{ q_\ell, q_m, \hdots, q_n \}$.
							\end{itemize}
							Repeat this process until all edges of all vertices exist.
						\item[\textbf{iii.}] For every vertex in our dfa that contains a final vertex of the nfa, assign that vertex in our dfa as a final vertex.
						\item[\textbf{iv.}] If the nfa accepts $\lambda$, then make the vertex $\{ q_0 \}$, the initial vertex, a final vertex as well.
					\end{enumerate}

				\item[\textbf{b.}] Observe that
					\begin{center}
					\begin{tabularx}{0.5\textwidth}{
						>{\raggedright\arraybackslash}X
						>{\raggedright\arraybackslash}X }
						$\delta^* \left( \{ q_0 \}, a \right) = \{ q_1, q_2 \}$ & $\delta^* \left( \{ q_0 \}, b \right) = \{ q_0, q_3 \}$ \\
						$\delta^* \left( \{ q_1 \}, a \right) = \{ q_1, q_2 \}$ & $\delta^* \left( \{ q_1 \}, b \right) = \{ q_0, q_1 \}$ \\
						$\delta^* \left( \{ q_2 \}, a \right) = \{ q_1, q_3 \}$ & $\delta^* \left( \{ q_2 \}, b \right) = \emptyset$ \\
						$\delta^* \left( \{ q_3 \}, a \right) = \emptyset$ & $\delta^* \left( \{ q_3 \}, b \right) = \emptyset$.
					\end{tabularx}
					\end{center}
					Now considering the newly added states $\{ q_0, q_1 \}$, $\{ q_0, q_3 \}$, $\{ q_1, q_2 \}$, $\{ q_1, q_3 \}$, and $\emptyset$, we have
					\begin{center}
					\begin{tabularx}{1\textwidth}{
						>{\raggedright\arraybackslash}X
						>{\raggedright\arraybackslash}X }
						$\delta^* \left( \{ q_0, q_1 \}, a \right) = \{ q_1, q_2 \} \cup \{ q_1, q_2 \} = \{ q_1, q_2 \}$ & $\delta^* \left( \{ q_0, q_1 \}, b \right) = \{ q_0, q_3 \} \cup \{ q_0, q_1 \} = \{ q_0, q_1, q_3 \}$ \\
						$\delta^* \left( \{ q_0, q_3 \}, a \right) = \{ q_1, q_2 \} \cup \emptyset = \{ q_1, q_2 \}$ & $\delta^* \left( \{ q_0, q_3 \}, b \right) = \{ q_0, q_3 \} \cup \emptyset = \{ q_0, q_3 \}$ \\
						$\delta^* \left( \{ q_1, q_2 \}, a \right) = \{ q_1, q_2 \} \cup \{ q_1, q_3 \} = \{ q_1, q_2, q_3 \}$ & $\delta^* \left( \{ q_1, q_2 \}, b \right) = \{ q_0, q_3 \} \cup \emptyset = \{ q_0, q_1 \}$ \\
						$\delta^* \left( \{ q_1, q_3 \}, a \right) = \{ q_1, q_2 \} \cup \emptyset = \{ q_1, q_2 \}$ & $\delta^* \left( \{ q_1, q_3 \}, b \right) = \{ q_0, q_1 \} \cup \emptyset = \{ q_0, q_1 \}$ \\
						$\delta^* \left( \emptyset, a \right) = \emptyset$ & $\delta^* \left( \emptyset, b \right) = \emptyset$.
					\end{tabularx}
					\end{center}
					Again, with the new states $\{ q_0, q_1, q_3 \}$, $\{ q_1, q_2, q_3 \}$, we obtain
					\begin{center}
					\begin{tabularx}{1\textwidth}{
						>{\raggedright\arraybackslash}X
						>{\raggedright\arraybackslash}X }
						$\delta^* \left( \{ q_0, q_1, q_3 \}, a \right) = \{ q_1, q_2 \} \cup \{ q_1, q_2 \} \cup \emptyset = \{ q_1, q_2 \}$ & $\delta^* \left( \{ q_0, q_1, q_3 \}, b \right) = \{ q_0, q_3 \} \cup \{ q_0, q_1 \} \cup \emptyset = \{ q_0, q_1, q_3 \}$ \\
						$\delta^* \left( \{ q_1, q_2, q_3 \}, a \right) = \{ q_1, q_2 \} \cup \{ q_1, q_3 \} \cup \emptyset = \{ q_1, q_2, q_3 \}$ & $\delta^* \left( \{ q_1, q_2, q_3 \}, b \right) = \{ q_0, q_1 \} \cup \emptyset \cup \emptyset = \{ q_0, q_1 \}$.
					\end{tabularx}
					\end{center}
					We can then see that our dfa is
					\begin{align*}
						M_d = (\;& \\
						&\{ \{q_0\}, \{q_0, q_1\}, \{q_0, q_3\}, \{q_1, q_2\}, \{q_0, q_1, q_3\}, \{q_1, q_2, q_3\} \}, \\
						&\{ a, b \}, \\
						&\delta_d, \\
						&\{ q_0 \}, \\
						&\{ \{q_0\}, \{q_0, q_1\}, \{q_1, q_2\}, \{q_0, q_1, q_3\}, \{q_1, q_2, q_3\} \} \\
						).\;&
					\end{align*}
			\end{enumerate}
		\end{proof}
\end{enumerate}

\end{document}