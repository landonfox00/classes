\documentclass[ 12pt ]{article}
\usepackage{amsmath, amsthm, amssymb, enumitem, graphicx, listings, mathrsfs, xcolor}
\usepackage[margin=0.5in]{geometry}
\graphicspath{ ./ }

\begin{document}

\noindent Landon Fox \\
\noindent CS 456 \\
\noindent December 15, 2020

\begin{center}
	\Large Final Exam
\end{center}

\begin{enumerate}
	% problem 1
	\item[\textbf{1.}] Define these finite state machines: deterministic (dfa) and nondeterministic finite automata (nfa); nondeterministic pushdown automata (npda); Turing machines
		(tm). Explain each component and describe the differences between machines.

		\begin{proof}[Solution]
			Let us begin by stating the definition of each machine.
			\begin{enumerate}
				\item[\textbf{dfa.}] A deterministic finite automata is defined by $$M = (Q, \Sigma, \delta, q_0, F),$$ where
					\begin{itemize}
						\item $Q$ is the finite set of internal states.
						\item $\Sigma$ is the finite set of symbols called the input alphabet.
						\item $\delta : Q \times \Sigma \to Q$ is a total function called the transition function.
						\item $q_0 \in Q$ is the initial state.
						\item $F \subseteq Q$ is the set of final states.
					\end{itemize}

				\item[\textbf{nfa.}] A nondeterministic finite automata is defined by $$M = (Q, \Sigma, \delta, q_0, F),$$ where
					\begin{itemize}
						\item $Q$ is the finite set of internal states.
						\item $\Sigma$ is the finite input alphabet.
						\item $\delta : Q \times (\Sigma \textcolor{red}{\cup \{ \lambda \}}) \to \textcolor{red}{2^Q}$ is the transition function.
						\item $q_0 \in Q$ is the initial state.
						\item $F \subseteq Q$ is the set of final states.
					\end{itemize}

				\item[\textbf{npda.}] A nondeterministic pushdown automata is defined by $$M = (Q, \Sigma, \Gamma, \delta, q_0, z, F),$$ where
					\begin{itemize}
						\item $Q$ is the finite set of internal states of the control unit.
						\item $\Sigma$ is the finite input alphabet.
						\item $\textcolor{red}{\Gamma}$ is the finite \textcolor{red}{stack} alphabet.
						\item $\delta : Q \times (\Sigma \cup \{ \lambda \}) \textcolor{red}{\times \Gamma} \to$ \textcolor{red}{set of finite subsets of $Q \times \Gamma^*$} is the
							transition function.
						\item $q_0 \in Q$ is the initial state of the control unit.
						\item $\textcolor{red}{z} \in \Gamma$ is the stack start symbol.
						\item $F \subseteq Q$ is the set of final states.
					\end{itemize}

				\item[\textbf{tm.}] A Turing machine is defined by $$M = (Q, \Sigma, \Gamma, \delta, q_0, \square, F),$$ where
					\begin{itemize}
						\item $Q$ is the set of internal states.
						\item $\Sigma \textcolor{red}{\subseteq \Gamma \setminus \{ \square \}}$ is the finite input alphabet.
						\item $\Gamma$ is the finite \textcolor{red}{tape} alphabet.
						\item $\delta : \textcolor{red}{Q \times \Gamma} \to \textcolor{red}{Q \times \Gamma \times \{ L, R \}}$ is the transition function ($\textcolor{red}{L}$ and
							$\textcolor{red}{R}$ represent the left and right movement of the \textcolor{red}{read-write head}, respectively).
						\item $q_0 \in Q$ is the initial state.
						\item $\textcolor{red}{\square} \in \Gamma$ is referred to as the blank.
						\item $F \subseteq Q$ is the set of final states.
					\end{itemize}
			\end{enumerate}

			Overall, the main differences between these machines is their methods of memory storage and their levels of determinism. The dfa is the the simplest of all listed machines;
			this is because it has no memory other than its finite states. Similarly, an nfa also uses only its finite states as memory. What distinguishes these two machines is the
			determinism of how the machines will react to a given input; for a dfa, we can determine the exact course of action for any given string while it is unclear how an nfa may
			react since it may transition to a variety of finite states at any step. What is interesting specifically between these two is that they generate the same languages despite
			their varying levels of determinism. In regard to the more sophisticated machines, both the npda and tm utilize data structures to store infinite memory which allows them to
			generate far more complex languages. Moreover, an npda uses a stack while a tm uses a tape. Additionally, it is notable that the npda is nondeterministic which is a
			similarity both the nfa and npda share. A unique detail of tms is that they have both an input and output (although the lines between them are often blurred) unlike all other
			listed machines since they are simply acceptors.
		\end{proof}


	% problem 2
	\item[\textbf{2.}] Prove that not all languages are regular.

		\begin{proof}
			Let $\Sigma = \{ a, b \}$ and consider the language $L = \{ a^n b^n : n \in \mathbb{N} \cup \{ 0 \} \}$. Suppose by contradiction that all languages are regular and so $L$
			must also. Clearly, $L$ is an infinite language. Let $w = a^m b^m$ where $m \in \mathbb{N}$ is the pumping length. We can see that $|w| = 2m \geq m$ which implies that $w$
			can be decomposed into $w = xyz$ with $|xy| \leq m$ and $|y| \geq 1$ such that $w_i = xy^iz \in L$ for all $i \in \mathbb{N} \cup \{ 0 \}$ by the pumping lemma. Consider
			the string $w_2 = xy^2z$ where $|y| = k \in \mathbb{N}$. If $y$ is entirely composed of a single character (i.e. $y = a^k$ or $y = b^k$), then $$n_a(w_2) = m + k > m =
			n_b(w_2)$$ or $$n_a(w_2) = m < m + k = n_b(w_2)$$ and so $w_2 \notin L$. In the case that $y$ contains both $a$'s and $b$'s; in other words, $y = a^{k_1}b^{k_2}$ where $k_1
			+ k_2 = k$, then $w_2 = a^m b^{k_2} a^{k_1} b^{k_2 + m} \notin L$ which is a contradiction. Thus, not all languages are regular and $L$ is an example of such.
		\end{proof}


	% problem 3
	\item[\textbf{3.}] Design a Turing machine that accepts the language $L = \{ w \in \{ a, b \}^+ : n_a(w) = 2k, k \in \mathbb{N} \cup \{ 0 \} \}$.

		\begin{proof}[Solution]
			Suppose $\Sigma = \{ a, b \}$ and $L = \{ w \in \{ a, b \}^+ : n_a(w) = 2k, k \in \mathbb{N} \cup \{ 0 \} \}$. Let us being by developing a strategy to construct a Turing
			machine that accepts $L$. With the assumption that we ban the null string, our machine must accept a string that is unaffected by the number of $b$'s present and will
			reject all strings with an odd number of $a$'s. Furthermore, for every $a$ encountered, we shall cycle between two states, only allowing a transition to the final state
			from the initial state since zero $a$'s is acceptable. Then we will remain at the current state for every $b$ that is witnessed. Additionally, let us move the read-write
			head to the right after every character that is read. \\

			Consider the Turing machine $$M = (\{ q_0, q_1, q_f \}, \{ a, b \}, \{ a, b, \square \}, \delta, q_0, \square, \{ q_f \}).$$ For $\delta$ to suit our needs, we need $q_0$
			to act as the \textit{even} state and $q_1$ the \textit{odd} state. Then, $q_0$ must be able to transition to itself via $b$, transition to the odd state via $a$, and
			transition to the final state via a blank. Hence,
			\begin{align*}
				\delta( q_0, a ) &= ( q_1, a, R ), \\
				\delta( q_0, b ) &= ( q_0, b, R ), \\
				\delta( q_0, \square ) &= ( q_f, \square, R ).
			\end{align*}
			Now, for the odd state, we need to transition to the even state via $a$ and remain in its current state if a $b$ is encountered. We need to ensure that there is no
			transition from the odd state to the final state. This provides
			\begin{align*}
				\delta( q_1, a ) &= ( q_0, a, R ), \\
				\delta( q_1, b ) &= ( q_1, b, R ).
			\end{align*}

		\end{proof}


	% problem 4
	\item[\textbf{4.}] Explain the Universal Turing machine and how it works.

		\begin{proof}[Solution]
			A Universal Turing machine is essentially a reprogrammable Turing machine. Moreover, a Universal Turing machine $M_u$ is a three tape Turing machine such that, given a
			description of a Turing machine $M$ and an input string $w$, $M_u$ can perform the computation of $M$ on $w$. The three tapes of $M_u$ are designed to act as the
			description, internal state, and tape contents of $M$. For these to tapes to hold this information, we must have an encoding of $\delta_M$, the current internal state
			of $M$, as well as any input string $w$. We can make use of an encoding using 1s and 0s for every state in $Q_M$ and every character in $\Gamma_M$ and place each of the
			binary expressions on their respective tape. Then $M_u$ can read all three tapes, reading the description and internal state to determine the next action given the current
			character from the input string.
		\end{proof}


	% problem 5
	\item[\textbf{5.}] Turing's thesis states that a Turing machine can compute any effective procedure. The implication is that there exists languages that are not computable. Justify
		this claim. \\

		\textbf{Lemma}: \textbf{[Cantor's Theorem]}. For any set $A$, it holds that $|A| < |2^A|$.

		\begin{proof}[Lemma Proof]
			First we want to show that we can injectively map a set $A$ to $2^A$. This is easy to show, map every element of $A$ to the singleton containing themselves (i.e. let $f(a) =
			\{ a \}$ for all $a \in A$). Then it must hold that $|A| \leq |2^A|$. Now let us show that $A$ cannot bijectively map to $2^A$. Suppose by contradiction that there does
			exist a bijective mapping of $A$ to $2^A$, denoted $f : A \to 2^A$. Consider a set $B \subseteq A$ that is defined as the elements of $A$ that do not map to a subset of $A$
			in $2^A$ that they are apart of; in other words, $B = \{ a : a \in A, a \notin f(a) \}$. Since $f$ is bijective and $B \subseteq A$, there must exist an element $b \in A$
			such that $f(b) = B$. If $b \in B$, then $f(b) \neq B$ which is a contradiction. Then it must hold that $b \notin B$; however, if that is the case then it must also hold
			that $b \in B$ by definition, again, a contradiction. Thus, $|A| \neq |2^A|$ and so $|A| < |2^A|$.
		\end{proof}

		\begin{proof}
			Since algorithms are equivalent to languages in regard to Turing machines and Turing machines have an injective correspondence to languages, it suffices to show that there
			are more languages than Turing machines then apply Turing's thesis. \\

			Consider an alphabet $\Sigma = \{ a, b \}$. Further, consider the set of all languages constructed from $\Sigma$. It follows that the set of languages from $\Sigma$ is 
			$2^{\Sigma^*}$. As a bijective mapping, for every string in $\Sigma^*$ convert every $a$ to 0 and $b$ to 1, then translate the resulting binary value to decimal. The
			resulting set will be an infinite subset of $\mathbb{N}$ and so $\Sigma^*$ is countable by definition. \\

			Provided that $\Sigma^*$ is countable, it follows by Cantor's Theorem that $2^{\Sigma^*}$ has strictly greater cardinality of $\Sigma^*$. Moreover, $|\mathbb{N}| =
			|\Sigma^*| < |2^{\Sigma^*}|$; therefore, $2^{\Sigma^*}$ is uncountable. \\

			Now, let us turn to the set of all Turing machines. We know that Turing machines have an encoding to binary values (encode $\Gamma$ using 1s, then create a scheme to encode
			every transition of $\delta$, as referenced \textbf{4}). Let this encoding along with the conversion of binary to decimal be a bijection to a subset of natural numbers.
			Then it follows that the set of all Turing machines are countable. \\

			Due to the fact that the set of all Turing machines is countable, its cardinality must not exceed $|\mathbb{N}|$. However, the set of all languages on $\Sigma$ has
			cardinality exceeding $|\mathbb{N}|$ because it is uncountable. Thus, there exists languages that do not have a corresponding Turing machine by the Pigeon Hole Principle.
			Then by Turing's strong thesis, it follows that those languages that do not have a Turing machine are uncomputable. 
		\end{proof}


	% problem 6
	\item[\textbf{6.}]
		\begin{enumerate}
			\item[\textbf{a.}] Define recursive and recursively enumerable languages and their associated finite state machines.
			\item[\textbf{b.}] State the Halting Problem and explain how it is related to computability.
		\end{enumerate}

		\begin{proof}[Solution]
			\begin{enumerate}
				\item[\textbf{a.}] Recursively enumerable languages are all languages that are accepted by Turing machines. Recursive languages are similarly generated by Turing
					machines; however, a language $L$ on $\Sigma$ is said to be recursive if and only if there exists a Turing machine $M$ such that $M$ generates $L$ and every string
					$w \in \Sigma^+$ halts in $M$, regardless of whether $w \in L$. The main distinction between these two types of languages is the fact that a Turing machine for a
					recursively enumerable language may not halt for a string that belongs to its complement; whereas, a Turing machine for a recursive language will halt for all
					strings constructed from the language's alphabet.

				\item[\textbf{b.}] The Halting Problem is defined as: Given a description of a Turing machine $M$ and a string $w$, does $M$, given the initial configuration $q_0w$,
					perform a computation on $w$ that eventually halts? What is interesting about this question is that it is undecidable; in other words, there does not exist a Turing
					machine such that it can determine whether a given Turing machine and string will halt in computation. Observe that this problem cannot be answered by directly
					performing the computation of $M$ on $w$ as we cannot be guarantee that it has not entered an infinite loop. A remarkable result of this question is that there does
					exist problems which are undecidable and so there exists algorithms that are not computable.
			\end{enumerate}
		\end{proof}
\end{enumerate}

\end{document}
