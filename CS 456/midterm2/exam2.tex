\documentclass[ 12pt ]{article}
\usepackage{amsmath, amsthm, amssymb, enumitem, graphicx, listings, mathrsfs}
\usepackage[margin=0.5in]{geometry}
\graphicspath{ ./ }

\begin{document}

\noindent Landon Fox \\
\noindent CS 456 \\
\noindent November 22, 2020

\begin{center}
	\Large Exam 2
\end{center}

\begin{enumerate}
	% problem 1
	\item[\textbf{1.}] Prove that context-free languages are closed under star-closure.

		\begin{proof}
			Let $L$ be a context-free language generated by the grammar $G = (V, T, S, P)$. Additionally, let $\widehat{G} = (V \cup \{ \widehat{S} \}, T, \widehat{S}, \widehat{P})$
			where $\widehat{S} \notin V$ and $\widehat{P} = P \cup \{ \widehat{S} \to S\widehat{S} | \lambda \}$. I claim that $L(\widehat{G}) = L^*$. Observe that $\widehat{S}
			\overset{*}{\Rightarrow} SS \hdots S$ providing an arbitrary number of $S$'s where $S \overset{*}{\Rightarrow} u$ for any $u \in L$. Furthermore, $w \in L(\widehat{G})$ must
			have the form $w = u_1 u_2 \hdots u_n$ for a general $n \in \mathbb{N} \cup \{ 0 \}$ where $u_i \in L$ and so $w \in L^*$ by definition. Conversely, if $w \in L^*$ then $w =
			u_1 u_2 \hdots u_n$. Using the same argument, there exists a derivation $\widehat{S} \overset{*}{\Rightarrow} SS \hdots S$ for $n$ $S$'s and the $i$th $S$ can have the
			derivation $S \overset{*}{\Rightarrow} u_i$. Moreover, $\widehat{S} \overset{*}{\Rightarrow} u_1 u_2 \hdots u_n$ illustrating that $w \in L(\widehat{G})$. Thus,
			$L(\widehat{G}) = L^*$ which implies that $L^*$, for a general $L$, is context-free by definition.
		\end{proof}


	% problem 2
	\item[\textbf{2.}] Explain how to prove that $L = \{ a^n b^n : n \geq 0, n \neq 456 \}$ is a context-free language.

		\begin{proof}[Solution]
			Suppose $\Sigma = \{ a, b \}$ and $L = \{ a^n b^n : n \geq 0, n \neq 456 \}$. As shown in class, we know that $L_1 = \{ a^n b^n : n \geq 0 \}$ is a context-free language.
			Additionally, the language $L_2 = \{ a^{456} b^{456} \}$ is finite so it must be regular. Furthermore, due to the closure of complementation of regular languages,
			$\overline{L_2}$ must also be regular. Thus, $L = L_1 \cap \overline{L_2}$ is context-free by Theorem 8.5.
		\end{proof}


	% problem 3
	\item[\textbf{3.}] Create a npda for the language $L = \{ a^n b^m : n \neq m \}$.

		\begin{proof}[Solution]
			Let $\Sigma = \{ a, b \}$ and $L = \{ a^n b^m : n \neq m \}$. To create a npda that generates the language $L$, we need our machine to push all $a$'s onto the stack then pop
			them off for every encountered $b$. Then to determine if a string is accepted, an $a$ must be a top the stack when we have exhausted all characters in our string or there
			must remain a $b$ in our string while the stack is empty (disregarding the start stack symbol). \\

			Furthermore, let our npda be defined as $$M = \{ \{ q_0, q_1, q_f \}, \Sigma, \Sigma \cup \{ z \}, \delta, q_0, z, \{ q_f \} \}.$$ To have $\delta$ suit our needs, we must
			first push all $a$'s onto the stack,
			\begin{align*}
				\delta( q_0, a, z ) &= \{ (q_0, az) \}, \\
				\delta( q_0, a, a ) &= \{ (q_0, aa) \}.
			\end{align*}
			Next, we need to transition to the acceptance of $b$'s. Since $n = 0$ or $m = 0$ (but not both), we require $\lambda$-transitions
			\begin{align*}
				\delta( q_0, \lambda, a ) &= \{ (q_1, a) \}, \\
				\delta( q_0, \lambda, z ) &= \{ (q_1, z) \}.
			\end{align*}
			Now we accept all $b$'s, removing an $a$ from the stack for each $b$ encountered, which provides
			\begin{align*}
				\delta( q_1, b, a ) &= \{ (q_1, \lambda) \}.
			\end{align*}
			Finally, we must determine if $n \neq m$. If there still exists an $a$ on the stack, then $n > m$; similarly, if a $b$ remains to be read and there are no $a$'s on the stack,
			then $n < m$ and so
			\begin{align*}
				\delta( q_1, \lambda, a ) &= \{ (q_f, a) \}, \\
				\delta( q_1, b, z ) &= \{ (q_f, z) \}, \\
				\delta( q_f, b, z ) &= \{ (q_f, z) \}.
			\end{align*}
		\end{proof}


	% problem 4
	\item[\textbf{4.}] Show that $L = \{ ww : w \in \{ a, b \}^* \}$ is not context-free language.

		\begin{proof}
			Suppose $\Sigma = \{ a, b \}$ and $L = \{ ww : w \in \{ a, b \}^* \}$. Further, suppose by contradiction that $L$ is a context-free language. Clearly $L$ is an infinite
			language and so the Pumping Lemma for context-free language is justifiable. Let $s = ww = a^m b^m a^m b^m \in L$. The pumping lemma informs us that $|s| = 4m \geq m$ implies
			that $s = uvxyz$ with $|vxy| \leq m$ and $|vy| \geq 1$. Consider the pumped string $s_0 = uxz \in L$ by assumption. Without loss of generality, suppose $v$ contains at least
			one $a$ since $|vy| \geq 1$. Then for $s_0 \in L$ to hold, an $a$ must be removed from the other half of the string (namely, the opposite $w$); however, $|vxy| \leq m$ which
			implies that an $a$ from the opposite $w$ cannot be contained in $vxy$ and so it cannot be removed which is a contradiction.
		\end{proof}
\end{enumerate}

\end{document}