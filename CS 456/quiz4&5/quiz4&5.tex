\documentclass[ 12pt ]{article}
\usepackage{amsmath, amsthm, amssymb, enumitem, graphicx, listings, mathrsfs}
\usepackage[margin=0.5in]{geometry}
\graphicspath{ ./ }

\begin{document}

\noindent Landon Fox \\
\noindent CS 456 \\
\noindent October 11, 2020

\begin{center}
	\Large Quiz 4/5
\end{center}

\begin{enumerate}
	% problem 1
	\item[\textbf{1.}] Find a regular grammar for $L = \{ a^n b^m : mn = 2k, k \in \mathbb{N} \cup \{ 0 \} \}$ given $\Sigma = \{ a, b \}$ and justify your answer.

		\begin{proof}
			Suppose $\Sigma = \{ a, b \}$ and let $L =\{ a^n b^m : mn = 2k, k \in \mathbb{N} \cup \{ 0 \} \}$. Now consider the following regular grammar $G = ( \{ S, A, B, C, D \},
			\Sigma, S, P )$ where $P$ is defined as the set of productions
			\begin{align*}
				S &\to A | C \\
				A &\to aaA | B \\
				B &\to bB | \lambda \\
				C &\to aC | D \\
				D &\to bbD | \lambda.
			\end{align*}
			Clearly we can see that $G$ is a regular grammar because it is a right-linear grammar. \\
			Let us now show that $L = L(G)$. Beginning with production $S$, there is a choice between $A$ and $C$ which both lead to a sequence of $a$'s followed by $b$'s. Observe that
			for $mn$ to be even, either $m$ or $n$ must be even. Furthermore, by choosing $A$, an even number of $a$'s must be chosen then an arbitrary number of $b$'s follow. Similarly,
			$C$ takes an arbitrary number of $a$'s then an even number of $b$'s. Hence, any string in $L$ must also belong to $L(G)$ and vice versa illustrating with $L = L(G)$.
		\end{proof}


	% problem 2
	\item[\textbf{2.}]
		\begin{enumerate}
			\item[\textbf{i.}] Explain the pumping lemma for regular languages.
			\item[\textbf{ii.}] Suppose $\Sigma = \{ a, b, c \}$, prove or disprove that $L = \{ a^n b^m c^{m + n} : m, n \geq 0 \}$ is regular.
		\end{enumerate}

		\begin{proof}
			\begin{enumerate}
				\item[\textbf{i.}] An extention of the pigeonhole principle, the pumping lemma for infinite regular languages, states that for any infinite regular language $L$, there
					exists a $p \in \mathbb{N}$ (labeled the pumping length) such that for any string $w \in L$ with $|w| \geq p$ then $w = xyz$ where $|xy| \leq p$ and $|y| \geq 1$ such
					that $w_k = xy^kz$ is also a member of $L$ for all $k \in \mathbb{N} \cup \{ 0 \}$. \\
					The pumping lemma states that for any infinite regular language then a very specific string $w$ that belongs to the language can be used to generate more string in
					the language by \textit{pumping up} or \textit{pumping down} the $y$ substring within the decomposition $w = xyz$ to $w_k = xy^kz$ for any value of $k$. The pumping
					lemma is based on the principle that infinite regular languages must have cycles such that it can either be skipped or repeated any number of times. Additionlly,
					the lemma is a foundational result regarding identification of nonregular languages; using a proof by contradiction, the pumping lemma can be used to show that pumped
					strings do not in fact belong to the language which illustrates that the given infinite regular language is not regular by assumption.

				\item[\textbf{ii.}] Suppose $\Sigma = \{ a, b, c \}$ and let $L = \{ a^n b^m c^{m + n} : m, n \geq 0 \}$. I claim that $L$ is not regular. Suppose by contradiction that
					$L$ is in fact regular. Clearly $L$ is an infinite language and regular by assumption, then by the pumping lemma there must exist a pumping length $p \in \mathbb{N}$
					for any string in our language. Let $w = b^p c^p \in L$ where $|w| = 2p \geq p$. Then it follows that $y = b^k$ for $1 \leq k \leq p$ since $|xy| \leq p$ and $|y|
					\geq 1$. Now, observe that a defining feature of $L$ is the fact that for any $s \in L$, $n_a(s) + n_b(s) = n_c(s)$. If we were to pump $w$ down to $w_0 = b^{p - k}
					c^p$ we can then see that $$n_a(w_0) + n_b(w_0) = p - k \neq p = n_c(w_0).$$ Thus, $w_0 \notin L$ which is a contradiction.
			\end{enumerate}
		\end{proof}

\end{enumerate}

\end{document}