\documentclass[ 12pt ]{article}
\usepackage{amsmath, amsthm, amssymb, enumitem, graphicx, listings, mathrsfs}
\usepackage[margin=0.5in]{geometry}
\graphicspath{ ./ }

\begin{document}

\noindent \textbf{Group 3}: Kyle Shelton, Landon Fox, Luke Heinrichs, Ryan Walker \\
\noindent Math 466 \\
\noindent October 4, 2020

\begin{center}
	\Large Programming Project 1: Foias Constant
\end{center}

\begin{enumerate}
	% problem 1
	\item[\textbf{1.}] \textit{Group collaboration} \\
		We distributed the work required for this project by questions. Kyle and Ryan worked on problems \textbf{2}-\textbf{4} and Landon and Luke worked on \textbf{6}-\textbf{10}.
		We all collaborated together to work on the extra credit questions. As a group we held a total of three meetings for this project where we met on Zoom. The meetings were
		conducted as follows:
		\begin{enumerate}
			\item[\textbf{Meeting 1.}] Our first meeting was September 12th, where we first met and divided up the work amongst us. We also took time to look over and discuss all
				questions. We gave suggestions and tips on how certain questions could be approached.
			\item[\textbf{Meeting 2.}] Our second meeting was on September 19th, where we checked the progress that everyone had made. For the most part, both pairs of groupings had
				all but a one or two questions finished already and the others were in progress. A few questions were asked such as how to take limits of recurrences and how many
				iterations were needed to obtain a sufficient approximation using the bisection method.
			\item[\textbf{Meeting 3.}] At this point we had finished all questions and were looking to optimize them. Our third meeting was our final meeting where we went over our
				code and explained how it performed. A few small adjustments were made but nothing substantial.
		\end{enumerate}
		The submitted report represents the original efforts of the team members listed (Kyle, Landon, Luke, and Ryan) and, in particular, does NOT contain the work of other students
		in the class.

	% problem 2
	\item[\textbf{2.}] Given $\alpha \in [1, 2]$ consider the recurrence relation $$x_{n+1} = \left ( 1 + \frac{1}{x_n} \right )^n;\;\; x_1 = \alpha.$$
		Set $\alpha = 1$ and write a program that computes and outputs $x_n$ for $n = 1, 2, \hdots, 50$.

		\begin{proof}[Solution]\renewcommand{\qedsymbol}{}
			Observe the following Julia code under the module \verb|Sequence|.
			\begin{lstlisting}[basicstyle=\ttfamily\footnotesize, numbers=left, tabsize=4, frame=single]
module Sequence

	export x, x_even, x_odd, output

	# x_init: global initial value of sequence.
	# 	used to varify if initial value has changed
	x_init = 1
	# x_values: used to store all values of sequence.
	# 	nth value of x_values is the nth term of sequence
	x_values = []

	# x: returns the nth term of sequence with initial condition x_1
	#	also stores all terms into x_values
	function x( x_1, n )
		global x_init, x_values
		# if initial value changes, remove all perviously computed values
		# 	to avoid conflicts
		if x_init != x_1
			x_init = x_1
			x_values = []
		end
		len = length( x_values )
		# if length of x_values exceeds n, then nth term already lies in x_values
		if len >= n
			return x_values[ n ]
		end
		# if x_values is empty, append initial value
		if len == 0
			push!( x_values, x_1 )
			len = 1
		end
		# let recursive helper return computed value
		return x_helper( n, len )
	end

	# x_helper: recursive helper to x function
	# 	computes nth term and stores all into x_values
	function x_helper( n, len )
		# base case: if length of x_values is equal to n, then return the value
		if n == len
			return x_values[ n ]
		end
		# value does not exist in x_values, compute (len + 1)th term and repeat
		push!( x_values, ( 1 + 1/x_values[ len ] )^len )
		return x_helper( n, len + 1 )
	end

	# x_even: returns (2n)th term of sequence
	x_even( x_1, n ) = x( x_1, 2 * n )

	# x_odd: returns (2n-1)th term of sequence
	x_odd( x_1, n ) = x( x_1, 2 * n - 1 )

	# output: prints n terms of sequence with initial condition x_1
	function output( x_1, n )
		x( x_1, n )
		for i in 1:n
			println( i, " : ", x_values[ i ] )
		end
	end

end
			\end{lstlisting}
			In the above code, the function \verb|x| uses a recursive helper function \verb|x_helper| and a list \verb|x_values| to compute the $n$th term of the sequence given
			the initial condition \verb|x_1|. More specifically, \verb|x_values| is used to store previously calculated values to ensure a complexity of $O(n)$ as opposed to a higher
			complexity when listing all values of the sequence upto $n$. Additionally, functions \verb|x_even| and \verb|x_odd| return $x_{2n}$ and $x_{2n - 1}$, respectively, for any
			$n \geq 1$. To output all values upto index $n$, the function \verb|output| can be used; computing the values of $x_n$ where $x_1 = 1$ for $n = 1, 2, \hdots, 50$, can be
			done via \verb|Sequence.output( 1, 50 )| which provides:
			\begin{lstlisting}[basicstyle=\ttfamily\footnotesize, tabsize=4, frame=single]
 1 : 1
 2 : 2.0
 3 : 2.25
 4 : 3.0137174211248285
 5 : 3.1461328651361042
 6 : 3.9749418398300125
 7 : 3.8436459461081727
 8 : 5.0466489745895124
 9 : 4.2471077065229235
10 : 6.705641594809635
11 : 4.014991943105427
12 : 11.546255851437039
13 : 2.7094151558090824
14 : 59.37293543097047
15 : 1.2634346914781713
16 : 6283.875266743144
17 : 1.0025492407161776
18 : 128267.71346883844
19 : 1.0001403407948934
20 : 523589.5387780858
21 : 1.0000381985532598
22 : 2.0963110569382557e6
23 : 1.000010494678316
24 : 8.387595658020036e6
25 : 1.0000028613721854
26 : 3.355323187755784e7
27 : 1.0000007748883877
28 : 1.3421632395741531e8
29 : 1.00000020861846
30 : 5.3686928798555815e8
31 : 1.0000000558795226
32 : 2.1474817879952881e9
33 : 1.0000000149011754
34 : 8.589932479998269e9
35 : 1.000000003958121
36 : 3.435973598800008e10
37 : 1.000000001047738
38 : 1.3743895080800003e11
39 : 1.0000000002764864
40 : 5.49755810924e11
41 : 1.0000000000727596
42 : 2.199023252272e12
43 : 1.0000000000190994
44 : 8.796093018596e12
45 : 1.0000000000050022
46 : 3.5184372084872e13
47 : 1.0000000000013074
48 : 1.40737488351004e14
49 : 1.000000000000341
50 : 5.62949953416608e14
			\end{lstlisting}
		\end{proof}


	% problem 3
	\item[\textbf{3.}] Based on the numerical evidence in the previous step, make a conjecture regarding the value of the limits $$\lim_{n \to \infty} x_{2n}\;\;\;\;\; \mathrm{and}
		\;\;\;\;\; \lim_{n \to \infty} x_{2n+1}\;\;\;\;\; \mathrm{when}\;\; x_1 = 1.$$ Explain your reasoning in as much mathematical detail as possible.

		\begin{proof}[Solution]\renewcommand{\qedsymbol}{}
			By inspecting the values above, it would appear as if $$\lim_{n \to \infty} x_{2n} = \infty \;\;\;\;\; \mathrm{and}\;\;\;\;\; \lim_{n \to \infty} x_{2n+1} = 1\;\;\;\;\;
			\mathrm{when}\;\; x_1 = 1.$$ Observe that for any index $n > 1$, $x_n$ takes the multiplicative reciprocal of the last value, adds one to it, and raises it to the power of
			$n$. Suppose the previous value, $x_{n-1}$, is very large, then we can see that $1 + \frac{1}{x_{n-1}}$ will become very close to 1 implying that $x_n = \left ( 1 +
			\frac{1}{x_n} \right )^n$ will not increase greatly from 1. Now consider the case where $x_{n-1}$ is very small, it is easy to see that $1 + \frac{1}{x_{n-1}}$ will exceed
			1 which illustrates that it will become very substantial when raised to the $n$th power. For our particular sequence, it can be seen that $x_n$ slowly increases until
			roughly index $n=12$ where this oscillation begins to take place; to enumerate, $x_{12} \approx 11.5$ which results in $x_{13} \approx 2.7$ then in turn leads $x_{14}
			\approx 59.4$ where this pattern of increasing and decreasing continues ensuring $x_{2n}$ and $x_{2n+1}$ diverges to infinity and converges to 1, respectively.
		\end{proof}


	% problem 4
	\item[\textbf{4.}] Change the program in the previous step to compute the values of $x_n$ when $\alpha = 2$. Look at the output and now make a conjecture regarding the value of
		the limits $$\lim_{n \to \infty} x_{2n}\;\;\;\;\; \mathrm{and}\;\;\;\;\; \lim_{n \to \infty} x_{2n+1}\;\;\;\;\; \mathrm{when}\;\; x_1 = 2.$$ Explain the reasoning behind your
		conjecture.

		\begin{proof}[Solution]\renewcommand{\qedsymbol}{}
			Due to the generality of the \verb|x| function under the \verb|Sequence| module, no changes are needed from its original design. To change the initial value of the sequence,
			simply change the value of the \verb|x_1| parameter. Computing and presenting the first fifty values of the sequence $x_n$ where $x_1 = 2$ can be done via
			\verb|Sequence.output( 2, 50 )| which provides:
			\begin{lstlisting}[basicstyle=\ttfamily\footnotesize, tabsize=4, frame=single]
 1 : 2
 2 : 1.5
 3 : 2.7777777777777772
 4 : 2.515456000000001
 5 : 3.8146945245827055
 6 : 3.202910914929112
 7 : 5.105425602411104
 8 : 3.4977433630005117
 9 : 7.475734025618049
10 : 3.0953609877250656
11 : 16.436481006193045
12 : 1.9149171433392773
13 : 154.77995881694767
14 : 1.0873244478087112
15 : 9230.628993458884
16 : 1.001626257595803
17 : 64689.92026265085
18 : 1.0002628246012701
19 : 261524.77368382475
20 : 1.0000726533613016
21 : 1.0478144924485891e6
22 : 1.0000200419068173
23 : 4.193379435467222e6
24 : 1.0000054848507496
25 : 1.6776111794572083e7
26 : 1.0000014902152625
27 : 6.710756392755206e7
28 : 1.000000402339224
29 : 2.6843394397513205e8
30 : 1.000000108034036
31 : 1.0737400839916067e9
32 : 1.0000000288710484
33 : 4.2949653119971633e9
34 : 1.0000000076834112
35 : 1.7179866940000143e10
36 : 1.0000000020372681
37 : 6.8719474216000046e10
38 : 1.0000000005384209
39 : 2.74877904132e11
40 : 1.0000000001418812
41 : 1.099511624656e12
42 : 1.0000000000372893
43 : 4.39804650766e12
44 : 1.000000000009777
45 : 1.7592186040632e13
46 : 1.000000000002558
47 : 7.0368744173524e13
48 : 1.000000000000668
49 : 2.81474976706144e14
50 : 1.000000000000174
			\end{lstlisting}
			By inspecting the values above, it would appear as if $$\lim_{n \to \infty} x_{2n} = 1 \;\;\;\;\; \mathrm{and} \;\;\;\;\; \lim_{n \to \infty} x_{2n+1} = \infty\;\;\;\;\;
			\mathrm{when}\;\; x_1 = 2.$$ Similar to \textbf{3}, if the sequence creates a somewhat large value then we can conlude that an oscillation will likely occur resulting
			in the subsequence of one parity to diverge and the other to converge to 1. Now observe that for index $n=11$, $x_{11} \approx 16.4$ exceeds most previous values and starts
			the visible difference the subsequences. Consequently, this can be seen around index $n=50$ where odd indices are extremely large and even indices approach 1 from above.
		\end{proof}


	% problem 5
	\item[\textbf{5.}] Use rigorous mathematical analysis to prove the conjectures made in \textbf{3} and \textbf{4}.

		\begin{proof}
			Since $x_{n+1}=(1+\frac{1}{x_n})^n$, then we can begin by looking at $\frac{1}{x_n}$. Knowing that $n\rightarrow\infty$ we know that $x_n$ keeps changing, and we will initially assume that it continues to increase. If $x_n$ increases, then by taking the recipricole, $\frac{1}{x_n}$ decreases and by the definition of convergence we know that it will converge to $0$. Thus pluging this into the initial equation, we know that as $x_n$ increases, $\frac{1}{x_n}\rightarrow0$, and thus $(1+\frac{1}{x_n})\rightarrow1$ and taking these to the $n^{th}$ power still shows $(1+\frac{1}{x_n})^n\rightarrow1$. Alternatively, if $x_n$ decreases (specifically as $x_n\rightarrow1$) then the recipricole $\frac{1}{x_n}\rightarrow1$ and thus $(1+\frac{1}{x_n})\rightarrow2)$. Then, taking the $n^{th}$ power of $2$ as $n\rightarrow\infty$ gives us an increasing sequence that diverges to $\infty$. Plugging these back into the original equation, we see that $x_{n+1}$ becomes an alternating series, with each $x_n$ term that is small producing a large $x_{n+1}$ term and each $x_n$ term that is large producing a small $x_{n+1}$ term. Thus by taking all the even terms, $x_{2n}$, of this sequence we will only collect a subset that has the smallest terms that are decreasing (due to the increasing terms we aren't collecting), and as proven above they must converge to $1$. Similarly we collect all the odd, $x_{2n+1}$, terms which gives us all the largest values that are diverging to $\infty$. Thus the $\lim_{n\rightarrow\infty}x_{2n}=1$ and $\lim_{n\rightarrow\infty}x_{2n+1}=\infty$.
		\end{proof}


	% problem 6
	\item[\textbf{6.}] Define $$\alpha_* = \sup \{ \alpha : |x_{2n+1}|\; \mathrm{is\; bounded\; as}\; n \to \infty \}.$$ Intuitively $\alpha_*$ is the largest value of $\alpha$ such
		that $|x_{2n+1}|$ is bounded. Explain in details how the interval bisection method could be used to approximate $\alpha_*$.

		\begin{proof}[Solution]\renewcommand{\qedsymbol}{}
			Suppose that $|x_{2n+1}|$ is bounded for the entirety of a nondegenerate interval of $\alpha$. Further, suppose we obtain two particular values of $\alpha$ where the
			lesser value has the sequence diverge and the greater requires it to be bounded as $n \to \infty$. As we would approximate a root of a polynomial, we can apply the
			bisection method to estimate $\alpha_*$. The algorithm would be defined as follows:
			\begin{itemize}
				\item Calculate the midpoint of our interval containing the infimum and determine if our sequence is divergent using that value.
				\item If the left endpoint is convergent and the midpoint is divergent or the left endpoint is divergent and the midpoint is divergent then let the midpoint be the
					new right endpoint. In any other condition, let midpoint be the new left endpoint.
				\item Repeat the process with the new subinterval until an appropriate precision is achieved.
			\end{itemize}
		\end{proof}


	% problem 7
	\item[\textbf{7.}] Write a computer program that bisects the interval $[1, 2]$ to find an approximation of $\alpha_*$ good to at least four significant digits. Include the full
		program listing and output.

		\begin{proof}[Solution]\renewcommand{\qedsymbol}{}
			Observe the following Julia code under the module \verb|Bisection|.
			\begin{lstlisting}[basicstyle=\ttfamily\footnotesize, numbers=left, tabsize=4, frame=single]
module Bisection

	export bisect

	# bisect: performs the bisection method returning approximated value
	#	n: number of decimal places of accuracy
	#	[a, b]: closed interval
	#	s: sequence
	#	f: bisection condition (divergence of the sequence)
	function bisect( n, a, b, s, f )
		# c: midpoint
		c = 0
		# ceil( n * log2( 10 ) ) is the number of iterations required for
		# 	n decimals of accuracy
		for i in 1:ceil( n * log2( 10 ) )
			c = ( a + b ) / 2
			f_a = f( a, s ); f_c = f( c, s )
			# if ( s_a conv and s_c div ) or ( s_a div and s_c conv ), then look
			# 	to left subinterval
			if !f_a && f_c || f_a && !f_c
				b = c
			# otherwise, look to right subinterval
			else
				a = c
			end
		end
		# return approximated value
		return c
	end

end
			\end{lstlisting}
			Additionally, the following function is of importance.
			\begin{lstlisting}[basicstyle=\ttfamily\footnotesize, numbers=left, tabsize=4, frame=single]
# div: determines if sequence s diverges with initial value s_1
# 	computes first few indices to see if they become large
function div( s_1, s )
	for i in 1:50
		if s( s_1, i ) > 100
			return true
		end
	end
	return false
end
			\end{lstlisting}
			In the above code, the function \verb|bisect| performs the entirety of the bisection method returning the approximated value to the proper accuracy. The \verb|div|
			determines if a provided sequence will diverge with a given initial condition. The function is passed as a parameter to \verb|bisect| as the bisection condition.
			Computing $\alpha_*$ to four decimal places can be done via \\ \verb|Bisection.bisect( 4, 1, 2, Sequence.x_odd, div )| which provides 1.18743896484375.
		\end{proof}


	% problem 8
	\item[\textbf{8.}] Explain theoretically how many times the interval $[1, 2]$ needs to be bisected to ensure the resulting approximation is good to at least four significant digits.
		What if six significant digits are desired? How about eight significant digits?

		\begin{proof}[Solution]\renewcommand{\qedsymbol}{}
			Suppose we have the interval $[1, 2]$. If the value of interest lies within this interval then we can guarentee it to be $\frac{3}{2}$ with an error of $\frac{1}{2}$.
			Then after the next iteration, we cut the interval in half and so our error is also cut in half. Similarly, it can be seen that our error is $\frac{1}{2^k}$ after $k$
			iterations of the algorithm. Hence, if we desire our approximation to have $n$ decimals of precision then to determine the minimal number of interations we have the
			following inequality $$\frac{1}{2^k} < 10^{-n}.$$ Then it follows that $k \geq \left \lceil n \lg 10 \right \rceil$. Thus, $k = \left \lceil n \lg 10 \right \rceil$
			provides an optimal result. \\

			To obtain four significant digits, only three decimal places are required illustrating that at least 10 iterations are required. Likewise, 17 and 24 iterations are
			required for six and eight significant digits, respectively.
		\end{proof}


	% problem 9
	\item[\textbf{9.}] Define $$\alpha^* = \inf \{ \alpha : |x_{2n}|\; \mathrm{is\; bounded\; as}\; n \to \infty \}.$$ Intuitively $\alpha^*$ is the largest value of $\alpha$ such that
		$|x_{2n}|$ is bounded. Modify the program from \textbf{7} to obtain an approximation of $\alpha^*$. Include the output and describe what modifications were made to the code.

		\begin{proof}[Solution]\renewcommand{\qedsymbol}{}
			Due to the generality of the \verb|bisect| function under the \verb|Bisection| module, no changes are needed from its original design. Computing $\alpha^*$ to four decimal
			places can be done via \\ \verb|Bisection.bisect( 4, 1, 2, Sequence.x_even, div )| which provides 1.18743896484375.
		\end{proof}


	% problem 10
	\item[\textbf{10.}] Based on the numerical evidence obtained in the previous two problems conjecture whether the values of $\alpha_*$ and $\alpha^*$ are equal or different.

		\begin{proof}[Solution]\renewcommand{\qedsymbol}{}
			Based on our numerical results it seems fair to conjecture that $\alpha_*$ and $\alpha^*$ are in fact equal. In other words, it would appear that $|x_{2n}|$ and $|x_{2n+1}|$
			are both bounded exclusively when the initial condition is $\alpha_* = \alpha^* \approx 1.18745235112650$.
		\end{proof}


	% problem 11
	\item[\textbf{11.}] Use rigorous mathematical analyisis to prove the conjecture stated in \textbf{10}.

		\begin{proof}
			Since $|x_{2n+1}|$ and $|x_{2n}|$ are both bounded, then we know that they must not diverge, so we must find a way to find the supremum or infemum of $\alpha$ such that $|x_{2n+1}|$ and $|x_{2n}|$ don't diverge. To do this, we will take the bisection method, starting with the largest bounded interval and noting that for one endpoint, the sequence converges and at the other it diverges. Using Newtons Binomial Theorem, we take the midpoint and note whether it also diverges or converges, then shrinking the interval so that we always have one endpoint diverging and one converging. Since we are dealing with significant digits, we can do this until the endpoints are the same point with significant digits and thus recieve our answer. As previously proven, if $|x_{2n+1}|$ diverges, then $|x_{2n}|$ will converge and vice versa. Thus as done similarly to Riemman Sums, we know that for an alternating series $|x_n|$ if $|x_{2n+1}|$ diverges and $|x_{2n}|$ converges then $\sup\{|x_{2n+1}|\}=\inf\{|x_{2n}|\}$ as required.
		\end{proof}

\end{enumerate}

\end{document}