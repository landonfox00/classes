\documentclass[ 12pt ]{article}
\usepackage{amsmath, amsthm, amssymb, enumitem, graphicx, listings, mathrsfs}
\usepackage[margin=0.5in]{geometry}
\graphicspath{ ./ }

\begin{document}

\noindent Landon Fox \\
\noindent Math 485 \\
\noindent September 23, 2020 \\
\noindent \textbf{Collaborated with Logan Leavitt}

\begin{center}
	\Large Homework 4
\end{center}

\begin{enumerate}
	% problem 1
	\item[\textbf{1.}] A six-sided die is rolled repeatedly until all the numbers one through six have appeared at least once each. What is the probability that this happens in the first
		$n$ rolls? Find the first $n$ where the probability exceeds $\frac{1}{2}$.

		\begin{proof}[Solution]
			Suppose a die is rolled $n$ times. Let us consider $S$ the set of all possible strings generated from the die. Additionally, let $A_i$ for $i \in [6]$ denote the set of all
			strings where the $i$th face does not occur. We can then see that
			\begin{align*}
				|S| &= 6^n \\
				|A_i| &= 5^n;\;\;\; \mathrm{for\; all}\; i \in [6] \\
				|A_{i_1} \cap A_{i_2}| &= 4^n;\;\;\; \mathrm{for\; all}\; i_1 \neq i_2 \in [6] \\
				&\vdots \\
				|A_{i_1} \cap \hdots \cap A_{i_k}| &= (6 - k)^n;\;\;\; \mathrm{for\; all}\; i_1 \neq \hdots \neq i_k \in [6].
			\end{align*}
			Now we seek the cardinality of $\bigcap_{i \in [6]} \bar{A_i}$, where all strings must include all six faces of the die. By the Principle of Inclusion-Exclusion
			we obtain $$\left | \bigcap_{i \in [n]} \bar{A_i} \right | = \sum_{k = 0}^6 (-1)^k \binom{6}{k} (6-k)^n = 6^n - 6 \cdot 5^n + 15 \cdot 4^n - 20 \cdot 3^n + 15 \cdot 2^n - 6.$$
			To obtain the likelyhood of this occuring, we divide by the cardinality of the sample space which provides $$\frac{1}{6^n} \sum_{k = 0}^6 (-1)^k \binom{6}{k} (6-k)^n =
			\frac{6^n - 6 \cdot 5^n + 15 \cdot 4^n - 20 \cdot 3^n + 15 \cdot 2^n - 6}{6^n}.$$ \\

			After inspection of the given probability density function, we can see that the probability will exceed $\frac{1}{2}$ after $n \geq 13$.
		\end{proof}


	% problem 2
	\item[\textbf{2.}] Suppose that $n$ soldiers march in a single column such that everyone (except the front one) can see only the person in front of him/her. How many ways are there
		to arrange them the next day such that everyone sees someone different than the one they saw today?

		\begin{proof}[Solution]
			Suppose $n$ soldiers march in a line such that no individual can see anyone other than the person directly in front of them. Let us now consider the marching arrangements
			they can make the next day. Let $A_i$ for $i \in [n]$ denote the arrangements such that the $i$th solider can see the individual that was in front of them previously.
			Now we seek to find the cardinality of $\bigcap_{i \in [n]} \bar{A_i}$, where all soldiers do not see the individual they did last. Observe that $|A_1| = 0$ since the first
			solider had no individual standing in front of them. Additionally, notice that if we desire the $i$th individual to stand behind the person they were previously, then there
			are $(n-1)$ positions for the individual in front to stand in and $(n-2)!$ ways to permute all others. Furthermore, if we group the $i$th individual and the person in front
			of them as a inserperable unit, then we can see that $$|A_{i_1} \cap \hdots \cap A_{i_k}| = (n-k)!$$ for all not equal $i_j \in [n] \setminus \{ 1 \}$. Then by the
			Principle of Inclusion-Exclusion we obtain $$\left | \bigcap_{i \in [n]} \bar{A_i} \right | = \sum_{k = 0}^{n-1} (-1)^k \binom{n-1}{k} (n-k)!$$ where we remove 1st individual
			since the first individual can be placed behind any other person.
		\end{proof}


	% problem 3
	\item[\textbf{3.}] A mathematics department has $n$ professors and $2n$ courses, two assigned to each professor each semester.
		\begin{enumerate}
			\item[\textbf{i.}] How many ways are there to assign them in the Fall semester?
			\item[\textbf{ii.}] How many ways are there to assign them in the Spring semester so that no professor teaches the same two courses in the Spring as in the Fall?
		\end{enumerate}

		\begin{proof}[Solution]
			Suppose there are $n$ professors and $2n$ courses to be taught where two are given to each professor.
			\begin{enumerate}
				\item[\textbf{i.}] Let us imagine that the professors write each of the $2n$ courses onto a piece of paper and let the first professor take the first two, the second take
					the next two, and so on. Since it makes no difference to any professor whether they take courses $A$ and $B$ or $B$ and $A$, it is clear that the number of ways
					classes can be assigned is $\frac{(2n)!}{2^n}$.

				\item[\textbf{ii.}] Suppose the professors were assigned courses in the Fall and they were to be assigned courses again in the Spring. Let $A_i$ for $i \in [n]$ denote
					the set of course assignments such that the $i$th professor teaches the same courses. Clearly we can see that $$|A_{i_1} \cap \hdots \cap A_{i_k}| =
					\frac{(2n - 2k)!}{2^{n-k}}$$ for all not equal $i_j \in [n]$ since we are not arranging the $k$ specified professors' classes. For each professor to teach
					different courses we seek the set $\bigcap_{i \in [n]} \bar{A_i}$. By the Principle of Inclusion-Exclusion we obtain $$\left | \bigcap_{i \in [n]} \bar{A_i} \right |
					= \sum_{k = 0}^n (-1)^k \binom{n}{k} \frac{(2n - 2k)!}{2^{n - k}}.$$
			\end{enumerate}
		\end{proof}


	% problem 4
	\item[\textbf{4.}] At a circular table there are $n$ students taking an exam. The exam has four versions. How many ways are there to assign the exams so that no two neighboring
		students get the same version?

		\begin{proof}[Solution]
			Suppose we have a circular table with $n$ students taking an exam with four distinct versions. Let $A_i$ for $i \in [n]$ denote the set of seating arrangments such that the
			student sitting in the $i$th chair will have the same exam as the student to their right. Assuming that student distinction will not be counted, we can see that $|A_i| =
			4^{n - 1}$ for all $i \in [n]$ since the $i$th individual has four options and all others except $i$ and his neighbor have $4^{n-2}$ Furthermore, we can see that
			$$|A_{i_1} \cap \hdots \cap A_{i_k}| = 4^{n - k}$$ for all not equal $i_j \in [n]$. For each student to not have neighbors with the same exam, then we seek the
			set $\bigcap_{i \in [n]} \bar{A_i}$. By the Principle of Inclusion-Exclusion we obtain $$\left | \bigcap_{i \in [n]} \bar{A_i} \right | = \sum_{k = 0}^n (-1)^k \binom{n}{k}
			4^{n - k}.$$ We can then simplify the sum by using Principle of Inclusion-Exclusion once again, noticing that it is equiviliant to number of strings of base four but without
			a fourth character; in other words, we are counting the number of ternary strings. Thus, the count is $3^n$.
		\end{proof}


	% problem 5
	\item[\textbf{5.}]
		\begin{enumerate}
			\item[\textbf{i.}] Prove that $$\sum_{k = 0}^n (-1)^k \binom{n}{k}(n-k)^n = n!.$$
			\item[\textbf{ii.}] Evaluate $\sum_{k = 0}^n (-1)^i \binom{n}{i}(n-i)^k$, when $0 \leq k < n$, $k = n$, and $k > n$.
		\end{enumerate}

		\begin{proof}
			\begin{enumerate}
				\item[\textbf{i.}] Suppose we have base $n$ strings with length $n$. Let $A_i$ for $i \in [n]$ denote the set of all strings where the $i$th character does not occur.
					Clearly, we can see that $|A_i| = (n - 1)^n$ and also, $$|A_{i_1} \cap \hdots \cap A_{i_k}| = (n-k)^n$$ for all not equal $i_j \in [n]$ which gives the number strings
					absent of $k$ specified characters. Then by the Principle of Inclusion-Exclusion, we can see that $\sum_{k = 0}^n (-1)^k \binom{n}{k}(n-k)^n$ represents the number
					of strings with all $n$ characters which is clearly a permutation of $n$ characters. Thus, $$\sum_{k = 0}^n (-1)^k \binom{n}{k}(n-k)^n = n!.$$

				\item[\textbf{ii.}] As we did in \textbf{i}, let us adopt the same sets of strings but generalize the string length to $k$. Then we can see that $\sum_{k = 0}^n
					(-1)^i \binom{n}{i}(n-i)^k$ has a similar connotation but with string length $k$. If $0 \leq k < n$ then we can see that we need to have all $n$ characters
					but we are limited to $k$, thus the sum must evaluate to zero. In the case that $k = n$, then we have the sum seen in \textbf{i} which provides $n!$. Finally,
					if $k > n$, then we must have all $n$ characters with the addition of $k - n$ repeats which is a description of the Stirling numbers of the second kind when
					permutations are accounted for. Hence, the sum evaluates to $n! \begin{Bmatrix} k \\ n \end{Bmatrix}$ in this last case.
			\end{enumerate}
		\end{proof}


	% problem 6
	\item[\textbf{6.}] How many ways are there to seat $n$ married couples at a rotating round table with $2n$ chairs in such a way that no person sits next to their spouse?

		\begin{proof}[Solution]
			Suppose there are $n$ couples to be seated at a round table with $2n$ chairs. Let us consider the set of all seating arrangements. Let $A_i$ for $i \in [n]$ denote the set
			of all arrangements where the $i$th couple sit next to one another. We can then see that $|A_i| = \frac{2(2n - 1)!}{2n}$ for all $i \in [n]$ because there are two ways the
			couple can sit aside one another, $(2n-1)!$ for all to be seated, and finally we divide by $2n$ to remove all rotations. Furthermore, this scales accordingly,
			$$|A_{i_1} \cap \hdots \cap A_{i_k}| = \frac{2^{k-1}}{n} (2n - k)!$$ for all not equal $i_j \in [n]$. Thus, by the Principle of Inclusion-Exclusion we obtain
			$$\frac{1}{n} \sum_{k = 0}^n (-1)^k \binom{n}{k} 2^{k-1} (2n - k)!.$$
		\end{proof}

\end{enumerate}

\end{document}