\documentclass[ 12pt ]{article}
\usepackage{amsmath, amsthm, amssymb, enumitem, graphicx, mathrsfs}
\usepackage[margin=0.5in]{geometry}
\graphicspath{ ./ }

\begin{document}

\noindent Landon Fox \\
\noindent Math 485 \\
\noindent September 16, 2020 \\
\noindent \textbf{Collaborated with Logan Leavitt}

\begin{center}
\Large Homework 3
\end{center}

\begin{enumerate}
	% problem 1
	\item[\textbf{1.}]
		\begin{enumerate}
			\item[\textbf{i.}] Let $p$ be a prime number and $0 < k < p$. Show that $p | \binom{p}{k}$.
			\item[\textbf{ii.}] Use \textbf{i} to show Fermat's Little Theorem: $a^p \equiv a\; (\mathrm{mod}\; p)$, for any $a \in \mathbb{Z}$.
		\end{enumerate}

		\begin{proof}
			\item[\textbf{i.}] Suppose $p$ is a prime number and $0 < k < p$. By the Commitee-Chair Identity, we can see that $\binom{p}{k} = \frac{p}{k}\binom{p-1}{k-1}
				\in \mathbb{Z}$. Additionally, $k \nmid p$ since $p$ is prime (unless $k=1$, but this is an irrelevant case); hence, $k | \binom{p-1}{k-1}$. Thus, $p | \binom{p}{k}$.

			\item[\textbf{ii.}] Suppose $p$ is prime and $a \in \mathbb{N} \cup \{ 0 \}$. As a base case, let $a=0$. Trivially, the assertion for $a=0$ holds. For the
				inductive step, suppose the assertion holds for some $b = a-1 \geq 0$. Then by the Binomial Theorem, $$a^p = (1+b)^p = \sum_{k \in [n] \cup \{ 0 \}}
				\binom{p}{k} b^k.$$ Applying mod $p$ to the equality, we have $$a^p \equiv \sum_{k \in [n] \cup \{ 0 \}} \binom{p}{k} b^k \equiv 1 + b^p\; ( \mathrm{mod}\; p )$$ via
				\textbf{i}. Then by the inductive hypothesis, $$a^p \equiv 1 + b^n \equiv 1 + b \equiv a \; ( \mathrm{mod}\; p )$$ proving the claim for $a \in \mathbb{N} \cup \{ 0 \}$.
				Let us now turn to the case that $a \in \mathbb{Z}^-$. If $p = 2$, then we can see that the parity of $a$ and $a^2$ are equal. Hence, $a^2 \equiv a\; (
				\mathrm{mod}\; p )$. Otherwise, if $p$ is odd, then it follows that $a^p \equiv a\; ( \mathrm{mod}\; p )$, for positive $a$, implies $-a^p \equiv (-a)^p \equiv
				-a\; ( \mathrm{mod}\; p )$.
		\end{proof}


	% problem 2
	\item[\textbf{2.}]
		\begin{enumerate}
			\item[\textbf{i.}] Show using combinatorial arguments $$k^2 = 2\binom{k}{2} + \binom{k}{1}$$ and $$k^3 = 6\binom{k}{3} + 6\binom{k}{2} + \binom{k}{1}.$$
			\item[\textbf{ii.}] Use \textbf{i} to find the formula for $\sum_{k \in [n]} k^2$ and $\sum_{k \in [n]} k^3$.
		\end{enumerate}
		\newpage

		\begin{proof}
			\item[\textbf{i.}] Suppose $k \in \mathbb{N}$. Further, suppose we have a set $\{ (x, y) : x, y \in [k] \}$. Clearly, we can see that there are $k$ options for
				$x$ and $y$, providing $k^2$ options. Another way we can count this is by selecting all possible points on the diagonal of the grid summed with all possible points
				off the diagonal. There are $\binom{k}{1}$ ways to choose a point on the diagonal. Additionally, there are $2\binom{k}{2}$ ways to select two unique off diagonal
				values then permute them. Thus, $$k^2 = 2\binom{k}{2} + \binom{k}{1}.$$ \\
				Let us now consider a set $\{ (x, y, z) : x, y, z \in [k] \}$. Again, we can see that there are $k$ options for $x$, $y$, and $z$, providing $k^3$ options. Another
				way we can count this is by selecting 1, 2, and 3 distinct values in $[k]$, then permuting them. Furthermore, we can select one value in $\binom{k}{1}$ ways; two
				values in $2 \cdot \binom{3}{1, 2} \binom{k}{2} = 6\binom{k}{2}$ ways because we need choose which value will be repeated; three values in $3! \binom{k}{3} = 
				6\binom{k}{3}$. Thus, $$k^3 = 6\binom{k}{3} + 6\binom{k}{2} + \binom{k}{1}.$$
			\item[\textbf{ii.}] Using \textbf{i}, the Hockey Stick Identity, and the definition of a binomial coefficient, we can see that
				\begin{align*}
					\sum_{k \in [n]} k^2 &= \sum_{k \in [n]} \left ( 2\binom{k}{2} + \binom{k}{1} \right ) \\
					&= 2\binom{n+1}{3} + \binom{n+1}{2} \\
					\sum_{k \in [n]} k^2 &= \frac{n(n+1)(2n+1)}{6}.
				\end{align*}
				Additionally, it can be seen that
				\begin{align*}
					\sum_{k \in [n]} k^3 &= \sum_{k \in [n]} \left ( 6\binom{k}{3} + 6\binom{k}{2} + \binom{k}{1} \right ) \\
					&= 6\binom{n+1}{4} + 6\binom{n+1}{3} + \binom{n+1}{2} \\
					\sum_{k \in [n]} k^3 &= \left ( \frac{n(n+1)}{2} \right )^2.
				\end{align*}
		\end{proof}


	% problem 3
	\item[\textbf{3.}] Prove each identity below by counting a set in two ways.
		\begin{enumerate}
			\item[\textbf{i.}] $\binom{2n}{n} = 2\binom{2n-1}{n-1}$
			\item[\textbf{ii.}] $\sum_{k \in [n]} \binom{n}{k}\binom{k}{l} = \binom{n}{l} 2^{n-l}$
			\item[\textbf{iii.}] $\sum_{k \in [n]} q^{k-1} = \frac{q^n-1}{q-1}$ for $q, n \in \mathbb{N}$
			\item[\textbf{iv.}] $\sum_{k \in [n]} k(n-k) = \sum_{k \in [n]} \binom{k}{2}$
			\item[\textbf{v.}] $\sum_{k \in [n] \cup \{ 0 \}} \binom{n}{k} 2^k = 3^n$
		\end{enumerate}
		\newpage

		\begin{proof}
			\item[\textbf{i.}] Suppose we have the set of all possible lattice paths from $(0, 0)$ to $(n, n)$. To count this set, we can see that there are $2n$ steps required;
				furthermore, we know that exactly $n$ of them must be horizontal and the rest vertical. Then it follows that there $\binom{2n}{n}$ paths. Let us consider another
				form of counting, two cases dependent on our first step. If our first step is horizontal, then there are $2n-1$ steps remaining and $n-1$ must be horizontal.
				Likewise, if our first step is vertical, there are $2n-1$ steps remaining and $n-1$ must be vertical. Moreover, it follows that there are $\binom{2n-1}{n-1}
				+ \binom{2n-1}{n-1} = 2\binom{2n-1}{n-1}$ paths. Thus, $$\binom{2n}{n} = 2\binom{2n-1}{n-1}.$$

			\item[\textbf{ii.}] Suppose we have the set of all ternary strings of length $n$ with precisely $l$ 2's. To count this set, we can first choose which positions will be
				2's then let all other digits be either 0's or 1's, providing a count of $\binom{n}{l} 2^{n-l}$. Let us now count this set in another way; let us choose $n-k$
				digits to be 1's, then choose $l$ from the $k$ to be 2's, finally letting the remaining to be 0's. This counting process provides $\binom{n}{n-k}\binom{k}{l} =
				\binom{n}{k}\binom{k}{l}$. Since each value of $k$ will give a distinct number of 1's, we can sum the cases without double/undercounting; hence the count provides
				$\sum_{k \in [n]} \binom{n}{k}\binom{k}{l}$. Thus, $$\sum_{k \in [n]} \binom{n}{k}\binom{k}{l} = \binom{n}{l} 2^{n-l}.$$

			\item[\textbf{iii.}] Suppose we have the set of all base $q$ strings of length $n$ with at least one leading non-zero digit being 1. Observe that there are $q^n - 1$
				base $q$ strings excluding the string of all zeros. Then to ensure the leading digit is 1, we must divide by $q-1$; providing $\frac{q^n - 1}{q - 1}$. Let us consider
				another counting, suppose the $k$th digit is the leading 1, digits preceeding it must be 0's, and all digits upto but not including $k$ can be any possible digit.
				Then it follows that there are $q^{k-1}$ strings. Furthermore, for any possible $k$, there are $\sum_{k \in [n]} q^{k-1}$ strings in the set. Thus,
				$$\sum_{k \in [n]} q^{k-1} = \frac{q^n - 1}{q - 1}.$$

			\item[\textbf{iv.}] Suppose we have the set of all binary strings of length $n+1$ with exactly three 1's. By the proof of the Hockey Stick Identity we can see that
				$\sum_{k \in [n]} \binom{k}{2}$ counts our set precisely; however, for our purposes let us reiterate. Suppose we assign a 1 to be placed at the $(k+1)$th location.
				Then let us place the other two 1's in the first $k$ digits which provides $\binom{k}{2}$ ways. Summing over all $k$, we obtain $\sum_{k \in [n]} \binom{k}{2}$.
				Now, let us turn to another form of counting where we assign a 1 at location $k+1$ and insist that it is surrounded on both sides by the remaining 1's. Furthermore,
				for any $k$, there are $k$ and $n-k$ positions available before and after digit $k$, respectively. Hence, the count can be expressed as $\sum_{k \in [n]} k(n-k)$.
				Thus, $$\sum_{k \in [n]} k(n-k) = \sum_{k \in [n]} \binom{k}{2}.$$

			\item[\textbf{v.}] Suppose we have the set of all ternary strings of length $n$. Clearly, we can see that count of the set can be expressed as $3^n$. To count the
				set in another way, let us choose $k$ digits to be 2's, then let the remaining be either 0 or 1. Then for all $k$, we can express the count as
				$\sum_{k \in [n] \cup \{ 0 \}} \binom{n}{k} 2^k$. Thus, $$\sum_{k \in [n] \cup \{ 0 \}} \binom{n}{k} 2^k = 3^n.$$
		\end{proof}


	% problem 4
	\item[\textbf{4.}] Prove the following by induction and counting in two ways. $$\sum_{k \in [n]} k \cdot k! = (n+1)! - 1$$

		\begin{proof}[Induction Proof]
			As a base case, let $n = 1$, obviously we can see that the equality will hold. Let us turn to the inductive step. Suppose $\sum_{k \in [n]} k \cdot k! = (n+1)! - 1$ holds
			for an arbitrary $n \geq 1$. Now consider,
			\begin{align*}
				\sum_{k \in [n]} k \cdot k! &= (n+1)! - 1 \\
				\sum_{k \in [n]} k \cdot k! + (n+1)(n+1)! &= (n+1)! + (n+1)(n+1)! - 1 \\
				\sum_{k \in [n+1]} k \cdot k! &= (n+2)! - 1,
			\end{align*}
			concluding the inductive step.
		\end{proof}

		\begin{proof}[Combinatorial Proof]
			Suppose $n+1$ individuals have an assigned seating. Further, suppose we have the set of all possible seating arrangements such that at least one individual is not sitting
			in their own seat. Clearly we can see that $(n+1)!-1$ represents the count of our set since there is only one way all individuals can sit properly. Let us consider another
			counting by the following:
			\begin{enumerate}
				\item[\textbf{a.}] Let the first $n-k$ people take their proper seat.
				\item[\textbf{b.}] Then, let the next individual take a seat that is not their own.
				\item[\textbf{c.}] Finally, let the remaining $k$ take arbitrary seats.
			\end{enumerate}
			By these rules, we can see that all seating arrangements are accounted for when considering all possible $k$. Additionally, for each particular $k$, there are precisely
			$k \cdot k!$ arrangements. Thus, $$\sum_{k \in [n]} k \cdot k! = (n+1)! - 1.$$
		\end{proof}


	% problem 5
	\item[\textbf{5.}] By counting in two ways, prove that $$\sum_{A \subseteq [n]} \sum_{B \subseteq [n]} |A \cap B| = n4^{n-1}.$$

		\begin{proof}
			Suppose we have a fully connected hypergraph, $( \mathcal{P}[n], [n] )$, and we count the number of items in the intersection of all possible edges $A$ and $B$. Clearly,
			$\sum_{A \subseteq [n]} \sum_{B \subseteq [n]} |A \cap B|$ depicts our count. Let us now consider how many times a particular node is counted within an edges'
			intersection. With binary representations in mind, we can see that a specific node will occur in $2^{n-1}$ edges. Furthermore, for a particular node to occur in two (not
			necessarily distinct) edges, the count will be $2^{n-1} \cdot 2^{n-1} = 4^{n-1}$. Then accounting for all $n$ nodes, our final count is exactly $n4^{n-1}$. Thus,
			$$\sum_{A \subseteq [n]} \sum_{B \subseteq [n]} |A \cap B| = n4^{n-1}.$$
		\end{proof}

\end{enumerate}

\end{document}
