\documentclass[ 12pt ]{article}
\usepackage{amsmath, amsthm, amssymb, enumitem, mathrsfs}
\usepackage[margin=0.5in]{geometry}

\begin{document}

\noindent Landon Fox \\
\noindent Math 485 \\
\noindent August 31, 2020

\begin{center}
\Large Homework 1
\end{center}

\begin{enumerate}
	% problem 1
	\item[\textbf{1.}] Given an $8 \times 8$ checkerboard.
		\begin{enumerate}
			\item[\textbf{i.}] If we remove the upper left and upper right corner, can we tile it with dominos?
			\item[\textbf{ii.}] If we remove the upper left and lower right corner, can we tile it with dominos?
			\item[\textbf{iii.}] If we remove any one corner, can we tile it with straight triominos?
		\end{enumerate}

		\begin{proof}
			Suppose we have an $8 \times 8$ checkerboard.
			\begin{enumerate}
				\item[\textbf{i.}] True. Suppose we remove the upper left and upper right corner of the checkerboard. Place all dominos horizontally such that each row has four dominos,
					except for the first which has exactly three.
				\item[\textbf{ii.}] False. Suppose we remove the upper left and lower right corner of the checkerboard. Further, suppose by contradiction that we can tile the board
					with dominos. Let $x_n$ and $y_n$ denote the horizontal and vertical dominos in the $n$th row such that $y_8 = 0$. Then the following equations
					\begin{align*}
						2x_1 + y_1 &= 7 \\
						2x_2 + y_2 + y_1 &= 8 \\
						&\vdots \\
						2x_7 + y_7 + y_6 &= 8 \\
						2x_8 + y_7 &= 7
					\end{align*}
					represents our particular board. Applying mod 2 to each equation we have
					\begin{align*}
						y_1 &\equiv 1\; (\mathrm{mod}\; 2) \\
						y_2 + y_1 &\equiv 0\; (\mathrm{mod}\; 2) \\
						&\vdots \\
						y_7 + y_6 &\equiv 0\; (\mathrm{mod}\; 2) \\
						y_7 &\equiv 1\; (\mathrm{mod}\; 2).
					\end{align*}
					By examining the equations, we can see that $y_n$ is of odd parity for all $n \in [7]$. Similarly, $x_n$ must be even for all $n \in [8]$ since the total number of
					dominos is 31. Let us now rotate the board by $90^\circ$ such that horizontal and vertical dominos swap. By the same logic, we will see that $y_n$ is also even for
					all $n \in [7]$, which is a contradiction.
				\item[\textbf{iii.}] False. Suppose we remove an upper corner of the checkerboard. Further, suppose by contradiction that we can tile the board with straight triominos.
					Let us use similar conventions as \textbf{(ii)} by letting $x_n$ and $y_n$ denote the horizontal and vertical dominos in the $n$th row such that $y_7 = y_8 = 0$.
					Then the equations
					\begin{align*}
						3x_1 + y_1 &= 7 \\
						3x_2 + y_2 + y_1 &= 8 \\
						3x_3 + y_3 + y_2 + y_1 &= 8 \\
						&\vdots \\
						3x_7 + y_6 + y_5 &= 8 \\
						3x_8 + y_6 &= 8
					\end{align*}
					represents our particular board. Applying mod 3 to each equation we have
					\begin{align*}
						y_1 &\equiv 1\; (\mathrm{mod}\; 3) \\
						y_2 + y_1 &\equiv 2\; (\mathrm{mod}\; 3) \\
						y_3 + y_2 + y_1 &\equiv 2\; (\mathrm{mod}\; 3) \\
						&\vdots \\
						y_6 + y_5 &\equiv 2\; (\mathrm{mod}\; 3) \\
						y_6 &\equiv 2\; (\mathrm{mod}\; 3).
					\end{align*}
					By examining the first six equations we can conclude that $y_6 \equiv 0\; (\mathrm{mod}\; 3)$ which is a contradiction to equation eight.
			\end{enumerate}
		\end{proof}


	% problem 2
	\item[\textbf{2.}] Let $n$ be any positive odd integer. Write the numbers $1, 2, \hdots, 2n$ on paper. Pick any two numbers $i$ and $j$ and write down $|i - j|$ on the paper and
		erase $i$ and $j$. Now with this new set of numbers repeat the process until we have only one number left on the paper. Prove that this last number must be odd.

		\begin{proof}
			Suppose we initially have $2n$ natural numbers, $n$ numbers are even and $n$ are odd, where $n$ itself is odd, written on paper and we proceed with the directions depicted
			above. Observe that only the subtraction between numbers of different pairity will give an odd result. Hence, if there is an odd without another odd to complement it,
			the result must be odd, no matter what $i$s and $j$s where chosen. Furthermore, since there are initially an odd number of odd values, the result must be odd.
		\end{proof}


	% problem 3
	\item[\textbf{3.}]
		\begin{enumerate}
			\item[\textbf{i.}] Prove using induction that a set with $n$ elements has $\frac{n(n-1)}{2}$ subsets containing exactly two elements, whenever $n \geq 2$.
			\item[\textbf{ii.}] Prove using induction that a set with $n$ elements has $\frac{n(n-1)(n-2)}{6}$ subsets containing exactly three elements whenever $n \geq 3$.
		\end{enumerate}

		\begin{proof}
			\begin{enumerate}
				\item[\textbf{i.}] As a base case, let $n=1$. Clearly, no subsets of cardinality two can be created; thus, $0 = \frac{1(1-1)}{2}$ subsets exist. For the inductive step,
					assume we have a set $A$ consisting of $n$ elements with exactly $\frac{n(n-1)}{2}$ subsets of length two. Let us now append a unique element $k$ to create a set
					$A \cup \{ k \}$ of cardinality $n+1$. To calculate the new number of subsets with length two, we can see that all previous subsets still remain; additionally, all
					subsets with $k$ must also be included. Hence, $$|A \cup \{ k \}| = \frac{n(n-1)}{2} + \binom{n}{1} = \frac{n(n+1)}{2}.$$
				\item[\textbf{ii.}] As a base case, let $n=2$. Clearly, no subsets of cardinality three can be created; thus, $0 = \frac{2(2-1)(2-2)}{6}$ subsets exist. For the
					inductive step, assume we have a set $A$ consisting of $n$ elements with exactly $\frac{n(n-1)(n-2)}{6}$ subsets of length three. Let us now append a unique element
					$k$ to create a set $A \cup \{ k \}$ of cardinality $n+1$. To calculate the new number of subsets with length three, we can see that all previous subsets still
					remain; additionally, all subsets with $k$ must also be included. Hence, $$|A \cup \{ k \}| = \frac{n(n-1)(n-2)}{6} + \binom{n}{2} = \frac{n(n-1)(n+1)}{6}.$$
			\end{enumerate}
		\end{proof}


	% problem 4
	\item[\textbf{4.}] Suppose there is a group of $n$ people, and each person is aware of a scandal no one else knows about. When two people communicate by telephone, they both know
		about all the scandals each knows about at that given time. Let $G(n)$ be the minimum number of phone calls needed so that everyone knows about all the scandals.  Show that
		$G(n) \leq 2n-4$ for $n \geq 4$.

		\begin{proof}
			Suppose we have a community of $n \geq 4$ individuals where communication takes place as depicted above. Observe that it sufficies to construct an example where the
			community can spread all scandals in $2n-4$ calls. As a base case, let $n=4$. The base case is trivial, let the first and second individuals call as well as the third and
			fourth, afterwards let the first call the third and the second call the fourth. For the inductive step, suppose we have a community of $n$ individuals which can spread all
			rumors in $2n-4$ calls. Now, let us introduce a new member. Before the community commences in their $2n-4$ calls, let the new member call any other individual in the
			community to introduce their scandal. Afterwards, let the community perform their calls, then the new individual can call another random community member to \textit{catch
			up} on all unknown rumors. Hence, $2n-4+2 = 2(n+1)-4$ calls were required to spread all scandals.
		\end{proof}


	% problem 5
	\item[\textbf{5.}] Consider the numbers $\{ 1, 2, 3, \hdots, 2^k \}$. Show that it is possible to arrange them in a row in such a way that the average of any two never appears in
		between them.
		\begin{enumerate}
			\item[\textbf{i.}] Give a constructive proof using induction on k.
			\item[\textbf{ii.}] Show that the above proof can be extended for $\{ 1, 2, 3, \hdots, n \}$, when $n$ is not a power of 2.
		\end{enumerate}

		\textbf{Lemma}: If $f \in \mathbb{P}_1$ and $x_1,\, x_2 \in \mathbb{R}$, then avg$( f(x_1), f(x_2) ) = f(\mathrm{avg}(x_1, x_2))$.

		\begin{proof}[Lemma Proof]
			Suppose $f \in \mathbb{P}_1$ and $x_1,\, x_2 \in \mathbb{R}$. Observe that $$\mathrm{avg}(f(x_1), f(x_2)) = \frac{ax_1+b + ax_2+b}{2} = a\frac{x_1+x_2}{2} + b =
			f(\mathrm{avg}(x_1, x_2)).$$
		\end{proof}

		\begin{proof}
			\begin{enumerate}
				\item[\textbf{i.}] Suppose $A_k = [2^k]$. As a base case, let $k = 1$. Since there are only two elements the assertion is obvious. For the inductive step, assume
					there exists a configuration of $A_k$ where no average of two elements can be found between them. Now consider $A_{k+1}$. Let's begin by sorting the set by parity
					such that $$A_{k+1} = \{ 1, 3, \hdots, 2^{k+1} - 3, 2^{k+1} - 1, 2, 4, \hdots, 2^{k+1} - 2, 2^{k+1} \}.$$ In this configuration, if two elements of different parity
					were chosen, we can see that their result will not belong to the integers nor $A_{k+1}$. In regard to elements of the same parity, we can arrange both the even and
					odd elements as the inductive hypothesis arranges $A_k$; observe that even and odd numbers $2n,\, 2n-1 \in \mathbb{P}_1(n)$, thus by our lemma averages between
					values of the same parity will not be found between them if and only if they are not found between values $n$.
				\item[\textbf{ii.}] Observe that as a direct consequence of \textbf{(ii)}, any subset of $[2^k]$ can also be sufficiently arranged. Additionally, it is obvious that
					there always exists a $k \in \mathbb{N}$ for any $n \in \mathbb{N}$ such that $n \leq 2^k$. Thus, the set $[n] \subseteq [2^k]$ can be arranged such that an average
					of two elements can never be found between them.
			\end{enumerate}
		\end{proof}
		\newpage


	% problem 6
	\item[\textbf{6.}] Given a positive integer $n$, two players play a game. They take turns in choosing distinct divisors of $n$ according to the following rules:
		\begin{enumerate}
			\item[\textbf{i.}] No divisor of $n$ that is a multiple of a previously mentioned number can be chosen.
			\item[\textbf{ii.}] The player who is forced to choose 1 loses the game.
		\end{enumerate}
		Show that the first player always has a winning strategy.

		\begin{proof}
			Suppose $n \in \mathbb{N} \setminus \{ 1 \}$ and two individuals play the game depicted above. Further, suppose by contradiction that the individual who plays first does not
			have a winning strategy; in other words, for a particular $n$ the second player can always win preventing the first player from ever winning. Now, let the first player steal
			the second player's strategy illustrating that both will simultaneously win which is a contradiction.
		\end{proof}
\end{enumerate}

\end{document}