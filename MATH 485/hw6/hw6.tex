\documentclass[ 12pt ]{article}
\usepackage{amsmath, amsthm, amssymb, enumitem, graphicx, listings, mathrsfs}
\usepackage[margin=0.5in]{geometry}
\graphicspath{ ./ }

\begin{document}

\noindent Landon Fox \\
\noindent Math 485 \\
\noindent October 10, 2020 \\
\noindent \textbf{Collaborated with Logan Leavitt}

\begin{center}
	\Large Homework 6
\end{center}

\begin{enumerate}
	% problem 1
	\item[\textbf{1.}]
		\begin{enumerate}
			\item[\textbf{i.}] Prove that for any $n + 1$ integers $a_1, a_2, \hdots, a_{n+1}$ there exists some $a_i$ and $a_j$ such that $n|(a_i - a_j)$.
			\item[\textbf{ii.}] Prove that for any 52 integers $a_1, a_2, \hdots, a_{52}$ there exists some $a_i$ and $a_j$ such that $100|(a_i - a_j)$ or $100|(a_i + a_j)$.
		\end{enumerate}

		\begin{proof}
			\begin{enumerate}
				\item[\textbf{i.}] Suppose we have $n+1$ integers $a_1, a_2, \hdots, a_{n+1}$. Clearly there are $n$ equivalence classes belonging to modulo $n$. Then by the pigeonhole
					principle, there must exist two integers $a_i$ and $a_j$ such that they belong to the same equivalence class. Thus, their difference, $a_i - a_j$, must belong to
					class 0 illustrating that $n | (a_i - a_j)$.

				\item[\textbf{ii.}] Suppose we have 52 integers $a_1, a_2, \hdots, a_{52}$. Let us consider two cases regarding the equivalence classes of 100. \\
					Let there be two integers $a_i$ and $a_j$ such that they belong to the same equivalence class of 100. Then, we can see that their difference, $a_i - a_j$ must
					belong to class 0 which implies that $100 | (a_i - a_j)$. \\
					Now suppose that all 52 integers belong to distinct equivalence classes of 100. Observe that there are 49 pairs of distinct values that sum to 100 that exclude values
					such as 0 and 50 (i.e. 1, 99 and 4, 96). Provided that we have 52 integers, the pigeonhole principle illustrates that there must exist two values $a_i$ and $a_j$ such
					that they sum to 100. Thus, $100 | (a_i + a_j)$.
			\end{enumerate}
		\end{proof}


	% problem 2
	\item[\textbf{2.}]
		\begin{enumerate}
			\item[\textbf{i.}] Show that every sequence of $mn + 1$ distinct numbers has an increasing subsequence of length $m + 1$ or a decreasing subsequence of length $n + 1$.
				Show that $mn + 1$ is tight.
			\item[\textbf{ii.}] Show that in any set of at least $mn + 1$ points in $\mathbb{R}^2$, we can find a polygonal path of either $m$ positive-slope edges or $n$ negative-slope
				edges.
		\end{enumerate}

		\begin{proof}
			\begin{enumerate}
				\item[\textbf{i.}] Suppose we have a sequence $\{ x_s \}_{s \in [mn + 1]}$ of distinct values. Further, suppose we have $(i_s, d_s)$ where $i_s$ the largest increasing
					subsequence beginning with $x_s$ and $d_s$ is the largest decreasing subsequence ending with $x_s$. Now, by contradiction let $1 \leq i_s \leq m$ and $1 \leq d_s$ for
					all $s \in [mn + 1]$. Then by the pigeonhole principle there must exist some indices $j < k$ such that $(i_j, d_j) = (i_k, d_k)$ since there are $mn$ possibilities
					for $(i_s, d_s)$ and $mn + 1$ terms in the sequence. \\
					If $x_j < x_k$ then it follows that $x_k$ can be inserted into the largest increasing sequence beginning with $x_j$ contradicting the value of $i_j$. Furthermore,
					$x_j > x_k$ must hold; if this were the case, we can see that a similar contradiction occurs with $d_k$ where $x_j$ can be inserted into te decreasing sequence
					ending with $x_k$ which proves the claim. \\
					In regard to tightness, observe the ordered sequence $\{ 1, 4, 2, 3 \}$ where $m = n = 2$. No increasing or decreasing subsequence of length three can be constructed.

				\item[\textbf{ii.}] Suppose we have $mn + 1$ points in $\mathbb{R}^2$ and we enumerate them in increasing order from left to right. Observe that we can rule out cases of
					points having strictly horizontal or vertical slopes since they can be thought of as positive or negative to our convenience. Now applying \textbf{2i} we can
					construct either an increasing of decreasing subsequence of length $m+1$ or $n+1$, respectively, creating our desired polygonal path.
			\end{enumerate}
		\end{proof}


	% problem 3
	\item[\textbf{3.}] Let $x_1, x_2, \hdots, x_{19}$ be positive integers no greater than 93. Let $y_1, y_2, \hdots, y_{93}$ be positive integers no greater than 19. Prove that there
		exists a sum of certain $x_i$’s that is equal to a sum of certain $y_i$’s.

		\begin{proof}
			Suppose we have two sequences $x_1, x_2, \hdots, x_{19}$, positive integers no greater than 93 and $y_1, y_2, \hdots, y_{93}$, positive integers no greater than 19.
			Consider the partial sums $$X_m =  x_1 + x_2 + \hdots + x_m$$ and $$Y_n =  y_1 + y_2 + \hdots + y_n.$$ Without loss of generality, let $X_{19} \leq Y_{93}$. Consider
			$W_n = X_{m(n)} - Y_n$ such that $m(n)$ is the most minimal index such that $X_{m(n)}$ is minimal yet still greater than $Y_n$. Suppose $X_{m(n)} \geq Y_n + 19$ then
			it follows that $X_{m(n)-1} \geq Y_n$ contradicting the minimality of $m(n)$. If $W_n = 0$ for any $n$ then we are done. Otherwise, there is all values of $W_n$ must
			be between 1 and 18. Thus, by the pidgeonhole principle, there must exist two indices $n_1 > n_2$ such that $X_{m(n_1)} - Y_{n_1} = X_{m(n_2)} - Y_{n_2}$ which illustrates
			that $X_{m(n_1) - m(n_2)} = Y_{n_1 - n_2}$. \\
			\textbf{Note}: This proof was assisted by the following website. https://prase.cz/kalva/putnam/psoln/psol934.html
		\end{proof}


	% problem 4
	\item[\textbf{4.}] Show that for every irrational $\alpha$ there exists infinitely many rationals $\frac{p}{q}$, such that $\left | \alpha - \frac{p}{q} \right | < \frac{1}{q^2}$.

		\begin{proof}
			Suppose we have arbitrary values $\alpha \in \mathbb{R} \setminus \mathbb{Q}$ and $n \in \mathbb{N} \cup \{ 0 \}$. Further, suppose, we have a partition $P = \left \{
			\frac{k}{n} : k \in [n] \cup \{ 0 \} \right \}$ and a set $A = \{ \{ k \alpha \} : k \in [n] \cup \{ 0 \} \}$. Then it follows by the pigeonhole principle that at least
			two values of $A$ must fall into the same interval of $P$; moreover, there exists $k_1, k_2 \in [n] \cup \{ 0 \}$ such that $$|(k_1 - k_2) \alpha - \lfloor k_1 \alpha
			\rfloor - \lfloor k_2 \alpha \rfloor| = |\{ k_1 \alpha \} - \{ k_2 \alpha \}| < \frac{1}{n}.$$ Since $0 \leq k_1, k_2 \leq n$ it must hold that
			$$\left | \alpha - \frac{\lfloor k_1 \alpha \rfloor + \lfloor k_2 \alpha \rfloor}{k_1 - k_2} \right | < \frac{1}{n(k_1 - k_2)} < \frac{1}{(k_1 - k_2)^2}.$$ After the
			substitution $p = \lfloor k_1 \alpha \rfloor + \lfloor k_2 \alpha \rfloor$ and $q = k_1 - k_2$ it follows that there is an infinite family of $\frac{p}{q} \in \mathbb{Q}$
			values that satisfy $$\left | \alpha - \frac{p}{q} \right | < \frac{1}{q^2}.$$
		\end{proof}


	% problem 5
	\item[\textbf{5.}] Given any positive integer $m$ show that there exists a Hemachandra-Fibonacci number which is a multiple of $m$.

		\begin{proof}
			Suppose $m \in \mathbb{N}$. Consider the first $m^2 + 1$ Hemachandra-Fibonacci numbers viewed as pairs under modulo $m$, $(F_1, F_2), (F_3, F_4), \hdots, (F_{m^2}, F_{m^2+1})$.
			We can see that there are $m^2$ possible pairs, hence by the pigeonhole principle there must exist two pairs with the same values; moreover, there exists $i$ and $j$ such
			that $(F_i, F_{i+1}) = (F_j, F_{j+1})$. Furthermore, it must follow that our sequence is periodic since two terms can dictate the entirety of the sequence. Thus, any value
			to appear, must appear again later in the sequence. Observe that $F_0 = 0$ belongs to equivalence class 0. Therefore, a value must appear later in the sequence that is
			equivilant to $F_0$ proving the claim.
		\end{proof}


	% problem 6
	\item[\textbf{6.}] A car must circuit through towns $T_1, T_2, \hdots, T_n$ starting from a town and traveling in a cyclic order and back to where it started. At the start of the
		journey, the fuel tank is empty. It can fill amount $p_i$ at town $T_i$. If $\sum p_i$ is precisely the amount needed to make the tour, show that there is a town, such that it
		can complete the circuit without running out of fuel.

		\begin{proof}
			Suppose we have $n$ locations about a circuit such that $p_i$ is the amount of fuel given at location $i$ and $f_i$ is the fuel required to travel from town $i$ to $i+1$.
			Further, suppose $\sum p_i$ is percisely enough to travel about the circuit. Let $c_i = p_i - f_i$. Then it follows that $$\sum_{i \in [k, k)} a_i = 0$$ assuming that
			all indices have modulo $n$ applied to them. Suppose by contradicition that it is not possible to travel about the circuit from any particular location. Then we can
			see that for every $k$ there exists an $\ell < k$ (mod $n$) such that $$\sum_{i \in [k, l)} a_i < 0,$$ in other words, for every city there exists a path between two
			cities where the driver will run out of gas before the circuit is over. Now let us consider all incomplete paths $[k_0, k_1)$, $[k_1, k_2)$, $\hdots$. We can see that there
			are an infinite amount, otherwise it would imply that a driver can travel across the entire circut. Furthermore, by the pigeonhole principle there must exist two values
			$k_i$ and $k_j$ such that $i < j$ and $k_i = k_j$. Observe that a path taken from $k_i$ to $k_j$ makes $m$ laps around the circuit. Moreover,
			$$0 > \sum_{i \in [k_i, k_{i+1})} a_i + \sum_{i \in [k_{i+2}, k_{i+3})} a_i + \hdots + \sum_{i \in [k_{j-1}, k_j)} a_i = m \sum_{i \in [n]} a_i = 0$$ which is a contradiction.
		\end{proof}

\end{enumerate}

\end{document}