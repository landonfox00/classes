\documentclass[ 12pt ]{article}
\usepackage{amsmath, amsthm, amssymb, enumitem, graphicx, listings, mathrsfs}
\usepackage[margin=0.5in]{geometry}
\graphicspath{ ./ }

\begin{document}

\noindent Landon Fox \\
\noindent Math 485 \\
\noindent October 3, 2020

\begin{center}
	\Large Homework 5
\end{center}

\begin{enumerate}
	% problem 1
	\item[\textbf{1.}] Let $a_n = 3a_{n-1} - 2a_{n-2} + 1$ for $n \geq 2$, $a_0 = 2$, $a_1 = 4$. Obtain a simple formula for an using
		\begin{enumerate}
			\item[\textbf{i.}] Characteristic equation method.
			\item[\textbf{ii.}] Generating function method.
		\end{enumerate}

		\begin{proof}
			Suppose $a_n = 3a_{n-1} - 2a_{n-2} + 1$ for $n \geq 2$ where $a_0 = 2$ and $a_1 = 4$.
			\begin{enumerate}
				\item[\textbf{i.}] Consider the homogeneous recurrence, $h_n = 3h_{n-1} - 2h_{n-2}$. Suppose $h_n = x^n$ where $x \in \mathbb{R}$. Then it follows that
					\begin{align*}
						x^n &= 3x^{n-1} - 2x^{n-2} \\
						x^2 &= 3x - 2,
					\end{align*}
					which provides the solutions $x_1 = 1$ and $x_2 = 2$. Furthermore, $h_n = c_1 + c_2 2^n$. \\

					Let us now guess a particular solution $p_n = cn$ and so
					\begin{align*}
						p_n &= 3p_{n-1} - 2p_{n-2} + 1 \\
						cn &= 3c(n-1) - 2c(n-2) + 1 \\
						c &= -1.
					\end{align*}
					Therefore, $p_n = -n$ which illustrates that $a_n = h_n + p_n = c_1 + c_2 2^n - n$. Now considering the initial conditions we have the system
					\begin{align*}
						2 &= c_1 + c_2 \\
						4 &= c_1 + c_2 - 1.
					\end{align*}
					Solving the system gives $c_1 = -1$ and $c_2 = 3$. Thus, $a_n = 3 \cdot 2^n - n -1$.

				\item[\textbf{ii.}] Suppose $A(x) = \sum_{n \geq 0} a_n x^n$ where $x$ lies in an arbitrarily small interval about the origin. Let us now apply $\sum_{n \geq 2} x^n$
					to our recurrence,
					\begin{align*}
						\sum_{n \geq 2} a_n x^n &= 3 \sum_{n \geq 2} a_{n-1} x^n - 2 \sum_{n \geq 2} a_{n-2} x^n + \sum_{n \geq 2} x^n \\
						A(x) - 4x - 2 &= 3x(A(x) - 2) -2x^2A(x) + \frac{x^2}{1 - x} \\
						A(x) &= \frac{3x^2 - 4x + 2}{(1-x)^2(1-2x)} \\
						A(x) &= \frac{3}{1-2x} - \frac{1}{(1-x)^2}
					\end{align*}
					Then applying the coefficient operator, we have
					\begin{align*}
						[x^n]A(x) &= [x^n] \frac{3}{1-2x} - [x^n] \frac{1}{(1-x)^2} \\
						a_n &= 3 \cdot 2^n - n - 1.
					\end{align*}
			\end{enumerate}
		\end{proof}


	% problem 2
	\item[\textbf{2.}] Consider a set of $n$ circles in the plane such that every pair of circles meet exactly twice and no three meet at a common point. Obtain and solve recurrence
		for the number of regions into which the circles partition the plane.

		\begin{proof}
			Suppose we have $n-1$ circles in a plane as depicted above providing us $a_{n-1}$ regions. Additionally, suppose we add another circle of the same kind. Then it follows
			that $2(n-1)$ regions are added to the existing $a_{n-1}$. Thus, $a_n = a_{n-1} + 2(n-1)$. \\

			Consider the homogeneous recurrence $h_n = h_{n-1}$. Clearly $x = 1$ is the only non-zero solution to the characteristic equation. Hence, $h_n = c$. Let us now guess
			a particular solution $p_n = c_1 n^2 + c_2 n$ and so,
			\begin{align*}
				c_1 n^2 + c_2 n &= c_1 (n - 1)^2 + c_2 (n - 1) + 2(n - 1) \\
				2c_1n &= 2n - 2 + c_1 - c_2.
			\end{align*}
			Then it can be seen that $c_1 = 1$ and $c_2 = -1$; moreover, $p_n = n^2 - n$ which illustrates that $a_n = n^2 - n + c$. For an initial condition, we can see that $a_1 = 2$.
			Thus, $a_n = n^2 - n + 2$ for all $n \geq 1$.
		\end{proof}


	% problem 3
	\item[\textbf{3.}] For $n \geq 1$, let $a_n$ be the number of binary $(n - 1)$ tuples with no consecutive 1s. Give two proofs that $a_n$ equals the Fibonacci number $F_n$ as
		follows:
		\begin{enumerate}
			\item[\textbf{i.}] Show that $\{a_n\}$ satisfies the same recurrence as $\{F_n\}$.
			\item[\textbf{ii.}] Establish a bijection from the set of binary $(n - 1)$ tuples with no consecutive 1s to the set of $\{1, 2\}$ lists summing to $n$.
		\end{enumerate}

		\begin{proof}
			\begin{enumerate}
				\item[\textbf{i.}] Suppose $a_n$ is the number of binary $(n-1)$ tuples with no consecutive 1s for $n \geq 1$. Beginnning with the initial conditions, we can clearly
					see that $a_1 = 1$ and $a_2 = 2$. For the recurrence, suppose we have an $(n-2)$ with the above conditions and we are to append an element to it. If we were to
					append a 0 then no restrictions must be entertained, providing $a_{n-1}$ ways. If we were to append a 1, then we must also assert that the last digit is a 0 giving
					$a_{n-2}$ possibilities. Thus, $a_n = a_{n-1} + a_{n-2}$ with $a_1 = 1$ and $a_2 = 2$ illustrating that $a_n = F_n$ for all $n \geq 1$.

				\item[\textbf{ii.}] To define a bijection it should suffice to devise a reversible algorithm to convert elements between sets. Starting with the binary tuple, the
					algorithm is defined as follows:
					\begin{itemize}
						\item Append a 0 to the end of the tuple.
						\item Substitute all consecutive (0, 1)s for 2s.
						\item Substitute all remaining 0s with 1s.
					\end{itemize}
					We can see that this process is easily reversible. Furthermore, observe that every element in the tuple is worth one unit of value after the conversion. Thus, by
					appending the 0 to the end allows the result to sum to $n$.
			\end{enumerate}
		\end{proof}


	% problem 4
	\item[\textbf{4.}]
		\begin{enumerate}
			\item[\textbf{i.}] A child has some money to spend on candy. There are four types of candies with prices two cents, one cent, two cents and five cents per piece,
				respectively. Find the ordinary generating funtion for the number of ways to spend the money indexed by the total cost.
			\item[\textbf{ii.}] How many ways are there to distribute $n$ identical balls among 2 boys and 2 girls so that each boy gets at least 1 ball and each girl gets at least 2
				balls? Express your answer as the coefficient of a suitable term in a certain ordinary generating function.
		\end{enumerate}

		\begin{proof}
			\begin{enumerate}
				\item[\textbf{i.}] If we have four distinct items with the respective costs two, one, two, and five cents, respectively, then we can see that the generating function in
					regard to cost is
					\begin{align*}
						f(x) &= (1 + x + x^2 + \hdots) (1 + x^2 + x^4 + \hdots) (1 + x^2 + x^4 + \hdots) (1 + x^5 + x^{10} + \hdots) \\
						&= \frac{1}{(1 - x)(1 - x^2)^2(1 - x^5)}.
					\end{align*}

				\item[\textbf{ii.}] If we have $n$ indentical balls to be given to two boys and two girls where each boy and girl is guarenteed one and two balls, respectively,
					then it follows that
					\begin{align*}
						f(x) &= (x + x^2 + x^3 + \hdots) (x + x^2 + x^3 + \hdots) (x^2 + x^3 + x^4 + \hdots) (x^2 + x^3 + x^4 + \hdots) \\
						&= \left ( \frac{x}{1 - x} \right )^2 \left ( \frac{x^2}{1 - x} \right )^2 \\
						f(x) &= \frac{x^6}{(1 - x)^4}.
					\end{align*}
					Now extracting the $n$th coefficient, we have $$[x^n] f(x) = [x^n] \frac{x^6}{(1-x)^4} = [x^{n-6}] \frac{1}{(1-x)^4} = \binom{n - 3}{3}.$$
			\end{enumerate}
		\end{proof}


	% problem 5
	\item[\textbf{5.}] Choose $n$ points on a circle so that no three chords meet at a common point inside the circle. Let $a_n$ be the number of regions formed inside the circle by all
		the $\binom{n}{2}$ chords.
		\begin{enumerate}
			\item[\textbf{i.}] Obtain a recurrence $a_n = a_{n-1} + f(n)$, where $f(n) = (n - 1) + \sum_{i \in [n-1]} (i - 1)(n - 1 - i)$ with $a_0 = 1$.
			\item[\textbf{ii.}] Solve the recurrence in \textbf{i} to obtain a formula for $a_n$ as a sum of three binomial coefficients.
		\end{enumerate}

		\begin{proof}
			\begin{enumerate}
				\item[\textbf{i.}] Suppose we have $a_{n-1}$ regions defined as above for an arbitrary $n \geq 1$. If we were to add another point, we can see that new regions will
					appear by counting how many intersections a newly created line will make with previous lines in addition to $n-1$ since there is an additional region for each line.
					Observe that we can choose two points to specify an existing line then picking another to draw a new line to our $n$th point. Hence, $$a_n = a_{n-1} + n - 1 +
					\binom{n-1}{3}.$$

					Let us now turn to the binomial coefficient $\binom{n}{3}$. Suppose we have a set of $n$ items and we seek to count how many ways we can choose three of them.
					Clearly we can choose three via $\binom{n}{3}$. Consider another method by selecting an arbitrary element $i$ and choosing one element to its left and another to
					its right. Therefore, $$\binom{n}{3} = \sum_{i \in [n]} (i - 1)(n - i)$$. \\

					Thus, $$a_n = a_{n-1} + n - 1 + \sum_{i \in [n-1]} (i - 1)(n - 1 - i).$$

				\item[\textbf{ii.}] To solve the recurrence $a_n = a_{n-1} + n - 1 + \binom{n-1}{3}$, consider the homogeneous recurrence $h_n = h_{n-1}$. Clearly $x = 1$ is the
					non-zero solution to the characteristic equation which implies that $h_n = c$. \\

					Now, let us guess a particular solution, $p_n = c_1 n + c_2 n^2 + c_3 n^3 + c_4 n^4$. Then with the aid of Wolfram Alpha, substituting it into the recurrence
					provides $$p_n = \frac{n^4}{24} - \frac{n^3}{4} + \frac{23n^2}{24} - \frac{3n}{4}.$$

					Hence, $a_n = \frac{n^4}{24} - \frac{n^3}{4} + \frac{23n^2}{24} - \frac{3n}{4} + c$. Accounting for the initial condition, we can see that $c = 1$. Finally,
					by noticing similarities to binomial coefficients and some trial and error, notice that $$a_n = \binom{n}{4} + \binom{n}{2} + 1.$$
			\end{enumerate}
		\end{proof}

\end{enumerate}

\end{document}
