\documentclass[ 12pt ]{article}
\usepackage{amsmath, amsthm, amssymb, enumitem, mathrsfs}
\usepackage[margin=0.5in]{geometry}

\begin{document}

\noindent Landon Fox \\
\noindent Math 485 \\
\noindent September 7, 2020

\begin{center}
\Large Homework 2
\end{center}

\begin{enumerate}
	% problem 1
	\item[\textbf{1.}] Consider a $8 \times 8$ chessboard.
		\begin{enumerate}
			\item[\textbf{i.}] How many ways can you place 8 identical rooks so that no two attack each other?
			\item[\textbf{ii.}] How many ways can you do this if you have 4 identical wooden and 4 identical marble rooks?
			\item[\textbf{iii.}] How many ways can you do this if all 8 rooks are distinguishable?
		\end{enumerate}

		\begin{proof}[Solution]
			Suppose we have an ordinary $8 \times 8$ chessboard.
			\begin{enumerate}
				\item[\textbf{i.}] Observe that once a rook is placed in a particular tile, that row and column can no longer host another rook. Thus, the count is
					$$8 \cdot 8 \cdot 7 \cdot 7 \cdot \hdots \cdot 1 \cdot 1 = (8!)^2.$$
				\item[\textbf{ii.}] As a result of \textbf{i}, we can see that the count is similar; however, for each position we need to choose which of the rook
					positions will be wooden. Hence, the count is $\binom{8}{4}(8!)^2$.
				\item[\textbf{iii.}] Again, from \textbf{i}, the count will be similar, yet we need to account for all permutations of the 8 rooks. Thus, the count is
					$(8!)^3$.
			\end{enumerate}
		\end{proof}


	% problem 2
	\item[\textbf{2.}]
		\begin{enumerate}
			\item[\textbf{i.}] Count all rectangles of positive area formed by segments in a grid of horizontal and vertical lines.
			\item[\textbf{ii.}] Use induction to prove the answer obtained in \textbf{i.} when $m = n$.
		\end{enumerate}

		\begin{proof}
			Suppose we have a grid consisting of $m$ and $n$ horizontal and vertical lines respectively.
			\begin{enumerate}
				\item[\textbf{i.}] Observe that counting all rectangles of positive area is identical to counting all pairs of horizontal and vertical lines. Moreover,
					a rectangle can be uniquely represented by the lines that construct it. Then it follows that the number of all possible rectanges must be $\binom{m}{2}\binom{n}{2}$.

				\item[\textbf{ii.}] Let $m = n$. I claim that the count of all rectanges formed by the grid is precisely $\binom{n}{2}^2$. As a base case, let $n=2$, clearly one square
					is formed which is illustrated by $1 = \binom{2}{2}^2$. For the inductive step, suppose we have a grid of $n$ horizontal and $n$ vertical lines with
					$\binom{n}{2}^2$ rectangles. Let us now append both a horizontal and vertical line to the grid. In this new grid, we can see that all previous $\binom{n}{2}^2$
					still remain with the addition of all rectangles formed from the new lines. By inspecting the new grid we can see that $2\binom{n}{2}\binom{n}{1} + \binom{n}{1}^2$
					new rectangles are formed by ensuring one of the new lines are included or both of the new lines are included. Hence, the new count is
					$$\binom{n}{2}^2 + 2\binom{n}{2}\binom{n}{1} + \binom{n}{1}^2 = \left ( \binom{n}{2} + \binom{n}{1} \right )^2 = \binom{n+1}{2}^2.$$
			\end{enumerate}
		\end{proof}


	% problem 3
	\item[\textbf{3.}] Alice has $n$ different balls. First, she splits them into two piles, then she picks one of the piles with at least two elements, and splits it into two;
		she repeats this until each pile has only one element left. How many steps does this take? Find the number of ways she can carry out this procedure.

		\begin{proof}[Solution]
			Suppose Alice has $n$ distinct balls and follows the procedure depicted above. It is clear that the process will take $n-1$ if she were to continue until each pile
			consists of only one ball. Observe that it is sufficient to count the number of procedures where they are reversed. Moreover, suppose all $n$ distinct balls are already
			seperated into $n$ piles of one and we were to combine each respective pile until only one remains. Repeatedly, Alice must choose two piles to combine to one; thus, the
			count is $$\prod_{k \in [n] \setminus \{ 1 \}} \binom{k}{2} = \binom{n}{2}\binom{n-1}{2} \hdots \binom{3}{2}\binom{2}{2} = \frac{2^{1-n}(n!)^2}{n}.$$
		\end{proof}


	% problem 4
	\item[\textbf{4.}] In a shop there are $k$ different kinds of postcards. We want to send postcards to $n$ friends. In how many ways can you send the postcards if,
		\begin{enumerate}
			\item[\textbf{i.}] there is an unlimited supply for each kind of postcard, and you want to send one card to each friend.
			\item[\textbf{ii.}] there is an unlimited supply for each kind of postcard, and you want to send one card to each friend so that everyone receives different cards.
			\item[\textbf{iii.}] there is an unlimited supply for each kind of postcard, and you want to send two different cards to each friend.
			\item[\textbf{iv.}] there are $k$ distinct postcards and want to distribute all of them to our $n$ friends. How many ways can this be done? What happens if we want to send
				at least one card to each friend?
			\item[\textbf{v.}] there is a limited supply, there being $a_i$ copies of the $i^{th}$ kind, and you want to distribute all of them.
		\end{enumerate}

		\begin{proof}[Solution]
			Suppose we have $k$ distinct types of postcards and $n$ friends.
			\begin{enumerate}
				\item[\textbf{i.}] Each friend has $k$ possible postcards that can be sent to them. Thus, there are $k^n$ ways to send the postcards.
				\item[\textbf{ii.}] The first friend has $k$ options, the second has $k-1$, and so forth. Thus, there are $\binom{k}{n}n!$ ways to send the postcards.
				\item[\textbf{iii.}] Each friend has $\binom{k}{2}$ possible postcards that can be sent to them. Thus, there are $\binom{k}{2}^2$ ways to send the postcards.
				\item[\textbf{iv.}] For each postcard, there are $n$ friends they can be sent to. Hence, there are $n^k$ ways to send the postcards. If each friend must recieve at
					least one postcard, then let us subtract all accounts where individuals did not recieve cards. To remove such instances we must use the inclusion-exclusion
					principle since there are non-empty intersections between occurences of individuals not recieving postcards. To further illustrate, suppose we sought to remove
					instances where only one or two individuals did not recieve cards, we would obtain $$n^k - n(n-1)^k + \binom{n}{2}(n-2)^k = \binom{n}{0}n^k - \binom{n}{1}(n-1)^k +
					\binom{n}{2}(n-2)^k$$ due to double counting. Furthermore, to remove all instances of individuals not recieving postcards we can see that the count must be
					$$\sum_{m \in [n] \cup \{ 0 \}} (-1)^m \binom{n}{m}(n-m)^k = n!\begin{Bmatrix}k \\ n\end{Bmatrix}.$$ This problem can be rephrased as the count of all mappings
					and surjective mappings, respectively, from $[k]$ to $[n]$.
				\item[\textbf{v.}] Observe that for each $a_i$ we can apply the stars and bars method providing $\binom{a_i + n - 1}{a_i}$ ways to send the particular postcard type.
					Furthermore, we can see that there are $$\prod_{\forall i} \binom{a_i + n - 1}{a_i}$$ ways to send the postcards.
			\end{enumerate}
		\end{proof}


	% problem 5
	\item[\textbf{5.}] The chords of a convex $n$-gon are the segments that join two corners.
		\begin{enumerate}
			\item[\textbf{i.}] Count the pairs of chords that cross inside the $n$-gon.
			\item[\textbf{ii.}] Count all triangles formed by drawing all chords of the convex $n$-gon, given that no three chords have a common internal point.
		\end{enumerate}

		\begin{proof}[Solution]
			\begin{enumerate}
				\item[\textbf{i.}] Clearly we can see that there are $\binom{n}{2}$ chords and $\binom{n}{2} - n = \frac{n^2-3n}{2}$ chords strictly within the shape.
				\item[\textbf{ii.}] To count all triangles within the $n$-gon, let us consider a few disjoint cases determined by the number of interal points of a triangle within the
					shape. Suppose no corners of our triangle lie within our shape; in other words, they lie entirely on the boarder of our $n$-gon. Then we can see that there are
					$\binom{n}{3}$ triangles of this kind. If we have a triangle with one internal point, then we can see that our triangle does not uniquely share this point. Exactly
					four other triangles of this kind share this internal point. Additionally, two chords intersect at this point. Thus, there are $4\binom{n}{4}$ triangles with one
					internal point. If we have a triangle with two internal points, there must be a single vertex used from our $n$-gon. Furthermore, our triangle must use two other
					chords created by four other vertices from our $n$-gon, generating $5\binom{n}{5}$ triangles since any of the other four vertices can act as the corner of our
					triangle. Finally, if all three points of our triangle lie internally, then a unique triangle is formed from three chords giving $\binom{n}{6}$. Thus, there are
					$$\binom{n}{3} + 4\binom{n}{4} + 5\binom{n}{5} + \binom{n}{6}$$ triangles formed from the chords of a convex $n$-gon.
			\end{enumerate}
		\end{proof}
		\newpage


	% problem 6
	\item[\textbf{6.}]
		\begin{enumerate}
			\item[\textbf{i.}] Count the number of compositions of $n$ such that every part is even.
			\item[\textbf{ii.}] Show bijectively that the number of compositions of $n$ into an even number of parts is equal to the number of compositions of $n$ into an odd number of
				parts.
			\item[\textbf{iii.}] Show bijectively that the number of compositions of $n$ into an even number of even parts is equal to the number of compositions of $n$ into an odd
				number of even parts.
		\end{enumerate}

		\begin{proof}[Solution]
			\begin{enumerate}
				\item[\textbf{i.}] Suppose $n$ is an even natural number. Observe that the number of compositions of $n$ in even parts is equiviliant to the number of compositions
					of $\frac{n}{2}$. To count the number of compositions of $\frac{n}{2}$, we can represent every composition as a sequence of $\frac{n}{2}$ elements:
					$$\{ 1\, \square\, 1\, \square \hdots \square\, 1 \},$$ where each square can either be a '+' or ',' providing $2^{\frac{n}{2} - 1}$ possible compositions. \\
					This counting regarding a sequence consisting of '+' or ',' is not my own; rather, I recall seeing this from a coworker at Math Center.
			\end{enumerate}
		\end{proof}
\end{enumerate}

\end{document}