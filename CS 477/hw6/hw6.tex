\documentclass[ 12pt ]{article}

\usepackage{amsmath}
\usepackage{amssymb}
\usepackage{cancel}
\usepackage{tikz}
\usepackage{listings}
\usepackage{enumerate}
\usepackage[margin=0.5in]{geometry}
\footskip = -0.75in
\lstset{
    tabsize=2
}

\begin{document}

% title page
\title{%
	Homework 6 \\
	\large CS 477 Analysis of Algorithims \\
	Section 1001}
\author{Landon Fox}
\date{April 16, 2020}
\maketitle
\newpage

\begin{itemize}
	% problem 1
	\item[] {\large 1)}
	\begin{itemize}
		% problem 1a
		\item[] {\large 1a)}
		\[
			opt(i,\,j) =
			\begin{cases} 
				0 & i > n \\
				max[\, min( x_i,\, f(j) ) + opt( i+1,\, 1 ),\, opt(i+1,\, j+1)\, ] & elsewise
			\end{cases}
		\]
		$i$: The current second, increments from $1$ to $n$. \\
		$j$: The amount of seconds that have passed since the last fire of the laser. \\
		$opt(i,\,j)$: The optimal value of alien drones that can be made harmless. \\
		$n$: The last second of the alien drone swarm. \\
		$x_i$: The sequence of alien drone attacks per second. \\

		To find the optimal value of alien drones made harmless during the entire sequence of attacks, $opt(1,\,1)$ is computed.
		In the case that $i \leq n$, the function returns the maximum between $min( x_i,\, f(j) ) + opt( i+1,\, 1 )$ and $opt(i+1,\, j+1)$.
		The first represents the optimal amount of drones if the laser were to be fired at this instance $i$ then makes the problem smaller
		by adding the optimal amount of drones from the interval after $i$ where we reset the the charge of the laser. The second term represents
		the option of saving the charge of the laser for the current second by adding no additional terms, simply finding the optimal value of the
		next interval after $i$ where we have an additional second to our laser's charge. Finally we have the base case where $i>n$ and is beyond
		the defined sequence thus returns zero. \\
		\newpage

		% problem 1b
		\item[] {\large 1b)}
		\begin{lstlisting}[language=C]
int opt_b( int i, int j, int n, int x[ n ], int c[ n ][ n ], 
	int ( *f )( int ) )
{
	if ( i <= n && j <= n )
	{
		if ( !c[ i - 1 ][ j - 1 ] )
			c[ i - 1 ][ j - 1 ] = MAX( MIN( x[ i - 1 ], f( j ) ) +
				opt_b( i + 1, 1, n, x, c, f ), opt_b( i + 1, j + 1, n, x, c, f ) );
		return c[ i - 1 ][ j - 1 ];
	}
	return 0;
}

int opt_val_b( int n, int x[ n ], int count[ n ][ n ] )
{
	for ( int i = 0; i < n; ++i )
	{
		for ( int j = 0; j < n; ++j )
			count[ i ][ j ] = 0;
	}
	return opt_b( 1, 1, n, x, count, f );
}

int main()
{
	int count[ 4 ][ 4 ];
	int time[ 4 ][ 4 ];
	const int x[ 4 ] = { 1, 10, 10, 1 };

	printf( "1a) optimal value: %d\n", opt_val_b( 4, x, count ) );
	printf( "table of values for subproblems:\n" );
	print_2d_array( 4, 4, count );
}
		\end{lstlisting}
		\begin{center}
		$count[n][n]$:
			\begin{tabular}{ |cc|cccc| }
			\hline
			  &   &   & j &   &   \\
			  &   & 1 & 2 & 3 & 4 \\
			\hline
			  & 1 & 5 & 0 & 0 & 0 \\ 
			i & 2 & 3 & 5 & 0 & 0 \\
			  & 3 & 2 & 3 & 5 & 0 \\
			  & 4 & 1 & 1 & 1 & 1 \\
			\hline
			\end{tabular}
		\end{center}
		\newpage

		% problem 1c
		\item[] {\large 1c)}
		\begin{lstlisting}[language=C]
int opt_c( int i, int j, int n, int x[ n ], int c[ n ][ n ],
	int t[ n ][ n ], int ( *f )( int ) )
{
	if ( i <= n && j <= n )
	{
		if ( !c[ i - 1 ][ j - 1 ] )
		{
			int u = MIN( x[ i - 1 ], f( j ) ) +
				opt_c( i + 1, 1, n, x, c, t, f );
			int v = opt_c( i + 1, j + 1, n, x, c, t, f );
			c[ i - 1 ][ j - 1 ] = MAX( u, v );
			t[ i - 1 ][ j - 1 ] = u > v ? i : t[ i ][ j ];
		}
		return c[ i - 1 ][ j - 1 ];
	}
	return 0;
}

int opt_val_c( int n, int x[ n ], int count[ n ][ n ],
	int time[ n ][ n ] )
{
	for ( int i = 0; i < n; ++i )
	{
		for ( int j = 0; j < n; ++j )
		{
			count[ i ][ j ] = 0;
			time[ i ][ j ] = 0;
		}
	}
	return opt_c( 1, 1, n, x, count, time, f );
}

int main()
{
	int count[ 4 ][ 4 ];
	int time[ 4 ][ 4 ];
	const int x[ 4 ] = { 1, 10, 10, 1 };

	opt_val_c( 4, x, count, time );
	printf( "1b) table of times:\n" );
	print_2d_array( 4, 4, time );
}
		\end{lstlisting}
		\begin{center}
		$time[n][n]$:
			\begin{tabular}{ |cc|cccc| }
			\hline
			  &   &   & j &   &   \\
			  &   & 1 & 2 & 3 & 4 \\
			\hline
			  & 1 & 3 & 0 & 0 & 0 \\ 
			i & 2 & 3 & 3 & 0 & 0 \\
			  & 3 & 3 & 3 & 3 & 0 \\
			  & 4 & 4 & 4 & 4 & 4 \\
			\hline
			\end{tabular}
			\newline
		\end{center}

		% problem 1d
		\item[] {\large 1d)}
		First fill $t[n][n]$ with time values from $opt\_val\_c$. Run via $opt\_sol(1,1,4,time)$ for optimal times during sequence.
		\begin{lstlisting}[language=C]
void opt_sol( int i, int j, int n, int t[ n ][ n ] )
{
	if ( i <= n && t[ i - 1 ][ j - 1 ] )
	{
		printf( "%d, ", t[ i - 1 ][ j - 1 ] );
		opt_sol( t[ i - 1 ][ j - 1 ] + 1, 1, n, t );
	}
}

int main()
{
	int count[ 4 ][ 4 ];
	int time[ 4 ][ 4 ];
	const int x[ 4 ] = { 1, 10, 10, 1 };

	opt_val_c( 4, x, count, time );
	printf( "1c) optimal seconds to shoot lazer: " );
	opt_sol( 1, 1, 4, time );
	printf( "\n" );
}
		\end{lstlisting}
		Outputs seconds 3, 4. \\
	\end{itemize}

	% problem 3
	\item[] {\large 3)}
	\begin{itemize}
		% problem 3a
		\item[] {\large 3a)}
		This is true, all possible longest common subsequences of $X$ and $Y$ should contain $A$ if it is both their first character.
		We know that $A$ is common to both sequences so the longest common subsequence of both $X$ and $Y$ will be $A + lcs(X_{2..n}, Y_{2..n})$.
		If this were not true it would imply that $A$ is not in any longest common subsequence therefore choosing $A$ must prevent it from
		gaining the longest subsequences. A fallacy, $lcs(X_{2..n}, Y_{2..n})$ is completely independent of $A$. Therefore $A$ must be in
		all longest common subsequences. \\

		\item[] {\large 3b)}
		This is not necessarily true, all possible longest common subsequences of $X$ and $Y$ should contain $A$ if it is both
		their last character. If this were not true it would imply that $A$ is not in any longest common subsequence. We know
		that $A$ is common to both sequences so then choosing $A$ must prevent it from gaining the longest subsequences. A
		fallacy, there is no other characters being prevented since $A$ is the last character in both $X$ and $Y$. Therefore
		$A$ must be in all longest common subsequences.

	\end{itemize}
\end{itemize}

\end{document}