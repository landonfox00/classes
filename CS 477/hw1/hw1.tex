\documentclass[ 12pt ]{article}

\usepackage{amsmath}
\usepackage{amssymb}
\usepackage{cancel}

\begin{document}

% title page
\title{%
	Homework 1 \\
	\large CS 477 Analysis of Algorithims \\
	Section 1001}
\author{Landon Fox}
\date{February 4, 2020}
\maketitle
\newpage

% problem 1
\section{}
Assuming the argument of the log from (1) has power 10 so it can become a coefficient
\begin{flalign}
&f_{2}(n) = 5lg( n+100 )^{10} &&= O( lgn ) \\
&f_{5}(n) = ln^2n &&= O( lg^kn )&& \\
&f_{6}(n) = \sqrt[3]{n} &&= O( \sqrt[3]{n} ) \\
&f_{4}(n) = 0.001n^4+3n^3+1 &&= O( n^4 ) \\
&f_{7}(n) = 3^n &&= O( 3^n ) \\
&f_{3}(n) = 2^{2n} &&= O( 4^n ) \\
&f_{1}(n) = (n-2)! &&= O( n! )
\end{flalign}

% problem 2
\section{}

% problem 2a
\subsection{}
Prove:
\begin{flalign}
\sum_{i=1}^{n}(-1)^{i+1}i^2 &= \frac{(-1)^{n+1}n(n+1)}{2}\;\;\forall n\geq1 \\
let\;\;\; P(n):\: \sum_{i=1}^{n}(-1)^{i+1}i^2 &= \frac{(-1)^{n+1}n(n+1)}{2}
\end{flalign}
Base Case:
\begin{flalign}
P(1):\; \sum_{i=1}^{1}(-1)^{i+1}i^2 &= \frac{(-1)^{1+1}(1)(1+1)}{2} \\
(-1)^2(1)^2 &= \frac{(-1)^2(2)}{2} \\
1 &= 1
\end{flalign}
Inductive Step:
\begin{flalign}
assume\;\;\; P(n):\; \sum_{i=1}^{n}(-1)^{i+1}i^2 &= \frac{(-1)^{n+1}n(n+1)}{2} \\
P(n+1):\; \sum_{i=1}^{n+1}(-1)^{i+1}i^2 &= \frac{(-1)^{n+2}(n+1)(n+2)}{2} \\
\sum_{i=1}^{n}(-1)^{i+1}i^2 + (-1)^{n+2}(n+1)^2 &= \frac{(-1)^n(n+1)(n+2)}{2} \\
\frac{(-1)^{n+1}n(n+1)}{2} + (-1)^{n}(n+1)^2 &= \frac{(-1)^n(n+1)(n+2)}{2} \\
(n+1)^2 &= \frac{(n+1)(n+2)}{2} + \frac{n(n+1)}{2} \\
(n+1)^2 &= \frac{2n^2+4n+2}{2} \\
(n+1)^2 &= (n+1)^2
\end{flalign}\begin{center}
$$P(1): T$$
$$P(n) \rightarrow P(n+1): T$$
$$\therefore P(n)\; \forall n\geq1\; \square$$
\end{center}

% problem 2b
\subsection{}
Prove:
\begin{flalign}
\sum_{i=1}^{n-1}\frac{1}{i(i+1)} &= 1-\frac{1}{n}\;\;\forall n\geq1 \\
let\;\;\; P(n):\: \sum_{i=1}^{n-1}\frac{1}{i(i+1)} &= 1-\frac{1}{n}
\end{flalign}
Base Case:
\begin{flalign}
P(1):\: \sum_{i=1}^{0}\frac{1}{i(i+1)} &= 1-\frac{1}{1} \\
0 &= 0
\end{flalign}
Inductive Step:
\begin{flalign}
assume\;\;\; P(n):\; \sum_{i=1}^{n-1}\frac{1}{i(i+1)} &= 1 - \frac{1}{n} \\
P(n+1):\; \sum_{i=1}^{n}\frac{1}{i(i+1)} &= 1-\frac{1}{n+1} \\
\sum_{i=1}^{n-1}\frac{1}{i(i+1)} + \frac{1}{n(n+1)} &= 1 - \frac{1}{n+1} \\
1-\frac{1}{n} + \frac{1}{n(n+1)} &= 1 - \frac{1}{n+1} \\
\frac{1}{n(n+1)} &= \frac{1}{n} - \frac{1}{n+1} \\
\frac{1}{n(n+1)} &= \frac{1}{n(n+1)}
\end{flalign}\begin{center}
$$P(1): T$$
$$P(n) \rightarrow P(n+1): T$$
$$\therefore P(n)\: \forall n\geq1\; \square$$
\end{center}

% problem 3
\section{}
Note:
\begin{flalign}
\sum_{i=1}^{n}i &= \frac{n(n+1)}{2} \\
\sum_{i=1}^{n}i^2 &= \frac{n(n+1)(2n+1)}{6}
\end{flalign}

% problem 3a
\subsection{}
\begin{flalign}
f(n) &= \sum_{i=1}^{n}(2i-1) \\
&= 2\sum_{i=1}^{n}i-\sum_{i=1}^{n}1 \\
&= n(n+1)-n \\
&= n^2 \\
f(n) &= O(n^2)
\end{flalign}

% problem 3b
\subsection{}
\begin{flalign}
f(n) &= \sum_{i=1}^{n}i(i+1) \\
&= \sum_{i=1}^{n}i^2 + \sum_{i=1}^{n}i \\
&= \frac{n(n+1)(2n+1)}{6} + \frac{n(n+1)}{2} \\
&= \frac{n^3+3n^2+2n}{3} \\
f(n) &= O(n^3)
\end{flalign}

% problem 4
\section{}

% problem 4a
\subsection{}
\begin{flalign}
f(n) &= (n^3+3)^{20} \\
f(n) &= \Theta(n^{60})
\end{flalign}

% problem 4b
\subsection{}
\begin{flalign}
f(n) &= \sqrt[3]{5n^6+2} \\
f(n) &= \Theta(n^2)
\end{flalign}

% problem 4c
\subsection{}
\begin{flalign}
f(n) &= 4n^2lg\frac{(n+1)^2}{2}+n^3 \\
f(n) &= \Theta(n^3)
\end{flalign}

% problem 4b
\subsection{}
\begin{flalign}
f(n) &= \frac{8^{n+1}}{2^n} \\
&= 8\frac{2^{3n}}{2^n} \\
&= 8\cdot2^{2n} \\
&= 8\cdot4^n \\
f(n) &= \Theta(4^n)
\end{flalign}

% problem 5
\section{}
Know:
\begin{flalign}
f(n) &= O(g(n)) \\
f(n) &\leq cg(n)\;\; \exists c \forall n>n_0
\end{flalign}
%
% problem 5a
\subsection{}
Prove or Disprove:
\begin{flalign}
2^{f(n)} &= O(2^{g(n)})
\end{flalign}
Counter-example:
\begin{flalign}
let\;\;\; f(n)=&2n,\; g(n)=n \\
satisfies\;\;\; &2n = O(n) \\
&2^{2n} \leq c2^n \\
&4^n \leq c2^n \\
&2^n \leq c \\
\therefore\cancel{\exists}c\; &\forall n>n_0(2^n \leq c)
\end{flalign}
%
% problem 5b
\subsection{}
Prove or Disprove:
\begin{flalign}
f(n)^2 &= O(g(n)^2)
\end{flalign}
Proof: \\
Assuming both f(n) and g(n) are positive functions for all values of n, \\
the square function is one to one and always increasing for positive inputs therefore it can be applied to the inequality
\begin{flalign}
f(n) &\leq cg(n) \\
\therefore 0 \leq f(n)^2 &\leq c^2g(n)^2
\end{flalign}\begin{flalign}
0 \leq f(n)^2 &\leq dg(n)^2 \\
\therefore f(n)^2 &= O(g(n)^2)
\end{flalign}

% problem 6
\section{}

% problem 6a
\subsection{}
\begin{flalign}
f(n) &= n^2 \\
f(2n) &= 4n^2 \\
&= O(n^2) \\
f(n+1) &= n^2+2n+1 \\
&= O(n^2)
\end{flalign}

% problem 6b
\subsection{}
\begin{flalign}
f(n) &= n^3 \\
f(2n) &= 8n^3 \\
&= O(n^3) \\
f(n+1) &= n^3+3n^2+3n+1 \\
&= O(n^3)
\end{flalign}

% problem 6c
\subsection{}
\begin{flalign}
f(n) &= 100n^2 \\
f(2n) &= 400n^2 \\
&= O(n^2) \\
f(n+1) &= 100n^2+200n+100 \\
&= O(n^2)
\end{flalign}

% problem 6d
\subsection{}
\begin{flalign}
f(n) &= nlgn \\
f(2n) &= 2n(lgn+1) \\
&= O(nlgn) \\
f(n+1) &= (n+1)lg(n+1) \\
&= O(nlgn)
\end{flalign}

% problem 6e
\subsection{}
\begin{flalign}
f(n) &= 2^n \\
f(2n) &= 4^n \\
&= O(4^n) \\
f(n+1) &= 2^{n+1} \\
&= O(2^n)
\end{flalign}

\end{document}
