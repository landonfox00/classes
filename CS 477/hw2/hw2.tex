\documentclass[ 12pt ]{article}

\usepackage{amsmath}
\usepackage{amssymb}
\usepackage{cancel}
\usepackage{tikz}
\usepackage{listings}
\usepackage{mathdots}

\begin{document}

% title page
\title{%
	Homework 2 \\
	\large CS 477 Analysis of Algorithims \\
	Section 1001}
\author{Landon Fox}
\date{February 11, 2020}
\maketitle
\newpage

% problem 1
\section{}

% problem 1a
\subsection{}
\begin{flalign}
T(n) = 3T(\frac{n}{3})+n^2
\end{flalign}
Master's Method:
\begin{flalign}
a=3,\, b=3,\, f(n)&=n^2,\, g(n)=n^{log_33}=n \\
n &\leq cn^2 \\
g(n) &\leq cf(n)
\end{flalign}
Case 3: f(n) dominates g(n) \\
Show:
\begin{flalign}
af(\frac{n}{b}) &\leq cf(n) \\
3f(\frac{n}{3}) &\leq cf(n) \\
3(\frac{n}{3})^2 &\leq cn^2 \\
\frac{n^2}{3} &\leq cn^2 \\
c = \frac{1}{3}&,\, c \leq 1 \\
\therefore T(n) &= \Theta(n^2)
\end{flalign}

% problem 1b
\subsection{}
\begin{flalign}
T(n) = 8T(\frac{n}{2})+n^3
\end{flalign}
Master's Method:
\begin{flalign}
a=8,\, b=2,\, f(n)&=n^3,\, g(n)=n^{log_28}=n^3 \\
c_1n^3 \leq\, &n^3 \leq c_2n^3 \\
c_1g(n) \leq &f(n) \leq c_2g(n)
\end{flalign}
Case 2: f(n) equals g(n) \\
\begin{flalign}
\therefore T(n) &= \Theta(n^3lgn)
\end{flalign}

% problem 1c
\subsection{}
\begin{flalign}
T(n) = 4T(\frac{n}{2})+\sqrt{n}
\end{flalign}
Master's Method:
\begin{flalign}
a=4,\, b=2,\, f(n)&=\sqrt{n},\, g(n)=n^{log_24}=n^2 \\
\sqrt{n} &\leq cn^2 \\
f(n) &\leq cg(n)
\end{flalign}
Case 1: g(n) dominates f(n) \\
\begin{flalign}
\therefore T(n) &= \Theta(n^2)
\end{flalign}

% problem 2
\section{}

% problem 2a
\subsection{}
The algorithm recursively finds the minimum integer of a given array.

% problem 2b
\subsection{}
Accounting for all if-statements, assignments, and return statements as basic operations:
\begin{flalign}
T(n) &=\, \sim 4 + T(n-1) \\
T(1) &=\, \sim 2
\end{flalign}
or more generally:
\begin{flalign}
T(n) &= c + T(n-1) \\
&= c + c + T(n-2) \\
&= c + c + c + T(n-3) \\
&= \underset{n-1}{\underbrace{c + c + c + \cdots}} + T(1) \\
&= c(n-1) + T(1) \\
T(n) &= \Theta(n)
\end{flalign}

% problem 3
\section{}

% problem 3a
\subsection{}
\begin{lstlisting}[language=Python]
def exp2(n):
  if n == 1:
    return 2
  return exp2(n - 1) + exp2(n - 1)
\end{lstlisting}

% problem 3b
\subsection{}
\begin{flalign}
T(n) &= 2T(n-1) + 1
\end{flalign}
\begin{center}
\begin{tikzpicture}[level 1/.style={sibling distance=18em},
  level 2/.style={sibling distance=10em},
  level 3/.style={sibling distance=5em},
  level 4/.style={sibling distance=3em},
  every node/.style = {align=center,
    top color=white, bottom color=white!20}]]
  \node {T(n)=1}
	child { node {T(n-1)=1}
		child { node {T(n-2)=1}
			child { node {$\iddots$}
				child { node {T(1)} }
				child { node {$\cdots$} }
			}
			child { node {$\ddots$}
				child { node {$\cdots$} }
				child { node {$\cdots$} }
			}
		}
		child { node {T(n-2)=1}
			child { node {$\iddots$}
				child { node {$\cdots$} }
				child { node {$\cdots$} }
			}
			child { node {$\ddots$}
				child { node {$\cdots$} }
				child { node {$\cdots$} }
			}
		}
	}
	child { node {T(n-1)=1}
		child { node {T(n-2)=1}
			child { node {$\iddots$}
				child { node {$\cdots$} }
				child { node {$\cdots$} }
			}
			child { node {$\ddots$}
				child { node {$\cdots$} }
				child { node {$\cdots$} }
			}
		}
		child { node {T(n-2)=1}
			child { node {$\iddots$}
				child { node {$\cdots$} }
				child { node {$\cdots$} }
			}
			child { node {$\ddots$}
				child { node {$\cdots$} }
				child { node {$\cdots$} }
			}
		}
	};
\end{tikzpicture}
\end{center}
\begin{flalign}
level&(n):\, 2^n\, nodes \\
T(n) &= \sum_{i=0}^{n-2}(1)(2^i) + 2^{n-1}T(1) \\
&= \frac{1-2^{n-2}}{1-2} + 2^{n-1}T(1) \\
&= 2^{n-2} + 2^{n-1}T(1) - 1 \\
&= 2^n(\frac{1}{4} + \frac{1}{2}T(1)) - 1 \\
T(n) &= \Theta(2^n)
\end{flalign}

% problem 5
\section{Extra Credit}
\begin{flalign}
T(n) &= \sqrt n T(\sqrt n) + n \\
\frac{T(n)}{n} &= \frac{T(\sqrt n)}{\sqrt n} + 1 \\
let\;\;\; S(n) &= \frac{T(n)}{n} \\
S(n) &= S(\sqrt n) + 1 \\
let\;\;\; m = lgn &\rightarrow 2^m = n \\
S(2^m) &= S(2^{\frac{m}{2}}) + 1 \\
let\;\;\; Q(m) &= S(2^m) \\
Q(m) &= Q(\frac{m}{2}) + 1 \\
&= \underset{lgm}{\underbrace{1 + 1 + 1 + \cdots}} + Q(1) \\
&= lgm + c \\
S(n) &= lglgn + c \\
T(n) &= nlglgn + cn \\
&= O(nlglgn)
\end{flalign}

\end{document}
