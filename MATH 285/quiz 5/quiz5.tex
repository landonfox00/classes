\documentclass[ 12pt ]{article}
\usepackage{amsmath, amsthm, amssymb, enumitem, mathrsfs, mathtools}
\usepackage[margin=0.5in]{geometry}

\begin{document}

\noindent Landon Fox \\
\noindent Math 285 \\
\noindent August 7, 2020

\begin{center}
\Large Quiz 5
\end{center}

\begin{enumerate}
	\item[\textbf{1.}] Use the definition if the Laplace Transform to find $\mathscr{L}\{f(t)\}$ given
		$$ f(x) = \begin{cases} t; & 0 \leq t < 1 \\ 1; & t \geq 1 \end{cases}. $$

		\begin{proof}[Solution]\renewcommand{\qedsymbol}{}
			\begin{align*}
				\mathscr{L}\{f(t)\} &= \int_0^\infty e^{-st} f(t)\; \mathrm{d}t \\
				&= \int_0^1 te^{-st}\; \mathrm{d}t + \int_1^\infty e^{-st}\; \mathrm{d}t \\
				&= \left [ -\frac{te^{-st}}{s} - \frac{e^{-st}}{s^2} \right ]_0^1 + \left [ -\frac{e^{-st}}{s} \right ]_1^\infty \;\;\;\;\; \mathrm{Using\; integration\; by\; parts} \\
				&= -\frac{e^{-s}}{s} - \frac{e^{-s}}{s^2} + \frac{1}{s^2} + \frac{e^{-s}}{s} - \lim_{t \to \infty} \frac{e^{-st}}{s} \\
				\mathscr{L}\{f(t)\} &= \frac{1 - e^{-s}}{s^2}
			\end{align*}
		\end{proof}


	\item[\textbf{2.}] Find the inverse Laplace Transform $$\mathscr{L}^{-1} \left \{ \frac{6}{s^2 + 49} \right \}.$$

		\begin{proof}[Solution]\renewcommand{\qedsymbol}{}
			\begin{align*}
				\mathscr{L}^{-1} \left \{ \frac{6}{s^2 + 49} \right \} &= \frac{6}{7} \mathscr{L}^{-1} \left \{ \frac{7}{s^2 + 49} \right \} \\
				\mathscr{L}^{-1} \left \{ \frac{6}{s^2 + 49} \right \} &= \frac{6}{7} \sin 7t
			\end{align*}
		\end{proof}


	\item[\textbf{3.}] Find the Laplace Transform $$\mathscr{L}\{t^{11}e^{-7t}\}.$$

		\begin{proof}[Solution]\renewcommand{\qedsymbol}{}
			\begin{align*}
				\mathscr{L}\{t^{11}e^{-7t}\} &= \mathscr{L}\{t^{11}\}_{s \to s+7} \\
				&= \left [ \frac{11!}{s^{12}} \right ]_{s \to s+7} \\
				\mathscr{L}\{t^{11}e^{-7t}\} &= \frac{11!}{(s+7)^{12}}
			\end{align*}
		\end{proof}


	\item[\textbf{4.}] Find the Inverse Laplace Transform $$\mathscr{L}^{-1} \left \{ \frac{1}{(s-3)^3} \right \}.$$

		\begin{proof}[Solution]\renewcommand{\qedsymbol}{}
			\begin{align*}
				\mathscr{L}^{-1} \left \{ \frac{1}{(s-3)^3} \right \} &= e^{3t} \mathscr{L}^{-1} \left \{ \frac{1}{s^3} \right \} \\
				&= \frac{1}{2} e^{3t} \mathscr{L}^{-1} \left \{ \frac{2}{s^3} \right \} \\
				\mathscr{L}^{-1} \left \{ \frac{1}{(s-3)^3} \right \} &= \frac{1}{2} t^2 e^{3t}
			\end{align*}
		\end{proof}


	\item[\textbf{5.}] Use the Laplace Transform to solve the differential equation given the initial conditions $$y'' + 9y = \cos 3t;\;\; y(0) = 4,\; y'(0) = 2.$$

		\begin{proof}[Solution]\renewcommand{\qedsymbol}{}
			Let us first apply the Laplace Transform to the differential equation and attempt to solve for $\mathscr{L}\{y\} = Y$.
			\begin{align*}
				\mathscr{L}\{y'' + 9y\} &= \mathscr{L}\{\cos 3t\} \\
				s^2Y - sy(0) - y'(0) + 9Y &= \frac{s}{s^2 + 9} \\
				(s^2 + 9)Y &= \frac{s}{s^2 + 9} + 4s - 2 \\
				Y &= \frac{4s^3 + 2s^2 + 37s + 18}{(s^2 + 9)^2}
			\end{align*}
			Before applying the Inverse Laplace Transform, let us apply partial fraction to simplify $Y$.
			\begin{align*}
				\frac{4s^3 + 2s^2 + 37s + 18}{(s^2 + 9)^2} &= \frac{\alpha s + \beta}{s^2 + 9} + \frac{\gamma s + \delta}{(s^2 + 9)^2} \\
				4s^3 + 2s^2 + 37s + 18 &= (\alpha s + \beta)(s^2 + 9) + \gamma s + \delta \\
				4s^3 + 2s^2 + 37s + 18 &= \alpha s^3 + \beta s^2 + (9 \alpha + \gamma)s + (9\beta + \delta) \\
			\end{align*}
			The above equation illustrates that $\alpha = 4$, $\beta = 2$, $\gamma = 37 - 9\alpha = 1$, $\delta = 18 - 9\beta = 0$. Thus, $$Y = \frac{4s + 2}{s^2 + 9} + \frac{s}{(s^2 + 9)^2}.$$
			Now applying the Inverse Laplace Transform,
			\begin{align*}
				\mathscr{L}^{-1}\{Y\} &= \mathscr{L}^{-1} \left \{\frac{4s + 2}{s^2 + 9} + \frac{s}{(s^2 + 9)^2} \right \} \\
				y &= \mathscr{L}^{-1} \left \{\frac{4s + 2}{s^2 + 9} + \frac{s}{(s^2 + 9)^2} \right \} \\
				&= \mathscr{L}^{-1} \left \{ 4\frac{s}{s^2 + 9} + \frac{2}{3}\frac{3}{s^2 + 9} + \frac{1}{6}\frac{6s}{(s^2 + 9)^2} \right \} \\
				&= 4\cos 3t + \frac{2}{3}\sin 3t + \frac{1}{6}t\sin 3t \\
				y &= 4\cos 3t + \frac{1}{6}(t + 4)\sin 3t
			\end{align*}
		\end{proof}
\end{enumerate}

\end{document}