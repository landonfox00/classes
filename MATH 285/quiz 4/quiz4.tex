\documentclass[ 12pt ]{article}
\usepackage{amsmath, amsthm, amssymb, enumitem, mathtools}
\usepackage[margin=0.5in]{geometry}

\begin{document}

\noindent Landon Fox \\
\noindent Math 285 \\
\noindent July 31, 2020

\begin{center}
\Large Quiz 4
\end{center}

\begin{enumerate}
	\item[\textbf{1.}] Find the general solution with variation of parameters of $y'' + y = \sin x$.

		\begin{proof}[Solution]\renewcommand{\qedsymbol}{}
			Let us first consider the homogenous differential equation $y'' + y = 0$. As a guess to the solution, let $y = e^{mx}$. Then we can see that the auxiliary
			equation is $m^2 + 1 = 0$. The complex solution is $m = \pm i$. Then it follows that our solution is in the form $y_h = k_1e^{ix} + k_2e^{-ix} = c_1 \cos x + c_2 \sin x$
			where $c_1, c_2 \in \mathbb{C}$ using Euler's formula. Moreover, we can see that our independent solutions are $y_1 = \cos x$, $y_2 = \sin x$. \\

			Now that we obtained the homogenous solution $y_h$, let us now find the particular solution $y_p$ via variation of parameters. By substitution, let
			$y_p = u_1y_1 + u_2y_2$ such that $u_1' = \frac{W_1}{W}$, $u_2' = \frac{W_2}{W}$. Now we will solve for $u_1$ and $u_2$.
			\begin{align*}
				W &= \begin{vmatrix} \cos x & \sin x \\ -\sin x & \cos x \end{vmatrix} = \cos^2 x + \sin^2 x = 1 \\
				W_1 &= \begin{vmatrix} 0 & \sin x \\ \sin x & \cos x \end{vmatrix} = -\sin^2 x \\
				W_2 &= \begin{vmatrix} \cos x & 0 \\ -\sin x & \sin x \end{vmatrix} = \cos x \sin x
			\end{align*}
			Now we can see that $u_1' = -\sin^2 x = \frac{1}{2}\cos 2x - \frac{1}{2}$ and $u_2' = \cos x \sin x = \frac{1}{2}\sin 2x$.
			\begin{align*}
				u_1 &= \int \left ( \frac{1}{2}\cos 2x - \frac{1}{2} \right )\, \mathrm{d}x \\
				u_1 &= \frac{1}{4}\sin 2x - \frac{1}{2}x \\ \\
				u_2 &= \int \frac{1}{2}\sin 2x\, \mathrm{d}x \\
				u_2 &= -\frac{1}{4}\cos 2x
			\end{align*}
			Thus,
			\begin{align*}
				y_p &= \left ( \frac{1}{4}\sin 2x - \frac{1}{2}x \right ) \cos x - \frac{1}{4}\cos 2x \sin x \\
				&= \left ( \frac{1}{2}\cos x\sin x - \frac{1}{2}x \right ) \cos x - \frac{1}{4}( \cos^2 x - \sin^2 x )\sin x \\
				&= \frac{1}{2} \cos^2 x \sin x - \frac{1}{2}x \cos x - \frac{1}{4} \cos^2 x \sin x + \frac{1}{4} \sin^3 x \\
				&= \frac{1}{4} (1 - \sin^2 x) \sin x + \frac{1}{4} \sin^3 x - \frac{1}{2}x \cos x \\
				&= \frac{1}{4} \sin x - \frac{1}{4} \sin^3 x + \frac{1}{4} \sin^3 x - \frac{1}{2}x \cos x \\
				y_p &= \frac{1}{4} \sin x - \frac{1}{2}x \cos x.
			\end{align*}
			Given that the general solution $y = y_p + y_h$, we can see that
			\begin{align*}
				y &= \frac{1}{4} \sin x - \frac{1}{2}x \cos x + c_1 \cos x + c_2 \sin x \\
				y &= \left ( c_1 - \frac{1}{2}x \right ) \cos x + c_3 \sin x.
			\end{align*}
		\end{proof}


	\item[\textbf{2.}] Find the general solution of $xy'' - 4y' = 0$.

		\begin{proof}[Solution]\renewcommand{\qedsymbol}{}
			Observe that our differential equation is quite similiar to a particular Cauchy-Euler equation. Multiplying both sides of the equation by $x$ gives us the proper form
			$x^2y'' - 4xy' = 0$. As a guess to our solution, let $y = x^m$. Then it follows,
			\begin{align*}
				y &= x^m \\
				y' &= mx^{m-1} \\
				y'' &= m(m-1)x^{m-2}.
			\end{align*}
			For our auxiliary equation we can see that it becomes $m(m-1) - 4m = m^2 - 5m = m(m - 5) = 0$, illustrating that $m_1 = 0$, $m_2 = 5$. Thus, our fundamental solution
			set becomes $y_1 = x^0 = 1$, $y_2 = x^5$. Hence, the general solution is $y = c_1 + c_2x^5$.
		\end{proof}


	\item[\textbf{3.}] Write the given system without the use of matrices
		$$\frac{\mathrm{d}}{\mathrm{d}t} \begin{pmatrix} x \\ y \\ z \end{pmatrix} = \begin{bmatrix} 1 & -1 & 2 \\ 3 & -4 & 1 \\ -2 & 5 & 6 \end{bmatrix} \begin{pmatrix} x \\ y \\ z \end{pmatrix} + \begin{pmatrix} 1 \\ 2 \\ 2 \end{pmatrix}e^{-t} - \begin{pmatrix} 3 \\ -1 \\ 1 \end{pmatrix}t.$$

		\begin{proof}[Solution]\renewcommand{\qedsymbol}{}
			\begin{align*}
				\frac{\mathrm{d}}{\mathrm{d}t} \begin{pmatrix} x \\ y \\ z \end{pmatrix} &= \begin{bmatrix} 1 & -1 & 2 \\ 3 & -4 & 1 \\ -2 & 5 & 6 \end{bmatrix} \begin{pmatrix} x \\ y \\ z \end{pmatrix} + \begin{pmatrix} 1 \\ 2 \\ 2 \end{pmatrix}e^{-t} - \begin{pmatrix} 3 \\ -1 \\ 1 \end{pmatrix}t \\
				\begin{pmatrix} x' \\ y' \\ z' \end{pmatrix} &= \begin{pmatrix} x - y + 2z \\ 3x - 4y + z \\ -2x + 5y + 6z \end{pmatrix} + \begin{pmatrix} e^{-t} \\ 2e^{-t} \\ 2e^{-t} \end{pmatrix} + \begin{pmatrix} -3t \\ t \\ -t \end{pmatrix} \\
				\begin{pmatrix} x' \\ y' \\ z' \end{pmatrix} &= \begin{pmatrix} x - y + 2z + e^{-t} - 3t \\ 3x - 4y + z + 2e^{-t} + t \\ -2x + 5y + 6z + 2e^{-t} - t \end{pmatrix} \\ \\
				\frac{\mathrm{d}x}{\mathrm{d}t} &= x - y + 2z + e^{-t} - 3t \\
				\frac{\mathrm{d}y}{\mathrm{d}t} &= 3x - 4y + z + 2e^{-t} + t \\
				\frac{\mathrm{d}z}{\mathrm{d}t} &= -2x + 5y + 6z + 2e^{-t} - t
			\end{align*}
		\end{proof}
		\newpage


	\item[\textbf{4.}] Find the Eigenvalues and Eigenvectors of the given matrix $A = \left [ \begin{smallmatrix} 1 & 2 \\ -3 & 6 \end{smallmatrix} \right ]$.

		\begin{proof}[Solution]\renewcommand{\qedsymbol}{}
			Let us first examine the characteristic equation $\det (A - \lambda I_2) = 0$.
			$$\det (A - \lambda I_2) = \begin{vmatrix} 1 - \lambda & 2 \\ -3 & 6 - \lambda \end{vmatrix} = (1 - \lambda)(6 - \lambda) + 6 = (\lambda - 3)(\lambda - 4) = 0$$
			We can see that Eigenvalues are $\lambda_1 = 3$, $\lambda_2 = 4$. Now to find the Eigenvectors, we solve the equations $(A - \lambda_1I_2)\vec{u} = \vec{0}$
			and $(A - \lambda_2I_2)\vec{v} = \vec{0}$.
			\begin{align*}
				[A - 3I_2 | \vec{0}] &= \begin{bmatrix} -2 & 2 & 0 \\ -3 & 3 & 0 \end{bmatrix} \sim \begin{bmatrix} 1 & -1 & 0 \\ -1 & 1 & 0 \end{bmatrix} \sim \begin{bmatrix} 1 & -1 & 0 \\ 0 & 0 & 0 \end{bmatrix} \\
				\vec{u} &= \begin{pmatrix} u_1 \\ u_2 \end{pmatrix} = \begin{pmatrix} u_2 \\ u_2 \end{pmatrix} \\ \\
				[A - 4I_2 | \vec{0}] &= \begin{bmatrix} -3 & 2 & 0 \\ -3 & 2 & 0 \end{bmatrix} \sim \begin{bmatrix} -3 & 2 & 0 \\ 0 & 0 & 0 \end{bmatrix} \sim \begin{bmatrix} 1 & -\frac{2}{3} & 0 \\ 0 & 0 & 0 \end{bmatrix} \\
				\vec{v} &= \begin{pmatrix} v_1 \\ v_2 \end{pmatrix} = \begin{pmatrix} \frac{2}{3}v_2 \\ v_2 \end{pmatrix}
			\end{align*}
			Arbitrarily choosing a vector from each solution, we see that our Eigenvectors are $\begin{pmatrix} 1 \\ 1 \end{pmatrix}$ and $\begin{pmatrix} 2 \\ 3 \end{pmatrix}$.
		\end{proof}

\end{enumerate}

\end{document}