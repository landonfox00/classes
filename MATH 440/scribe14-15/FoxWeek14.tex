\documentclass[ 12pt ]{article}
\usepackage{amsmath, amsthm, amssymb, csquotes, enumitem, graphicx, listings, mathrsfs}
\usepackage[margin=0.5in]{geometry}
\graphicspath{ ./ }

\newcounter{lecture_num}
\theoremstyle{plain}

\theoremstyle{plain}
\newtheorem{theorem}{Theorem}[lecture_num]
\newtheorem{proposition}[theorem]{Proposition}
\newtheorem{lemma}[theorem]{Lemma}
\newtheorem{corollary}[theorem]{Corollary}

\theoremstyle{definition}
\newtheorem{definition}[theorem]{Definition}
\newtheorem{notation}[theorem]{Notation}
\newtheorem{observation}[theorem]{Observation}
\newtheorem{question}[theorem]{Question}

\theoremstyle{remark}
\newtheorem{remark}[theorem]{Remark}
\newtheorem{example}[theorem]{Example}

\begin{document}

\noindent Landon Fox \\
\noindent Math 440 \\
\noindent May 3, 2021

\begin{center}
	\Large Lecture Summary Week 14/15
\end{center}

\setcounter{lecture_num}{35}
\setcounter{theorem}{0}
\section*{Lecture 35}

\subsection*{Path Connectedness}

Path connectedness is a stronger variation of connectedness.

\begin{definition}
	A topological space $X$ is \textbf{path connected} if and only if for every $x, y \in X$ there exists a continuous function $\gamma : [0, 1] \to X$ such that $\gamma(0) = x$ and
	$\gamma(1) = y$.
\end{definition}

\begin{proposition}
	If $X$ is path connected, then $X$ is connected.
\end{proposition}

\begin{remark}
	Converse of Proposition 35.2 is false.
\end{remark}

\begin{example}
	Examples of path connected spaces.
	\begin{itemize}
		\item Open and closed balls in $\mathbb{R}^n$ with $n \geq 1$ under the Euclidean topology.
		\item Let $X = \mathbb{R}^n \setminus \{ \textbf{0} \}$. If $n \geq 2$, then $X$ is path connected.
	\end{itemize}
\end{example}

\begin{theorem}
	If $X$ is path connected and $f : X \to Y$ is a continuous function, then the subspace $f(X) \subseteq Y$ is also path connected.
\end{theorem}

\begin{corollary}
	If $X$ is path connected and $X \cong Y$, then $Y$ is path connected. Moreover, path connectedness is a topological property.
\end{corollary}

\begin{corollary}
	For $n \geq 2$, $\mathbb{R}$ is not homeomorphic to $\mathbb{R}^n$.
\end{corollary}


\setcounter{lecture_num}{36}
\setcounter{theorem}{0}
\section*{Lecture 36}

\subsection*{Path Connectedness}

The \textbf{topologist's sine curve} is an example of a connected space that is not path connected. Define $f : (0, \infty) \to \mathbb{R}$ as $f(x) = \sin \left ( \frac{1}{x} \right )$.
Additionally, let $S = \Gamma_f \subseteq \mathbb{R}^2$. The topologist's sine curve is the connected space $\overline{S}$. 

\begin{theorem}
	The topologist's sine curve is not path connected.
\end{theorem}

\subsection*{Quotient Topology}

\begin{definition}
	Let $(X, \mathcal{T})$ be a topological space and $\sim$ an equivalence relation on $X$. Let $X/\sim$ be the set of equivalence classes and let $\pi : X \to X/\sim$ be the surjection
	$\pi(x) = [x]$. The quotient topology $\pi_* \mathcal{T}$ on $X/\sim$ is defined by: $V \in \pi_* \mathcal{T}$ if and only if $\pi^{-1}(V) \in \mathcal{T}$.
\end{definition}


\setcounter{lecture_num}{37}
\setcounter{theorem}{0}
\section*{Lecture 37}

\subsection*{Quotient Topology}

\begin{remark}
	The quotient topology on any topological space is also a topological space.
\end{remark}

\begin{remark}
	The function $\pi : (X, \mathcal{T}) \to (X/\sim, \pi_* \mathcal{T})$ is continuous.
\end{remark}

\begin{proposition} $ $
	\begin{enumerate}
		\item The quotient topology is the \textit{largest} topology on $X/\sim$ such that $\pi : X \to X/\sim$ is continuous.
		\item A subset $A \subseteq X/\sim$ is closed with respect to the quotient topology if and only if $\pi^{-1}(A) \subseteq X$ is closed.
	\end{enumerate}
\end{proposition}

\begin{theorem}
	Let $f : X \to Y$ be a map between topological spaces. Let $\sim$ be an equivalence relation on $X$. If $f(x) = f(x')$ whenever $x \sim x'$, then the function $\overline{f} : X/\sim
	\to Y$ is defined as $\overline{f}([x]) = f(x)$ is continuous with respect to $\pi_*\mathcal{T}$ on $X/\sim$. Moreover, $\overline{f}$ is the unique map such that $\overline{f} \circ
	\pi = f$.
\end{theorem}

\begin{definition}
	Let $f : X \to Y$ be continuous. Further, let $\overline{f} : X/\sim_f \to Y$ be the induced map from Theorem 37.4. If $\overline{f}$ is a homeomorphism, then we say $f$ is a
	\textbf{quotient map}.
\end{definition}

\begin{example} $ $
	\begin{itemize}
		\item Every surjective open map is a quotient map.
		\item Every surjective closed map is a quotient map.
	\end{itemize}
\end{example}


\end{document}
