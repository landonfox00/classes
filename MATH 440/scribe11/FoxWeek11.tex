\documentclass[ 12pt ]{article}
\usepackage{amsmath, amsthm, amssymb, csquotes, enumitem, graphicx, listings, mathrsfs}
\usepackage[margin=0.5in]{geometry}
\graphicspath{ ./ }

\newcounter{lecture_num}
\theoremstyle{plain}

\theoremstyle{plain}
\newtheorem{theorem}{Theorem}[lecture_num]
\newtheorem{proposition}[theorem]{Proposition}
\newtheorem{lemma}[theorem]{Lemma}
\newtheorem{corollary}[theorem]{Corollary}

\theoremstyle{definition}
\newtheorem{definition}[theorem]{Definition}
\newtheorem{notation}[theorem]{Notation}
\newtheorem{observation}[theorem]{Observation}
\newtheorem{question}[theorem]{Question}

\theoremstyle{remark}
\newtheorem{remark}[theorem]{Remark}
\newtheorem{example}[theorem]{Example}

\begin{document}

\noindent Landon Fox \\
\noindent Math 440 \\
\noindent April 12, 2021

\begin{center}
	\Large Lecture Summary Week 11
\end{center}

\setcounter{lecture_num}{27}
\setcounter{theorem}{0}
\section*{Lecture 27}

\subsection*{Product Topology}

Consider a cylinder $$\{ (x, y, z) \in \mathbb{R}^3 : x^2 + y^2 = 1,\; 0 \leq z \leq 1 \} \subseteq \mathbb{R}^3$$ equipped with the subspace topology. We have an intrinsic notion of the
topology on the cylinder based on the topology of $S^1$ and $[0, 1]$ as well as an extrinsic view via the subspace topology. These notions provide a deeper notion to what is the product
topology.

\begin{proposition}
	Let $(X, \mathcal{T}_X)$ and $(Y, \mathcal{T}_Y)$ be topological spaces and define $$\mathcal{B}_{X \times Y} = \{ U \times V : U \in \mathcal{T}_X, V \in \mathcal{T}_Y \}.$$ Then it
	follows that $B_{X \times Y}$ is a basis for a topology on the Cartesian product $X \times Y$.
\end{proposition}

\begin{definition}
	The \textbf{product topology} $\mathcal{T}_{X \times Y}$ on $X \times Y$ is the topology generated by the basis $\mathcal{B}_{X \times Y}$.
\end{definition}

\begin{theorem}
	If $\mathcal{B}$ is a basis for a topology on $X$ and $\mathcal{C}$ is a basis for a topology on $Y$, then $$\mathcal{D} = \{ B \times C : B \in \mathcal{B}, C \in \mathcal{C} \}$$
	is a basis for the product topology on $X \times Y$.
\end{theorem}

\begin{notation}
	If $X$ and $Y$ are sets, then let $p_X : X \times Y \to X$ and $p_Y : X \times Y \to Y$ be surjective functions such that $p_X(x, y) = x$ and $p_Y(x, y) = y$. The functions $p_X$ and
	$p_y$ are referred to as \textbf{projections onto $X$ and $Y$}.
\end{notation}

\begin{theorem}
	Let $X$ and $Y$ be topological spaces. Equip $X \times Y$ with the product topology. Then the following holds.
	\begin{enumerate}
		\item $p_X$ and $p_Y$ are continuous.
		\item $p_X$ and $p_Y$ are open maps.
		\item The product topology is the \textit{smallest} topology on $X \times Y$ with the property that $p_X$ and $p_Y$ are continuous.
	\end{enumerate}
\end{theorem}


\setcounter{lecture_num}{28}
\setcounter{theorem}{0}
\section*{Lecture 28}

\subsection*{Product Topology}

\begin{proposition}
	The product topology $\mathcal{T}_{\mathbb{R} \times \mathbb{R}}$ on $\mathbb{R}^2$ is the Euclidean topology $\mathcal{T}_{\mathrm{Euc}, \mathbb{R}^2}$ on $\mathbb{R}^2$.
\end{proposition}

\begin{notation}
	Let $X$, $Y$, and $Z$ be topological spaces. Define maps $q_X : Z \to X$ and $q_Y : Z \to Y$.
	\begin{enumerate}
		\item The function $q : Z \to X \times Y$ defined as $q(z) = ( q_X(z), q_Y(z) )$ is continuous.
		\item If $\widetilde{q} : Z \to X \times Y$ is a continuous function such that $p_X \circ \widetilde{q} = q_X$ and $p_Y \circ \widetilde{q} = q_Y$, then $\widetilde{q} = q$.
	\end{enumerate}
\end{notation}


\setcounter{lecture_num}{29}
\setcounter{theorem}{0}
\section*{Lecture 29}

\subsection*{Product Topology}

\begin{corollary}
	Let $X$, $Y$, and $Z$ be topological spaces. A function $f : Z \to X \times Y$ is continuous if and only if the function $p_X \circ f : Z \to X$ and $p_Y \circ f : Z \to Y$ are
	continuous.
\end{corollary}

\subsection*{Hausdorffness and Products}

\begin{theorem}
	If $X$ and $Y$ are Hausdorff spaces then the product topology on $X \times Y$ is also Hausdorff.
\end{theorem}

\begin{theorem}
	Let $X$ be a topological space and define $$\Delta = \{ (x, x') \in X \times X : x = x' \}$$ as the \textbf{diagonal} of $X$. Then $X$ is Hausdorff if and only if $\Delta \subseteq X
	\times X$ is closed in the product topology.
\end{theorem}

\begin{proposition}
	Let $X$ and $Y$ be topological spaces where $Y$ is Hausdorff. Additionally, let $f, g : X \to Y$ be continuous functions. Then the \textbf{equalizer} of $f$ and $g$, $$\mathcal{E}(f,
	g) = \{ x \in X : f(x) = g(x) \}$$ is closed in $X$.
\end{proposition}

\begin{corollary}
	Let $f, g : X \to Y$ be continuous and $Y$ Hausdorff. Suppose $A \subseteq X$ is a dense subset. Then if $f(a) = g(a)$ for all $a \in A$, we have $f = g$.
\end{corollary}

\begin{remark}
	Let $p_A(t) : \mathrm{Mat}_n(\mathbb{C}) \to \mathbb{C}[t]$ denote the characteristic polynomial of $A$, $p_A(t) = \det(A - tI_n)$. Then it follows that $p_{AB}(t) = p_{BA}(t)$ for
	all $A, B \in \mathrm{Mat}_n(\mathbb{C})$.
\end{remark}


\end{document}
