\documentclass[ 12pt ]{article}
\usepackage{amsmath, amsthm, amssymb, csquotes, enumitem, graphicx, listings, mathrsfs}
\usepackage[margin=0.5in]{geometry}
\graphicspath{ ./ }

\newcounter{lecture_num}
\theoremstyle{plain}

\theoremstyle{plain}
\newtheorem{theorem}{Theorem}[lecture_num]
\newtheorem{proposition}[theorem]{Proposition}
\newtheorem{lemma}[theorem]{Lemma}
\newtheorem{corollary}[theorem]{Corollary}

\theoremstyle{definition}
\newtheorem{definition}[theorem]{Definition}
\newtheorem{notation}[theorem]{Notation}
\newtheorem{observation}[theorem]{Observation}
\newtheorem{question}[theorem]{Question}

\theoremstyle{remark}
\newtheorem{remark}[theorem]{Remark}
\newtheorem{example}[theorem]{Example}

\begin{document}

\noindent Landon Fox \\
\noindent Math 440 \\
\noindent March 29, 2021

\begin{center}
	\Large Lecture Summary Week 9
\end{center}

\setcounter{lecture_num}{22}
\setcounter{theorem}{0}
\section*{Lecture 22}

\subsection*{Properties Inherited from Subspaces}

\begin{proposition}
	If $A \subseteq X$ is a subspace and $X$ is metrizable, then $A$ is $T_i$ for $i = 0, 1, 2, 3, 4$.
\end{proposition}

\begin{lemma}
	If $X$ is metrizable and $A \subseteq X$ is a subspace, then $A$ is metrizable.
\end{lemma}

\begin{theorem} $ $
	\begin{enumerate}
		\item If $X$ is $T_2$, then every subspace $A \subseteq X$ is $T_2$.
		\item If $X$ is $T_1$ and $T_3$, then every subspace of $X$ is $T_1$ and $T_3$.
	\end{enumerate}
\end{theorem}

\setcounter{lecture_num}{23}
\setcounter{theorem}{0}
\section*{Lecture 23}

\subsection*{Compactness}

Compactness is a topological property that implies an analog of the Extreme Value Theorem from Analysis.

\begin{definition}
	An \textbf{open cover} of a topological space $X$ is a collection $\mathcal{U} = \{ U_\alpha \}_{\alpha \in \mathcal{A}}$ of open subsets of $X$ such that $X = \bigcup_{\alpha \in
	\mathcal{A}} U_\alpha$. If $U' = \{ U_\beta \}_{\beta \in \mathcal{A}'} \subseteq \mathcal{U}$ is a subcollection such that $X = \bigcup_{\beta \in \mathcal{A}'} U_\beta$, then
	$\mathcal{U}'$ is called a \textbf{subcover} of $\mathcal{U}$.
\end{definition}

\begin{example}
	Let $X = [0, 1] \subseteq \mathbb{R}$. Then $$\mathcal{U} = \left \{ \left [ 0, \frac{1}{10} \right ), \left ( \frac{1}{3}, 1 \right ] \right \} \bigcup \left \{ \left ( \frac{1}{n+2},
	\frac{1}{n} \right ) \right \}_{n \in \mathbb{N} \setminus \{ 1 \}}$$ is an open cover. Additionally, $$\mathcal{U} = \left \{ \left [ 0, \frac{1}{10} \right ), \left ( \frac{1}{3}, 1
	\right ] \right \} \bigcup \left \{ \left ( \frac{1}{n+2}, \frac{1}{n} \right ) \right \}_{2 \leq n \leq 9}$$ is a subcover.
\end{example}

\begin{definition}
	A topological space $X$ is \textbf{compact} if and only if every open cover of $X$ has a finite subcover.
\end{definition}

\begin{example}
	The Euclidean topology on $\mathbb{R}$ is not compact. It can be shown that $\mathcal{U} = \{ (-n, n) \}_{n \in \mathbb{N}}$ has no finite subcover.
\end{example}

\begin{remark}[Heine-Borel Theorem]
	Every closed interval $[a, b] \subseteq \mathbb{R}$ is compact. More generally, every closed and bounded subspace of $\mathbb{R}^n$ is compact.
\end{remark}

\begin{theorem}
	Let $f : X \to Y$ be a continuous function between topological spaces $X$ and $Y$. If $X$ is compact, then the subspace $f(X) \subseteq Y$ is compact.
\end{theorem}

\begin{remark}
	Compactness is a topological property.
\end{remark}

\begin{corollary}
	The Euclidean topology on $\mathbb{R}$ is not homeomorphic to $[a, b]$ for any $a, b \in \mathbb{R}$. Hence, $[a, b] \ncong (a, b)$.
\end{corollary}

\end{document}
