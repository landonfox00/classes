\documentclass[ 12pt ]{article}
\usepackage{amsmath, amsthm, amssymb, csquotes, enumitem, graphicx, listings, mathrsfs}
\usepackage[margin=0.5in]{geometry}
\graphicspath{ ./ }

\newcounter{lecture_num}
\theoremstyle{plain}

\theoremstyle{plain}
\newtheorem{theorem}{Theorem}[lecture_num]
\newtheorem{proposition}[theorem]{Proposition}
\newtheorem{lemma}[theorem]{Lemma}
\newtheorem{corollary}[theorem]{Corollary}

\theoremstyle{definition}
\newtheorem{definition}[theorem]{Definition}
\newtheorem{notation}[theorem]{Notation}
\newtheorem{observation}[theorem]{Observation}
\newtheorem{question}[theorem]{Question}

\theoremstyle{remark}
\newtheorem{remark}[theorem]{Remark}
\newtheorem{example}[theorem]{Example}

\begin{document}

\noindent Landon Fox \\
\noindent Math 440 \\
\noindent February 22, 2021

\begin{center}
	\Large Lecture Summary Week 4
\end{center}

\setcounter{lecture_num}{10}
\setcounter{theorem}{0}
\section*{Lecture 10}

\subsection*{Closure, Interior, and Boundary}

\begin{proposition}
	Let $X$ be a topological space and $Y \subseteq X$ a subspace. Then $B \subseteq Y$ is closed in $Y$ if and only if there exists a closed subset $A \subseteq X$ such that $B = Y \cap
	A$.
\end{proposition}

\begin{example}
	Let $A = [-1, 0) \subseteq \mathbb{R}$. Then $\overline{A} = A \cup L(A) = [-1, 0] \subseteq \mathbb{R}$. Let $Y = [-1, 0) \cup (0, 2] \subseteq \mathbb{R}$. Providing $Y$ the
	subspace topology, then $\overline{A} = Y \cap [-1, 0] = [-1, 0)$. Note that \textbf{Proposition 10.1} implies that if $A$ is a closed subset of $Y$, then $\overline{A} = A$ as a
	subset of $Y$.
\end{example}

\begin{definition}
	Let $X$ be a topological space and $A \subseteq X$ a subset. The \textbf{interior of $A$}, denoted as $\mathring{A}$, is the union of all open subsets contained in $A$; that is,
	$$\mathring{A} = \bigcup_{\substack{U \subseteq X\; \mathrm{open}, \\ U \subseteq A}} U.$$ The \textbf{boundary of $A$} is the subset $\partial A = \overline{A} \setminus
	\mathring{A}$.
\end{definition}

\begin{remark}
	Basic facts about $\mathring{A}$ and $\partial A$ in a topological space $X$.
	\begin{enumerate}
		\item $\mathring{A}$ is open in $X$ and is the largest open subset of $X$ contained in $A$.
		\item $\partial A \subseteq X$ is a closed subset of $X$. In fact, $\partial A = \overline{A} \cap (X \setminus \mathring{A})$.
	\end{enumerate}
\end{remark}

\begin{example} $ $
	\begin{itemize}
		\item If $X = \mathbb{R}$ and $A = (0, 1)$, then $\overline{A} = [0, 1]$, $\mathring{A} = A$, and $\partial A = \overline{A} \setminus \mathring{A} = \{0, 1\}$.
		\item If $X = \mathbb{R}^2$ and $A = \overline{B}_\epsilon(\textbf{x})$, then $\overline{A} = A$, $$\mathring{A} = \{ \textbf{y} \in \mathbb{R}^2 : d_2(\textbf{x}, \textbf{y}) <
			\epsilon \},\;\;\; \mathrm{and}\;\;\; \partial A = \{ \textbf{y} \in \mathbb{R}^2 : d_2(\textbf{x}, \textbf{y}) = \epsilon \}.$$
	\end{itemize}
\end{example}

\begin{proposition}
	Let $X$ be a topological space. Further, let $A \subseteq X$ and $x \in X$. Then $x \in \mathring{A}$ if and only if there exists an open neighborhood $V \subseteq X$ of $x$ such
	that $V \subseteq A$.
\end{proposition}

\begin{proposition}
	Let $X$ be a topological space. Additionally, let $A \subseteq X$ and $x \in X$. Then $x \in \partial A$ if and only if for every open neighborhood $U \subseteq X$ of $x$ it holds
	that $U \cap A \neq \varnothing$ and $U \cap (X \setminus A) \neq \varnothing$.
\end{proposition}


\setcounter{lecture_num}{11}
\setcounter{theorem}{0}
\section*{Lecture 11}

\subsection*{Density and Separability}

Recall from Math 310 the density of the rationals in the real numbers; for all $r \in \mathbb{R}$ and for all $\epsilon > 0$, there exists a $q \in \mathbb{Q}$ such that $r - \epsilon <
q < r + \epsilon$. An important consequence is that every continuous function $f : \mathbb{R} \to \mathbb{R}$ is uniquely determined by its values on the rationals.

\begin{definition}
	Let $X$ be a topological space. A subset $A \subseteq X$ is \textbf{dense} in $X$ if and only if $\overline{A} = X$. We say $X$ is \textbf{separable} if and only if $X$ contains a
	countable dense subset; moreover, there exists a subset $A \subseteq X$ such that $|A| = \aleph_0$ and $\overline{A} = X$.
\end{definition}

\begin{example}
	Common examples of density and separability.
	\begin{itemize}
		\item $\mathbb{Q} \subset \mathbb{R}$ is dense and countable.
		\item $\mathbb{Q}^n \subset \mathbb{R}^n$ is dense and countable and so $(\mathbb{R}^n, \mathcal{T}_{\mathrm{euclid}})$ is separable.
		\item If $X = (\mathbb{R}, \mathcal{T}_{\mathrm{cof}})$ and $U \subseteq \mathbb{R}$ then $U$ is dense in $\mathbb{R}$.
	\end{itemize}
\end{example}

\subsection*{Basis of a Topological Space}

Recall how we characterized open subsets of $\mathbb{R}^n$; a subset $U \subseteq \mathbb{R}^n$ is open if and only if $U$ can be expressed as a union of open balls.

\begin{definition}
	If $X$ is a set, a \textbf{basis} for a topology on $X$ is a collection $\mathcal{B}$, called basis elements, of subsets of $X$ such that
	\begin{enumerate}
		\item for each $x \in X$, there exists at least one basis element $B \in \mathcal{B}$ such that $x \in B$.
		\item if $B_1, B_2 \in \mathcal{B}$ and $x \in B_1 \cap B_2$, then there must exists a $B_3 \in \mathcal{B}$ such that $x \in B_3 \subseteq B_1 \cap B_2$.
	\end{enumerate}
\end{definition}

\begin{proposition}
	If $X$ is a topological space, then $A \subseteq X$ is dense if and only if $A \cap U \neq \varnothing$ where $U$ is any nonempty open subset of $X$.
\end{proposition}

\end{document}
