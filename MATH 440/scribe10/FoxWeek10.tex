\documentclass[ 12pt ]{article}
\usepackage{amsmath, amsthm, amssymb, csquotes, enumitem, graphicx, listings, mathrsfs}
\usepackage[margin=0.5in]{geometry}
\graphicspath{ ./ }

\newcounter{lecture_num}
\theoremstyle{plain}

\theoremstyle{plain}
\newtheorem{theorem}{Theorem}[lecture_num]
\newtheorem{proposition}[theorem]{Proposition}
\newtheorem{lemma}[theorem]{Lemma}
\newtheorem{corollary}[theorem]{Corollary}

\theoremstyle{definition}
\newtheorem{definition}[theorem]{Definition}
\newtheorem{notation}[theorem]{Notation}
\newtheorem{observation}[theorem]{Observation}
\newtheorem{question}[theorem]{Question}

\theoremstyle{remark}
\newtheorem{remark}[theorem]{Remark}
\newtheorem{example}[theorem]{Example}

\begin{document}

\noindent Landon Fox \\
\noindent Math 440 \\
\noindent April 5, 2021

\begin{center}
	\Large Lecture Summary Week 10
\end{center}

\setcounter{lecture_num}{24}
\setcounter{theorem}{0}
\section*{Lecture 24}

\subsection*{Heine-Borel Theorem and Compactness}

\begin{definition}
	Let $(A, \leq)$ be a totally ordered set.
	\begin{enumerate}
		\item A nonempty subset $S \subseteq A$ is \textbf{bonded above} if and only if there exists an $a \in A$ such that $x \leq a$ for all $x \in S$. We say $a$ is an \textbf{upper
			bound}. The value $\sup S \in A$ denotes the least upper bound of $S$.
		\item The set $A$ has the \textbf{least upper bound property} if and only if every nonempty subset $S \subseteq A$ that is bounded above has a least upper bound.
	\end{enumerate}
\end{definition}

\begin{example}
	It can be shown that $\mathbb{R}$ has the least upper bound property.
\end{example}

\begin{theorem}[Heine-Borel]
	Let $a < b \in \mathbb{R}$. Then $[a, b] \subseteq \mathbb{R}$ is compact.
\end{theorem}

\begin{theorem}
	If $X$ is a compact topological space and $A \subseteq X$ is a closed subspace, then $A$ is compact.
\end{theorem}

\begin{theorem}
	If $X$ is Hausdorff and $A \subseteq X$ is a compact subspace, then $A$ is a closed subset of $X$.
\end{theorem}

\setcounter{lecture_num}{25}
\setcounter{theorem}{0}
\section*{Lecture 25}

\subsection*{Compactness Properties}

\begin{corollary}
	Let $X$ be compact and $Y$ Hausdorff.
	\begin{enumerate}
		\item Every map $f : X \to Y$ is a closed map.
		\item If $f : X \to Y$ is a continuous bijection, then $f$ is a homeomorphism.
	\end{enumerate}
\end{corollary}

\begin{definition}
	Suppose $f : X \to Y$ is continuous and injective. Give $f(X) \subseteq Y$ the subspace topology and let $\widetilde{f} : X \to f(X)$ denote the induced map onto the image. We say
	$f$ is a \textbf{topological embedding} if and only if $\widetilde{f}$ is a homeomorphism.
\end{definition}

\begin{example}
	The field of Knot Theory often studies topological embeddings in the form $S^1 \to \mathbb{R}^3$.
\end{example}

\setcounter{lecture_num}{26}
\setcounter{theorem}{0}
\section*{Lecture 26}

\subsection*{Compactness Properties}

\begin{definition}
	A collection $\mathcal{C}$ of subsets of a space $X$ has the \textbf{finite intersection property} if and only if for every finite subcollection $\{ C_1, \hdots, C_n \} \subseteq
	\mathcal{C}$, it holds that the intersection $$\bigcap_{i = 1}^n C_i \neq \varnothing.$$
\end{definition}

\begin{theorem}
	A topological space is compact if and only if for every collection $\mathcal{C}$ of closed subsets in $X$ having the finite intersection property, it holds that $$\bigcap_{C \in
	\mathcal{C}} C \neq \varnothing.$$
\end{theorem}

\begin{corollary}
	If $X$ is compact and $A_1 \supseteq A_2 \supseteq \hdots$ is a nested sequence of nonempty closed subsets of $X$, then $$\bigcap_{i \in \mathbb{N}} A_i \neq \varnothing.$$
\end{corollary}

\begin{definition}
	Let $(X, d)$ be a metric space. A subset $A \subseteq X$ is \textbf{bounded} if and only if there exists an $M \in \mathbb{N}$ such that $d(a_1, a_2) \leq M$ for all $a_1, a_2 \in A$.
\end{definition}

\begin{theorem}
	A subset $K \subseteq \mathbb{R}$ is compact if and only if $K$ is closed and bounded.
\end{theorem}

\end{document}
