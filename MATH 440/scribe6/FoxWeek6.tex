\documentclass[ 12pt ]{article}
\usepackage{amsmath, amsthm, amssymb, csquotes, enumitem, graphicx, listings, mathrsfs}
\usepackage[margin=0.5in]{geometry}
\graphicspath{ ./ }

\newcounter{lecture_num}
\theoremstyle{plain}

\theoremstyle{plain}
\newtheorem{theorem}{Theorem}[lecture_num]
\newtheorem{proposition}[theorem]{Proposition}
\newtheorem{lemma}[theorem]{Lemma}
\newtheorem{corollary}[theorem]{Corollary}

\theoremstyle{definition}
\newtheorem{definition}[theorem]{Definition}
\newtheorem{notation}[theorem]{Notation}
\newtheorem{observation}[theorem]{Observation}
\newtheorem{question}[theorem]{Question}

\theoremstyle{remark}
\newtheorem{remark}[theorem]{Remark}
\newtheorem{example}[theorem]{Example}

\begin{document}

\noindent Landon Fox \\
\noindent Math 440 \\
\noindent March 8, 2021

\begin{center}
	\Large Lecture Summary Week 6
\end{center}

\setcounter{lecture_num}{15}
\setcounter{theorem}{0}
\section*{Lecture 15}

\subsection*{Continuous Functions}

\begin{definition}
	Let $(X, \mathcal{T}_X)$ and $(Y, \mathcal{T}_Y)$ be topological spaces. Further, let $x \in X$. A function $f : X \to Y$ is \textbf{continuous at $x$} if and only if for all open
	subsets $V \subseteq Y$ containing $f(x)$, there exists an open subset $U \subseteq X$ such that $x \in U$ and $f(U) \subseteq V$. We say that $f : X \to Y$ is \textbf{continuous} or
	a \textbf{map} if and only if for all $x \in X$, $f$ is continuous at $x$.
\end{definition}

\begin{theorem}
	A function $f : X \to Y$ is continuous if and only if for all open subsets $V \subseteq Y$ it holds that $f^{-1}(V) \subseteq X$ is open.
\end{theorem}

\begin{proposition}
	Let $(X, d_X)$ and $(Y, d_Y)$ be metric spaces and $\mathcal{T}_{d_X}$, $\mathcal{T}_{d_Y}$ their corresponding metric topologies. Then a function $f : X \to Y$ is a map between
	$(X, d_X)$ and $(Y, d_Y)$ if and only if for all $a \in X$ and for all $\epsilon > 0$ there exists a $\delta_a > 0$ such that $d_X(x, a) < \delta_a$ implies that $d_Y\left( f(x),
	f(a) \right) < \epsilon$.
\end{proposition}

\begin{remark}
	\textbf{Proposition 15.3} illustrates that all continuous functions studied from analysis are examples of maps between topological spaces.
\end{remark}


\setcounter{lecture_num}{16}
\setcounter{theorem}{0}
\section*{Lecture 16}

\subsection*{Continuous Functions}

\begin{theorem}
	Let $(X, \mathcal{T}_X)$ and $(Y, \mathcal{T}_Y)$ be topological spaces and consider a function $f : X \to Y$.
	\begin{enumerate}
		\item The function $f$ is continuous if and only if for all closed subsets $B \subseteq Y$, it holds that $f^{-1}(B) \subseteq X$ is also closed.
		\item Let $\mathcal{B}_X$ and $\mathcal{B}_Y$ be bases for $\mathcal{T}_X$ and $\mathcal{T}_Y$, respectively. Then $f$ is continuous if and only if for all $B' \in \mathcal{B}_Y$
		and for all $x \in f^{-1}(B')$ there exists a set $B \in \mathcal{B}_X$ such that $x \in B \subseteq f^{-1}(B')$.
	\end{enumerate}
\end{theorem}

\begin{example}
	Let $(X, \mathcal{T})$ be a topological space and let $A \subseteq X$ be any subset equipped with the subspace topology. Further, let $i_A : A \to X$ be the \textbf{inclusion
	function} defined as $i_A(a) = a$. Then $i_A$ is continuous.
\end{example}

\begin{remark}
	The subspace topology $\mathcal{T}_\mathrm{sub}$ of $A \subseteq X$ can be characterized uniquely as the \textit{smallest} topology on $A$ with the property that $i_A : A \to X$ is
	continuous.
	\begin{enumerate}
		\item Take $A = X$. Then $$i_X = \mathrm{id}_X : X \to X$$ is continuous.
		\item Let $X$ and $Y$ be topological spaces. For a particular $y \in Y$, define $c_y : X \to Y$ as $c_y(x) = y$ for all $x \in X$. Then $c_y$, a \textbf{constant map}, is
			continuous.
	\end{enumerate}
\end{remark}

\begin{proposition}
	If $f : X \to Y$ and $g : Y \to Z$ are continuous functions between spaces $X$, $Y$, and $Z$, then $g \circ f : X \to Z$ is continuous.
\end{proposition}

\begin{corollary}[Restriction of Domain]
	If $f : X \to Y$ is continuous and $A \subseteq X$ is equipped with subspace topology, then the restriction $$f \mid_A = f \circ i_A : A \to Y$$ is continuous.
\end{corollary}

\begin{proposition}
	Let $f : X \to Y$ be continuous. Further, let $B \subseteq Y$ be a subspace. Finally, let $j_B : B \to Y$ be the inclusion and suppose $f(X) \subseteq B$. Then there exists a unique
	continuous function $g : X \to B$ such that $f = j_b \circ g$.
\end{proposition}

\begin{theorem}[The Pasting Lemma]
	Let $X$ and $Y$ be topological spaces. Additionally, let $\mathcal{A} = \{ A_\alpha \}_{\alpha \in \mathcal{I}}$ be a collection of subspaces of $X$ such that $$X = \bigcup_{a \in
	\mathcal{I}} A_\alpha.$$ Suppose $\{ f_\alpha : A_\alpha \to Y \}_{\alpha \in \mathcal{I}}$ is a collection of continuous functions such that $$f_\alpha \mid_{A_\alpha \cap A_\beta}
	= f_\beta \mid_{A_\alpha \cap A_\beta}$$ for all $\alpha, \beta \in \mathcal{I}$.
	If either
	\begin{enumerate}
		\item[i.] $\mathcal{A}$ is a collection of open subspaces of $X$
		\item[ii.] $\mathcal{A}$ is a finite collection of subspaces of $X$
	\end{enumerate}
	then there exists a unique map $f : X \to Y$ such that $f \mid_{A_\alpha} = f_\alpha$ for all $\alpha \in \mathcal{I}$.
\end{theorem}


\setcounter{lecture_num}{17}
\setcounter{theorem}{0}
\section*{Lecture 17}

\subsection*{Homeomorphisms}

Homeomorphisms provide a fundamental measure of \textit{sameness} in topology.

\begin{definition}
	A continuous bijective function $f : X \to Y$ between topological spaces is a \textbf{homeomorphism} if and only if its set-theoretic inverse $f^{-1} : Y \to X$ is continuous. Two
	spaces $X$ and $Y$ are \textbf{homeomorphic} if and only if there exists a homeomorphism $f : X \to Y$ between them; we write $X \cong Y$.
\end{definition}

\begin{remark} $ $
	\begin{enumerate}
		\item For any topological space $X$, $\mathrm{id}_X : X \to X$ is a homeomorphism.
		\item If $f : X \to Y$ is a homeomorphism, then $f^{-1} : Y \to X$ is also a homeomorphism.
		\item Given $$X \overset{f}{\rightarrow} Y \overset{g}{\rightarrow} Z,$$ if $f$ and $g$ are homeomorphisms, then $g \circ f : X \to Z$ is a homeomorphism.
	\end{enumerate}
\end{remark}

\end{document}
