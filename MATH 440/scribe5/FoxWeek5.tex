\documentclass[ 12pt ]{article}
\usepackage{amsmath, amsthm, amssymb, csquotes, enumitem, graphicx, listings, mathrsfs}
\usepackage[margin=0.5in]{geometry}
\graphicspath{ ./ }

\newcounter{lecture_num}
\theoremstyle{plain}

\theoremstyle{plain}
\newtheorem{theorem}{Theorem}[lecture_num]
\newtheorem{proposition}[theorem]{Proposition}
\newtheorem{lemma}[theorem]{Lemma}
\newtheorem{corollary}[theorem]{Corollary}

\theoremstyle{definition}
\newtheorem{definition}[theorem]{Definition}
\newtheorem{notation}[theorem]{Notation}
\newtheorem{observation}[theorem]{Observation}
\newtheorem{question}[theorem]{Question}

\theoremstyle{remark}
\newtheorem{remark}[theorem]{Remark}
\newtheorem{example}[theorem]{Example}

\begin{document}

\noindent Landon Fox \\
\noindent Math 440 \\
\noindent March 1, 2021

\begin{center}
	\Large Lecture Summary Week 5
\end{center}

\setcounter{lecture_num}{12}
\setcounter{theorem}{0}
\section*{Lecture 12}

\subsection*{Basis of a Topology}

\begin{proposition}
	Let $X$ be a set. Let $\mathcal{B}$ be a basis for a topology on $X$. Then the following holds.
	\begin{enumerate}
		\item The collection of subsets $$\mathcal{T}_\mathcal{B} = \{ U \subset X : \forall x \in U\; \exists B \in \mathcal{B}\; \mathrm{s.t.}\; x \in B \subseteq U \}$$ is a topology
			on $X$ labeled the \textbf{topology generated by $\mathcal{B}$}.
		\item A set $U$ is a subset $U \subseteq (X, \mathcal{T}_\mathcal{B})$ if and only if $U$ is a union of elements in $\mathcal{B}$.
	\end{enumerate}
\end{proposition}

Now we ask a question. Given a topological space $(X, \mathcal{T})$, can we find a basis $\mathcal{B} \subseteq \mathcal{T}$ such that $\mathcal{T}_\mathcal{B} = \mathcal{T}$?

\begin{proposition}
	If $(X, \mathcal{T})$ is a topological space and $\mathcal{B} \subseteq \mathcal{T}$ such that for all $U \in \mathcal{T}$ and for all $x \in U$ there exists $B \in \mathcal{B}$ such
	that $x \in B \subseteq U$, then $\mathcal{B}$ is a basis and $\mathcal{T}_\mathcal{B} = \mathcal{T}$
\end{proposition}

\begin{example} $ $
	\begin{itemize}
		\item The Euclidean topology on $\mathbb{R}$ has a basis $$\mathcal{B}_\mathrm{euc} = \{ (a, b) : a < b \in \mathbb{R} \}.$$
		\item The lower limit topology on $\mathbb{R}$ has a basis $$\mathcal{B}_\ell = \{ [a, b) : a < b \in \mathbb{R} \}.$$
		\item The upper limit topology on $\mathbb{R}$ has a basis $$\mathcal{B}_u = \{ (a, b]: a < b \in \mathbb{R} \}.$$
		\item The $K$-topology on $\mathbb{R}$ has a basis $$\mathcal{B}_K = \mathcal{B}_\mathrm{euc} \cup \{ (a, b) \setminus K : a < b \in \mathbb{R} \}$$ where $$K = \left \{
			\frac{1}{n} : n \in \mathbb{N} \right \}.$$
	\end{itemize}
\end{example}

\begin{theorem}[Comparison Lemma]
	Let $\mathcal{T}, \mathcal{T}'$ be topologies on $X$. Further, let $\mathcal{B}, \mathcal{B}'$ be bases for $\mathcal{T}, \mathcal{T}'$, respectively. Then $\mathcal{T} \subseteq
	\mathcal{T}'$ if and only if for all $x \in X$ and for all $B \in \mathcal{B}$ containing $x$, there exists $B' \in \mathcal{B}'$ such that $x \in B' \subseteq B$.
\end{theorem}


\setcounter{lecture_num}{13}
\setcounter{theorem}{0}
\section*{Lecture 13}

\subsection*{Comparing Topologies}

\begin{example}
	Can we compare $\mathcal{T}_\mathrm{euc}$ and $\mathcal{T}_\ell$? In fact, we can. By using Theorem 12.4, it can be shown that $\mathcal{T}_\mathrm{euc} \leq \mathcal{T}_\ell$ but
	$\mathcal{T}_\mathrm{euc} \ngeq \mathcal{T}_\ell$
\end{example}


\setcounter{lecture_num}{14}
\setcounter{theorem}{0}
\section*{Lecture 14}

\subsection*{2nd Countable Spaces}

The basis for a topology is not unique. Sometimes we can find \textit{smaller} ones.

\begin{example}
	Consider the Euclidean topology $\mathcal{T}_\mathrm{euc}$ on $\mathbb{R}$. We know that $$\mathcal{B}_\mathrm{euc} = \{ (a, b) : a < b \in \mathbb{R} \}$$ is a basis of
	$\mathcal{T}_\mathrm{euc}$. On the other hand, consider $$\widetilde{\mathcal{B}} = \{ (p, q) : p < q \in \mathbb{Q} \} \subset \mathcal{B}_\mathrm{euc}.$$ It can be shown that
	$\widetilde{\mathcal{B}}$ is a basis of $\mathcal{T}_\mathrm{euc}$. Additionally, there exists an injective function $\widetilde{\mathcal{B}} \to \mathbb{Q} \times \mathbb{Q}$
	implying that $|\widetilde{\mathcal{B}}| \leq |\mathbb{Q} \times \mathbb{Q}|$ and so $\mathcal{\mathcal{B}}$ is countable.
\end{example}

\begin{definition}
	A topological space $(X, \mathcal{T})$ is \textbf{2nd countable} if and only if there exists a countable basis $\mathcal{B} = \{ B_i \}_{i \geq 1}$ for the topology $\mathcal{T}$.
\end{definition}

\begin{example}
	Consider $(\mathbb{R}, \mathcal{T}_\mathrm{disc})$. Every subset of $\mathbb{R}$ is open and so $(\mathbb{R}, \mathcal{T}_\mathrm{disc})$ is not 2nd countable.
\end{example}

\begin{theorem}
	If $(X, \mathcal{T})$ is 2nd countable, then $(X, \mathcal{T})$ is separable. The converse of is, however, false.
\end{theorem}

\begin{proposition}
	Consider the lower limit topology $(\mathbb{R}, \mathcal{T}_\ell)$.
	\begin{enumerate}
		\item $(\mathbb{R}, \mathcal{T}_\ell)$ is separable.
		\item $(\mathbb{R}, \mathcal{T}_\ell)$ is not 2nd countable.
	\end{enumerate}
\end{proposition}

\begin{theorem}
	Let $(X, d)$ be a metric space. Let $\mathcal{T}_d$ be the metric topology on $X$ induced by $d$. If $(X, \mathcal{T}_d)$ is separable, then it is 2nd countable.
\end{theorem}

\end{document}
