\documentclass[ 12pt ]{article}
\usepackage{amsmath, amsthm, amssymb, csquotes, enumitem, graphicx, listings, mathrsfs}
\usepackage[margin=0.5in]{geometry}
\graphicspath{ ./ }

\newcounter{lecture_num}
\theoremstyle{plain}

\theoremstyle{plain}
\newtheorem{theorem}{Theorem}[lecture_num]
\newtheorem{proposition}[theorem]{Proposition}
\newtheorem{lemma}[theorem]{Lemma}
\newtheorem{corollary}[theorem]{Corollary}

\theoremstyle{definition}
\newtheorem{definition}[theorem]{Definition}
\newtheorem{notation}[theorem]{Notation}
\newtheorem{observation}[theorem]{Observation}
\newtheorem{question}[theorem]{Question}

\theoremstyle{remark}
\newtheorem{remark}[theorem]{Remark}
\newtheorem{example}[theorem]{Example}

\begin{document}

\noindent Landon Fox \\
\noindent Math 440 \\
\noindent April 26, 2021

\begin{center}
	\Large Lecture Summary Week 13
\end{center}

\setcounter{lecture_num}{33}
\setcounter{theorem}{0}
\section*{Lecture 33}

\subsection*{Building Connected Spaces}

\begin{lemma}
	If $f : X \overset{\cong}{\to} Y$ is a homeomorphism, and $C \subseteq X$ is a subspace then the restriction $f \mid_C : C \overset{\cong}{\to} f(C)$ is a homeomorphism.
\end{lemma}

\begin{remark}
	The unit circle omitting two distinct points is disconnected. That is, $S^1 \setminus \{ x, y \}$ for any $x \neq y \in S^1$ is disconnected.
\end{remark}

\begin{lemma}
	Let $X$ be a topological space and $C \subseteq X$ is an open and closed subset. If $A \subseteq X$ is a connected subspace, then either $A \subseteq C$ of $A \subseteq X \setminus C$.
\end{lemma}

\begin{theorem}
	Let $X$ be a topological space and $A, B \subseteq X$ are connected subspaces. If $A \cap B \neq \varnothing$, then $A \cup B$ is connected.
\end{theorem}

\begin{remark}
	If $A, B \subseteq X$ are connected and $A \cap B \neq \varnothing$, then $A \cap B$ is not necessarily connected.
\end{remark}


\setcounter{lecture_num}{34}
\setcounter{theorem}{0}
\section*{Lecture 34}

\subsection*{Building Connected Spaces}

\begin{definition}
	Let $X$ be a topological space. For any $x, y \in X$, we write $x \sim y$ if and only if there exists a connected subspace $A \subseteq X$ such that $x, y \in A$. It can be shown that $\sim$ is an equivalence relation on $X$. Equivalence classes of $\sim$ are called the \textbf{connected components of $X$}.
\end{definition}

\begin{example} $ $
	\begin{itemize}
		\item If $X$ is nonempty and connected, then $X$ has one connected component.
		\item If $X$ has the discrete topology, then the components of $X$ are the singletons.
	\end{itemize}
\end{example}

\begin{theorem}
	The connected components of $X$ are the \textbf{maximal connected} subspaces of $X$.
\end{theorem}

\begin{theorem}
	If $X$ is a topological space and $A \subseteq X$ is nonempty and connected, then its closure $\overline{A}$ is connected.
\end{theorem}

\begin{corollary}
	If $A \subseteq X$ is connected compact, then $A$ is closed.
\end{corollary}

\begin{remark}
	Connected components are not necessarily open subsets of $X$.
\end{remark}

\begin{corollary}
	If $A \subseteq X$ connected and $B \subseteq X$ is a subset such that $A \subseteq B \subseteq \overline{A}$, then $B$ is connected.
\end{corollary}

\begin{theorem}
	If $X$ and $Y$ are nonempty connected spaces, then $X \times Y$ is connected.
\end{theorem}


\end{document}
